\documentclass[../electromagnetism.tex]{subfiles}
\begin{document}
\section{Basic Laws}
\subsection{Kirchhoff Laws}
Consider a finite cable as a cylinder. Said $V$ our cylinder we have that if the charge distribution $\rho$ doesn't change with time, due to the charge continuity equation we also must have that the divergence of the volumetric charge density doesn't change, therefore
\begin{equation*}
	\nabla\cdot\vec{J}=\pdv{\rho}{t}=0
\end{equation*}
Integrating on the cylinder therefore
\begin{equation*}
	\iiint_V\nabla\cdot\vec{J}\dd^3{x}=\oiint_S\vec{J}\cdot\ver{n}\dd^2{s}=0
\end{equation*}
Where $S$ is the surface of the cylinder. It's clear that $S=S_1\cup S_2\cup S_3$ where $S_1,S_2$ are the two surfaces at the ends of the cylinder and $S_3$ is the tube itself.\\
By definition, since the normal of the tube is orthogonal to $\vec{J}$ that
\begin{equation*}
	\oiint_S\vec{J}\cdot\ver{n}\dd^2{s}=\iint_{S_1}\vec{J}\cdot\ver{n}_1\dd^2{s}+\iint_{S_2}\vec{J}\cdot\ver{n}_2\dd^2{s}
\end{equation*}
Thus, remembering that the current $I$, when $\partial_t\rho=0$, is exactly defined as the flux of $\vec{J}$, that 
\begin{equation*}
	I=\iint_S\vec{J}\cdot{n}\dd^2{s}
\end{equation*}
And therefore, said $i_1$ and $i_2$ the currents flowing through the surface 1 and 2, that
\begin{equation}
	I=i_1+i_2=0
	\label{eq:kirchhoffintro.dc}
\end{equation}
This indicates that the current in a circuit is \textit{surface independent}.\\
\begin{dfn}[Node]
	A node is the point of convergence of multiple conductors. Taken a surface $S$ that contains the node, then 
	\begin{equation*}
		\oiint_S\vec{J}\cdot\ver{n}\dd^2{s}=0
	\end{equation*}
\end{dfn}
By definition of node then, it's possible to define the following law
\begin{thm}[First Kirchhoff Law]<++>
\end{thm}<++>
\section{Resistors}
\subsection{Substitution Principle}
\subsection{Multiple Generators}
\subsection{Tripolar Configurations}
\section{Nodal and Mesh Analysis}
\subsection{Nodal Analysis}
\subsection{Mesh Analysis}
\section{Equivalent Circuits}
\subsection{Thevenin Theorem}
\subsection{Norton Theorem}
\subsection{Real Generators}
\section{Capacitors and Inductors}
\subsection{Definitions}
\subsection{Capacitor Circuits}
\subsection{Inductor Circuits}
\end{document}
%47/139 pg
