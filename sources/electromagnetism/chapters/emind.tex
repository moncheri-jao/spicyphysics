\documentclass[../electromagnetism]{subfiles}
\begin{document}
\section{Faraday's Law}
So far we managed to build four equations for the two fields, in \textit{static} conditions. These are, whenever there no dielectrics and no magnets, are
\begin{equation}
	\left\{\begin{aligned}
			\del_iE^i&=\frac{\rho}{\epsilon_0}\\
			\del_iB^i&=0\\
			\cpr{i}{j}{k}\del^jE^k&=0\\
			\cpr{i}{j}{k}\del^jB^k&=\mu_0J^i
	\end{aligned}\right.
	\label{eq:mwstatic}
\end{equation}
Faraday in his works had a major discovery while using two simple circuits.\\
Consider a closed circuit A with only a galvanometer, and a circuit B with a battery and a switch.\\
Using the switch for controlling the current flow on the circuit B (which is NOT connected to circuit A), he saw that the galvanometer measures a current while the current in B is changing.\\
Taking the same setup but keeping the switch closed on B, if either of the two circuits are in motion, there is current flow in A. The same happens if circuit B is substituted by a magnet!\\
Now take a non-rigid circuit immersed in a region with a constant magnetic field $B^i$. Deforming the circuit will also induce a current flow on it.\\
Calling $f_{em}$ the \textit{electromotive force} that drives the current in circuit A, Faraday deduced experimentally that
\begin{equation}
	f_{em}=-\dv{\Phi}{t}
	\label{eq:faradayneumann}
\end{equation}
Where $\Phi$ is the magnetic flux passing inside the circuit.\\
Note that we know, by definition, that this electromotive force must be driven by an electric field $E^i$, where
\begin{equation}
	f_{em}=\oint_CE^i_{ind}\hat{t}_i\dd l
	\label{eq:fem}
\end{equation}
This field \emph{can't} be conservative! Using the definition of the electric field as force per unit charge, and using Lorentz's force law we have a little hint that this induced field is the sum of a pure electric field plus a second field generated by the movement of charges
\begin{equation}
	E^i_{ind}=E^i+\cpr{i}{j}{k}v^jB^k
	\label{eq:faradayneumannindfield}
\end{equation}
Noting that the charges are constrained to the circuit, we can divide the velocity $v^i$ with a component parallel to the circuit $v^i_{\parallel}$ and a perpendicular component $v^i_{\perp}$. It's obvious then that
\begin{equation}
	f_{em}=\oint_C\left(E^i+\cpr{i}{j}{k}v^j_{\perp}B^k\right)\hat{t}_i\dd l
	\label{eq:femind}
\end{equation}
Suppose now that we do not move the circuit, then $v^i=v^i_{\parallel}$ and $E^i_{ind}=E^i$, where this electric field is for sure not conservative.\\
All this jargon, condenses itself in one simple but powerful law, \textit{Faraday-Neumann-Lenz's law}, which indicates exactly what Faraday discovered experimentally
\begin{thm}[Faraday-Neumann-Lenz, Electromagnetic Induction]
    Given a time-dependent magnetic field $B^i(t,x^i)$, an electric field is induced by the variation of its flux, where
    \begin{equation*}
        \oint_{\del S}E^i\hat{t}_i\dd l=-\dv{t}\iint_{S}B^i\hat{n}_i\dd s
    \end{equation*}
    Or, in its differential counterpart
    \begin{equation}
        \cpr{i}{j}{k}\del^jE^k=-\pdv{B^i}{t}
    \label{eq:fnldiff}
    \end{equation}
\end{thm}
\begin{proof}
    Suppose that there is some circuit $\del S$ that spans some surface $S$ inside of it, which is immersed in a time dependent magnetic field $B^i(t)$, then \eqref{eq:faradayneumann} holds, and therefore
    \begin{equation*}
	\dv{\Phi}{t}=\dv{t}\iint_{S(t)}B^i(x,t)\hat{n}_i\dd s
    \end{equation*}
    The derivative on the right can be seen as the variation of the surface $S(t)$ when the $B$ field is fixed in time at $t_0$, plus the integral over the surface $S(t_0)$ of the derivative of $B$ with respect to time, i.e.
    \begin{equation*}
	\dv{\Phi}{t}=\dv{t_0}\iint_{S(t)}B^i(x,t_0)\hat{n}_i\dd s+\iint_{S(t_0)}\pdv{B^i}{t}\hat{n}_i\dd s
    \end{equation*}
    Since we know already that \eqref{eq:faradayneumann} holds, we have that if $f_{ind}$ is the induced $f_{em}$, we have that
    \begin{equation*}
	f_{ind}=\oint_{\del S(t_0)}E^i\hat{t}_i\dd l=\oint_{\del S(t_0)}E^i\hat{t}_i\dd l=\iint_{S(t_0)}\pdv{B^i}{t}\hat{n}_i\dd s
    \end{equation*}
    But
    \begin{equation}
        \dd\Phi=\iint_SB^i\dv{s}{t}\dd t=\int\oint_{\del S}\epsilon_{ijk}B^iv^j_{D}\hat{t}^k\dd l\dd t
    \end{equation}
    Where $v_D$ is the velocity in the direction of the movement of the circuit. Therefore, ``dividing'' by $\dd t$, we get
    \begin{equation*}
	\dv{\Phi}{t}=\dv{t}\iint_{S(t)}B^i(x,t_0)\hat{n}_i\dd s=\oint_{\del S}\cpr{i}{j}{k}\hat{t}_iv^j_DB^k\dd l
    \end{equation*}
    Since we know that an additional $f_{ind}$ is given by the deformation of the circuit $\del S(t_0)\to\del S(t)$, we have in total
    \begin{equation*}
	f_{em}=-\dv{\Phi}{t}=\oint_{\del S}E^i+\cpr{i}{j}{k}\hat{t}_iv^j_DB^k\dd l
    \end{equation*}
    Since $v_D=v_\parallel+v_\perp$ and $v_\parallel\parallel\hat{t}$ we get that since $\vec{E}_{ind}=\vec{E}+\vec{v}\times\vec{B}$, $\vec{E}_{ind}$ has $\vec{B}$ as a source, and we can condense it all in a single integro-differential equation, which is Faraday-Neumann-Lenz's law
    \begin{equation*}
	\nabla\times\vec{E}\cdot\hat{\vec{n}}\dd s=-\dv{t}\iint_S\pdv{\vec{B}}{t}\cdot\hat{\vec{n}}\dd s
    \end{equation*}
\end{proof}
\subsection{Self-Induction}
Having deduced our previous results, the first thing that we might check is how a circuit behaves with itself.\\
Consider a closed circuit with some current $I(t)$ such that $\del_tI\approx0$. If the magnetic permeability of the body is constant we can apply Biot-Savart for evaluating the field, and we have
\begin{equation}
    B^i(t)=\frac{\mu_0I(t)}{4\pi}\oint_C\frac{\cpr{i}{j}{k}\hat{t}^jr^k}{r^3}\dd l
    \label{eq:bsautoind}
\end{equation}
Evaluating the flux of this field we have, since the current is independent from the integrated variables, that $\Phi\propto I(t)$, and therefore
\begin{equation*}
	\Phi(t)=LI(t)\qquad L=\frac{\mu_0}{4\pi}\oint_C\iint_S\frac{\cpr{i}{j}{k}\hat{t}^jr^k}{r^3}\dd l\dd s
\end{equation*}
The constant $L$ only depends on the geometry of the circuit as it's easy to see from the integral, and it's known as the \textit{self-induction coefficient} or also as \textit{autoinduction coefficient}.\\
From Faraday's law, since this flux depends on time (through our current $I(t)$), it generates an electromotive force $f_L$, as follows
\begin{equation}
	f_L=-\dv{\Phi}{t}=-L\dv{I}{t}
	\label{eq:emforce}
\end{equation}
The autoinduction coefficient has units the following units:
\begin{equation*}
	[L]=\frac{[\Phi]}{[I]}=\frac{W}{A}=\frac{Vs}{A}=\Omega s=H
\end{equation*}
The SI unit $H$ is known as \textit{Henry} and it's equal to Watts/Ampere. Note that this can also be calculated via Ohm's law, noting that $V=RI$ and that $[RI]=\Omega A$
\subsection{Mutual-Induction}
Consider now a setup similar to the previous one, but with two circuits $C_1$ and $C_2$, which are close enough to each other such that the generated magnetic fluxes through each circuit are not negligible.\\
The fluxes as before will be proportional to the currents, and without evaluating the self-induction of both circuits we have
\begin{equation}
	\begin{aligned}
		\Phi_1(B_2)&=\iint_{S_1}B_2^i\hat{n}_i\dd s\propto I_2(t)\\
		\Phi_2(B_1)&=\iint_{S_2}B_1^i\hat{n}_i\dd s\propto I_1(t)
	\end{aligned}
	\label{eq:fluxescircuitsmind}
\end{equation}
Using the previous considerations, we have then, in index form, that
\begin{equation}
	\Phi_i=M_{ij}I_j(t)
	\label{eq:mutualind}
\end{equation}
The coefficients $M_{ij}$ are known as the \textit{mutual induction coefficients}. Obviously $M_{ij}=M_{ji}$.
\section{Magnetic Energy}
With what we wrote before, we might consider a circuit with a given self-induction coefficient $L$ and some time-dependent current $I(t)$ flowing through it. We can evaluate the work of that the magnetic force exerts on these charges as follows. Per unit time
\begin{equation}
	\dv{w}{t}=-f_{em}I(t)
	\label{eq:work/timemag}
\end{equation}
Using Faraday's law we know that
\begin{equation*}
	\dv{\Phi}{t}=-f_{em}=-L\dv{I}{t}
\end{equation*}
Therefore 
\begin{equation}
	\dv{w}{t}=L\left( \dv{I}{t} \right)^2=\frac{1}{2}LI^2(t)
	\label{eq:mfenfaraday1}
\end{equation}
We can go forward with this calculus, noting that then, since $\Phi=LI$, and 
\begin{equation*}
	\Phi=\iint\cpr{i}{j}{k}\del^jA^k\hat{n}_i\dd s
\end{equation*}
Then, using Stokes' theorem
\begin{equation*}
	\oint A^i\hat{t}_i\dd l=LI
\end{equation*}
And therefore, vectorizing the current as $I^i=I\hat{t}^i$
\begin{equation*}
	w=\frac{1}{2}\oint A^iI_i\dd l
\end{equation*}
Or in general
\begin{equation}
	w=\frac{1}{2}\iiint A^iJ_i\dd^3x=\frac{1}{2\mu_0}\iiint\cpr{i}{j}{k}A_i\del^jB^k\dd^3x
	\label{eq:workmagnetic}
\end{equation}
Where we used Ampere's law to get the last integral. Playing around with the last curl, using vector notation, we have
\begin{equation*}
	\nabla\cdot(\vec{A}\times\vec{B})=\vec{B}\cdot\nabla\times\vec{A}-\vec{A}\cdot\nabla\times\vec{B}=B^2-\vec{A}\cdot\nabla\times\vec{B}
\end{equation*}
Applying Stokes' theorem on the divergence and noting that the surface integral goes to 0 when we integrate over all space, we end up with the following result
\begin{equation*}
	w=\frac{1}{2\mu_0}\iiint_{\R^3}B^2\dd^3x
\end{equation*}
The parallelism with the energy of an electric field is astounding. Written side by side we have
\begin{equation}
	\begin{aligned}
		W_{es}&=\frac{1}{2}\int_{\R^3}V\rho\dd^3x=\frac{\epsilon_0}{2}\int_{\R^3}E^2\dd^3x\\
		W_{ms}&=\frac{1}{2}\int_{\R^3}A^iJ_i\dd^3x=\frac{1}{2\mu_0}\int_{\R^3}B^2\dd^3x
	\end{aligned}
	\label{eq:esmsenergy}
\end{equation}
\section{Maxwell's Equations}
So far, we found 2 pairs of coupled differential equations for the electric and magnetic field
\begin{equation}
	\left\{ \begin{aligned}
			\del_iE^i&=\frac{\rho}{\epsilon_0}\\
			\cpr{i}{j}{k}\del^jE^k&=-\del_{t}B^i
	\end{aligned}\right.
	\label{eq:mwcouple1}
\end{equation}
And
\begin{equation}
	\left\{ \begin{aligned}
		\del_iB^i&=0\\
		\cpr{i}{j}{k}\del^jB^k&=\mu_0J^i
	\end{aligned}\right.
	\label{eq:mwcouple2}
\end{equation}
The second pair of equations holds only if the current field is divergenceless, but using Gauss' law and the current conservation equation we have
\begin{equation}
	\del_iJ^i+\del_t\rho=\del_iJ^i+\epsilon_0\del_t\del_iE^i=0
	\label{eq:cf+cc}
\end{equation}
Grouping the divergences we see that the time derivative of the electric field behaves exactly like a current, commonly called the ``displacement current''. In order to fix all the equations now we can add this new current in the last couple of the Maxwell equations and get the well known fundamental equations of electromagnetism
\begin{equation}
	\left\{\begin{aligned}
		\del_iE^i&=\frac{\rho}{\epsilon_0}\\
		\cpr{i}{j}{k}\del^jE^k&=-\del_tB^i\\
		\del_iB^i&=0\\
		\cpr{i}{j}{k}\del^jB^k&=\mu_0J^i+\epsilon_0\mu_0\del_tE^i
	\end{aligned}\right.
	\label{eq:maxwellequationsfull}
\end{equation}
These equations account for moving charges and are absolutely general in nature. It will be seen later that they're also Lorentz invariant, therefore they preserve between Lorentz transformations and therefore are relativistically covariant.\\
For linear dielectric and magnetic media we can rewrite easily with the already known rules, Maxwell's equations
\begin{equation}
	\left\{\begin{aligned}
			\del_iD^i&=\rho\\
			\cpr{i}{j}{k}\del^jD^k&=\epsilon\mu\del_tH^i\\
			\del_iH^i&=0\\
			\cpr{i}{j}{k}\del^jH^k&=J^i+\del_tD^i
	\end{aligned}\right.
	\label{eq:maxwellfulllinearmedia}
\end{equation}
\subsection{Poynting's Vector, Energy Conservation}
As we have seen via previous calculations, the energies of the two separated fields are, calling them $U_e$ and $U_m$
\begin{equation}
	\begin{aligned}
		U_e&=\frac{\epsilon_0}{2}\int_VE^2\dd^3x\\
		U_m&=\frac{1}{2\mu_0}\int_VB^2\dd^3x
	\end{aligned}
	\label{eq:fieldenergies}
\end{equation}
We can imagine that the energy of the combined electromagnetic field will be a sum of the two, i.e.
\begin{equation}
	U_{em}=\frac{1}{2}\int_V\epsilon_0E^2+\frac{B^2}{\mu_0}\dd^3x
	\label{eq:emfieldenergy}
\end{equation}
We try to confirm this using the work done by some particle. Substituting into the force the formula for Lorentz's force we get
\begin{equation*}
	F^i\dd l_i=q\left( E^i+\cpr{i}{j}{k}v^jB^k \right)\dd l_i=q\left( E^i+\cpr{i}{j}{k}v^jB^k \right)v_i\dd t=qE^iv_i\dd t=\dd W
\end{equation*}
Which is what we expected.\\
Going to a microscopic consideration we substitute $q=\rho\dd^3x$, we get $J^i=\rho v^i$, and therefore, integrating with respect to time we have
\begin{equation*}
	W=\int_VE^iJ_i\dd^3x\dd t\implies E^iJ_i=\dv{w}{t}
\end{equation*}
But, from Maxwell equations we have
\begin{equation*}
	\cpr{i}{j}{k}\del^jB^k=\mu_0J^i+\frac{1}{c^2}\pdv{E^i}{t}\implies J^i=\frac{\cpr{i}{j}{k}\del^jB^k}{\mu_0}-\frac{1}{\mu_0c^2}\pdv{E^i}{t}
\end{equation*}
Where we used $\mu_0\epsilon_0=c^-2$ (it will be clear later, for now check multiplying those two and see that it adds up)\\
Therefore, we rewrite $E^iJ_i$ as follows
\begin{equation*}
	E^iJ_i=\frac{1}{\mu_0}E_i\cpr{i}{j}{k}\del^jB^k-\epsilon_0E_i\cpr{i}{j}{k}\del^jB^k
\end{equation*}
Note that, tho:
\begin{equation*}
	\del_i\cpr{i}{j}{k}E^j\frac{B^k}{\mu_0}=\frac{1}{\mu_0}B_i\cpr{i}{j}{k}\del^jE^k-E^i\cpr{i}{j}{k}\del^j\frac{B^k}{\mu_0}
\end{equation*}
Therefore
\begin{equation*}
	E_i\cpr{i}{j}{k}\del^j\frac{B^k}{\mu_0}=-\frac{1}{\mu_0}B_i\cpr{i}{j}{k}\del^jE^k-\del_i\cpr{i}{j}{k}E^j\frac{B^k}{\mu_0}=-\frac{B_i}{\mu_0}\pdv{B^i}{t}-\del_i\cpr{i}{j}{k}E^j\frac{B^k}{\mu_0}
\end{equation*}
Which, gives us back
\begin{equation}
	E^iJ_i=-\frac{1}{2}\pdv{t}\left( \epsilon_0E^2+\frac{1}{\mu_0}B^2 \right)-\frac{1}{\mu_0}\del_i\cpr{i}{j}{k}E^jB^k=\dv{w}{t}
	\label{eq:diffworkpoynting}
\end{equation}
We immediately recognize the volumetric density of energy of the electromagnetic field, let's denote it as $u_{em}$, and we get
\begin{equation}
	E^iJ_i=-\pdv{u_{em}}{t}-\del_i\cpr{i}{j}{k}E^j\left( \frac{B^k}{\mu_0} \right)
	\label{eq:poynting-1}
\end{equation}
We begin to have a better view of the phenomenon, we see a variation of energy on the right plus the divergence of some vector that we define now.
\begin{dfn}[Poynting Vector]
	The \textit{Poynting vector} is a vector defined as follows:
	\begin{equation}
		S^i=\frac{1}{\mu_0}\cpr{i}{j}{k}E^jB^k
		\label{eq:poyntingdef}
	\end{equation}
	It has dimensions of a flux of energy, as we will see.
\end{dfn}
With the previous definition, everything becomes much clearer in terms of notation, in fact
\begin{equation*}
	E^iJ_i=-\pdv{u_{em}}{t}-\del_iS^i
\end{equation*}
Integrating in a random volume $V$ we get, as said before, our flux of energy!
\begin{equation*}
	\int_VE^iJ_i\dd^3x=-\pdv{t}\int_Vu_{em}\dd^3x - \oiint_{\del V}S^i\hat{n}_i\dd s
\end{equation*}
It's clear that in order to make sense we must sum energies with energies, giving the previously stated dimensions of the Poynting vector as an energy flux.\\
Rewriting $E^iJ_i$ as our work variation we have that it's nothing else than the time derivative of the volumetric density of mechanical energy, and writing $u_{em}+u_{mech}$ as our total energy variation, we have
\begin{equation}
	\int_{V}^{}\pdv{u}{t}\dd^3{x}=-\oiint_{\del V}S^i\hat{n}_i\dd s
	\label{eq:energyvariationpoynting}
\end{equation}
The associated PDE is clearly the conservation of energy of the whole system
\begin{equation}
	\pdv{u}{t}+\del_iS^i=0
	\label{eq:energyconspoynting}
\end{equation}
This shape also gives the real idea of what's Poynting's vector: an energy ``current'' 
\subsection{Stress Tensor, Momentum Conservatipon}
Lorentz'force as we have seen, in terms of microscopic evaluations is written (it's a force density in this case) as
\begin{equation}
	f^i=\rho E^i+\cpr{i}{j}{k}J^jB^k
	\label{eq:lorentzforcedensity}
\end{equation}
We rewrite it in terms of fields only using the two following Maxwell equations
\begin{equation}
	\begin{aligned}
		\del_iE^i&=\frac{\rho}{\epsilon_0}\quad\implies\quad\rho=\frac{1}{\epsilon_0}\del_iE^i\\
		\cpr{i}{j}{k}\del^jB^k&=\mu_0J^i+\frac{1}{c^2}\pdv{E^i}{t}\quad\implies\quad J^i=\frac{1}{\mu_0}\cpr{i}{j}{k}\del^jB^k-\frac{1}{\mu_0c^2}\pdv{E^i}{t}
	\end{aligned}
	\label{eq:currentandchargeasfields}
\end{equation}
Therefore, Lorentz's force becomes, using $\epsilon_0=(\mu_0c^2)^{-1}$
\begin{equation*}
	f^i=\epsilon_0E^i\del_jE^j+\cpr{i}{j}{k}\left( \frac{1}{\mu_0}\cpr{j}{m}{l}\del^lB^m-\epsilon_0\pdv{E^j}{t} \right)B^k
\end{equation*}
Or, moving inside the cross product for clarity
\begin{equation*}
	f^i=\epsilon_0E^i\del_jE^j+\frac{1}{\mu_0}\cpr{i}{j}{k}B^k\cpr{j}{l}{m}\del^lB^m-\epsilon_0\cpr{i}{j}{k}\del_t{E^j}B^k
\end{equation*}
Using the product rule on the time derivative at the last factor we have that
\begin{equation*}
	\del_t\cpr{i}{j}{k}E^jB^k=\cpr{i}{j}{k}\del_t(E^j)B^k+\cpr{i}{j}{k}E^j\del_t(B^k)\implies\cpr{i}{j}{k}\del_t(E^j)B^k=\del_t\cpr{i}{j}{k}E^jB^k-\cpr{i}{j}{k}E^j\del_t(B^k)
\end{equation*}
From the second Maxwell equation tho we have
\begin{equation*}
	\cpr{i}{j}{k}\del^jE^k=-\del_tB^i
\end{equation*}
Therefore
\begin{equation*}
	-\cpr{i}{j}{k}E^j\del_t(B^k)=\cpr{i}{j}{k}E^j\cpr{k}{l}{m}\del^lB^m
\end{equation*}
And everything comes back to
\begin{equation*}
	\cpr{i}{j}{k}\del_t(E^j)B^k=\del_t\left( \cpr{i}{j}{k}E^jB^k \right)+\cpr{i}{j}{k}E^j\cpr{k}{l}{m}\del^lE^m
\end{equation*}
And inserting it back into the Lorentz force density gives
\begin{equation}
	f^i=\epsilon_0E^i\del_jE^j+\frac{1}{\mu_0}\cpr{i}{j}{k}B^k\cpr{j}{l}{m}\del^lB^m-\epsilon_0\pdv{t}\left( \cpr{i}{j}{k}E^jB^k \right)+\epsilon_0\cpr{i}{j}{k}E^j\cpr{k}{l}{m}\del^lE^m
	\label{eq:forcedensitylorentz1}
\end{equation}
Or, rearranging the two fields
\begin{equation}
	f^i=\epsilon_0\left( E^i\del_jE^j-\cpr{i}{j}{k}\cpr{k}{l}{m}E^j\del^lE^m \right)+\frac{1}{\mu_0}\left( B^i\del_jB^j-\cpr{i}{j}{k}\cpr{j}{l}{m}\del^lB^mB^k \right)-\epsilon_0\pdv{t}\left( \cpr{i}{j}{k}E^jB^k \right)
	\label{eq:lfdrearr}
\end{equation}
Where we used $\del_iB^i=0$ in order to symmetrize the shape of the equation.\\
Using the properties of the Levi-Civita symbol we have that (note that $g_{ij}=\delta_{ij}$ in this metric)
\begin{equation}
	\cpr{i}{j}{k}\cpr{k}{l}{m}=\delta^i_l\delta_{jm}-\delta_{jl}\delta^i_l
	\label{eq:levicivitaproductrule}
\end{equation}
We have (using a generic vector here, it can be either $B^i$ or $E^i$)
\begin{equation*}
	\cpr{i}{j}{k}\cpr{k}{l}{m}A^j\del^lA^m=A_m\del^iA^m-A^l\del_lA^i
\end{equation*}
So
\begin{equation*}
	f^i=\epsilon_0\left( E^i\del_jE^j+E^l\del_lE^i-E^m\del^iE_m \right)+\frac{1}{\mu_0}\left( B^i\del_jB^j+B^l\del_lB^i-B^m\del^iB_m \right)-\frac{1}{c^2}\pdv{S^i}{t}
\end{equation*}
Looking closely we see that the big mess inside the parentheses is simply
\begin{equation*}
	\del_j\left( E^iE^j \right)-\frac{1}{2}\del_j\left( E^kE_k \right)=\del_j\left( E^iE^j-\frac{1}{2}\delta^{ij}E^kE_k \right)
\end{equation*}
Seen this, we define the following symmetric rank-2 tensor $\sigma^{ij}$
\begin{dfn}[Maxwell Stress Tensor]
	The Maxwell stress tensor is defined as follows:
	\begin{equation}
		\sigma^{ij}=\epsilon_0\left( E^iE^j-\frac{1}{2}\delta^{ij}E^kE_k \right)+\frac{1}{\mu_0}\left( B^iB^j-\frac{1}{2}\delta^{ij}B^kB_k \right)
	\end{equation}
	It's a rank 2 tensor and it's obviously symmetric 
\end{dfn}
The Lorentz force density becomes then the following simply
\begin{equation}
	f^i=\pdv{\sigma^{ij}}{x^k}-\frac{1}{c^2}\pdv{S^i}{t}
	\label{eq:lorentzforce}
\end{equation}
%END OF CHAPTER
\end{document}
