\documentclass[../electromagnetism]{subfiles}
\begin{document}
\section{Electric Forces and the Electric Field}
It has been found from the forefathers of Electrodynamics that, empirically, the force exerted between two charged objects has the following characteristics
\begin{enumerate}
\item It's directed from one object to the other
\item It depends on the \textit{product} of the charges
\item It's proportional to the inverse squared of the distance between the objects $d^{-2}$
\end{enumerate}
The experimental results brought with great clarity then, that for two point charges $q_1,q_2$, said $r=\norm{\vec{r}_1-\vec{r}_2}$ and $\hat{\vec{r}}$ the associated versor, the electrostatic force is
\begin{equation}
	\vec{F}=k_e\frac{q_1q_2}{r^2}\hat{\vec{r}}
	\label{eq:electrostaticforce}
\end{equation}
Here, $k_e$ is a coupling constant which takes different values for different choices of units. In the SI system we have
\begin{equation}
	k_e=\frac{1}{4\pi\epsilon_0}
	\label{eq:escoupling}
\end{equation}
With $\epsilon_0$ being the \textit{permittivity of free space}, which has value
\begin{equation}
	\epsilon_0=8.85\cdot10^{-12}\ \mathrm{\frac{C}{Nm^2}}
	\label{eq:epsilonnot}
\end{equation}
These forces are obviously additive.\\
Suppose now that you have a set of $n$ charges $q_i$ and you add an imaginary test charge $Q$ in order to \textit{theoretically} test the force field generated by these charges. We have then
\begin{equation}
	\vec{F}=\sum_{i=1}^n\vec{f}_i=\sum_{i=1}^n\frac{Qq_i}{4\pi\epsilon_0 r_i^2}\ver{r}_i=Q\sum_{i=1}^n\frac{q_i}{4\pi\epsilon_0r_i^2}\ver{r}_i
	\label{eq:forceonimcharge}
\end{equation}
The element inside the sum can be seen as the \textit{field} generated by the single particle $q_i$, denoted as $\vec{E}_i$. This field is the Electrostatic field. It's clear that then we can define a total field $\vec{E}$ by superposition of the single charge fields, and we can write, for a system of charges
\begin{equation}
	\vec{F}=Q\sum_{i=1}^n\vec{E}_i=Q\vec{E}
	\label{eq:esfield}
\end{equation}
Then, in general, we can say
\begin{equation}
	\vec{E}=\frac{\vec{F}}{Q}
	\label{eq:efielddef}
\end{equation}
For our system of $n$ charges the previous calculation is pretty straightforward and we directly get
\begin{equation}
	\vec{E}=\sum_{i=1}^n\frac{q_i}{4\pi\epsilon_0}\frac{\ver{r}}{r^2}
	\label{eq:esfieldsystem}
\end{equation}
The passage to continuous distributions of charge is straightforward. We define the following ``transformations''
\begin{equation*}
	\left\{ \begin{aligned}
		q_i&\longrightarrow\dd q\\
		\sum_i&\longrightarrow\int
\end{aligned}\right.
\end{equation*}
The electric field of such distribution is then
\begin{equation}
	\vec{E}=\frac{1}{4\pi\epsilon_0}\int_{}^{}\frac{\ver{r}}{r^2}\dd q
	\label{eq:efieldcont}
\end{equation}
In general, $\dd q$ can be expressed mathematically with a charge density which can be linear, superficial or volumetric. I.e.
\begin{equation}
	\dd q\to\begin{dcases}
		\lambda(\vec{r}')\dd l&\text{linear distribution}\\
		\sigma(\vec{r}')\dd s&\text{superficial distribution}\\
		\rho(\vec{r}')\dd^3x'&\text{volumetric distribution}
	\end{dcases}
	\label{eq:chargedist}
\end{equation}
The electric field will then be calculated with the integral \eqref{eq:efieldcont} extended to the appropriate set (a curve, a surface or a volume)
\subsection{Divergence of the Electrostatic Field}
As we have defined previously the electric field it's clear that if the distribution is complicated enough the integrals might be hard to solve or straight up nonsolvable. We then want to find different ways for calculating the field.\\
In general a vector field is determined by both its divergence and its curl. We firstly remember the definition of the 3D Dirac delta function $\delta^3(\vec{r})$, which is simply
\begin{equation}
	\delta^3(\vec{r})=\frac{1}{4\pi}\nabla\cdot\left( \frac{\ver{r}}{r^2} \right)
	\label{eq:delta3r}
\end{equation}
We then take the definition of $\vec{E}$ for a continuous volumetric distribution and simply apply the divergence operator.
\begin{equation}
	\nabla\cdot\vec{E}=\frac{1}{4\pi\epsilon_0}\nabla\cdot\int_{V}^{}\rho(\vec{r}')\frac{\ver{r}}{r^2}\dd^3x'
	\label{eq:Ediv}
\end{equation}
Noting that the integral is with respect to the primed coordinates (the ones with respect to the distribution) we can bring inside the divergence operator, and remembering that in this case $r=\norm{\vec{r}-\vec{r}'}$, with the definition of the 3D delta we get
\begin{equation*}
	\nabla\cdot\vec{E}=\frac{1}{\epsilon_0}\int_{V}^{}\rho(\vec{r}')\delta^3(\vec{r}-\vec{r}')\dd^3x'=\frac{1}{\epsilon_0}\rho(\vec{r})
\end{equation*}
Therefore, due to the generality of $\rho$ we have that for \textit{every} electrostatic field, the following equation holds
\begin{equation}
	\nabla\cdot\vec{E}=\frac{\rho}{\epsilon_0}
	\label{eq:1stmax}
\end{equation}
This is Maxwell's first equation for the electrostatic field.\\
A really important property comes from this equation, \textit{Gauss' law}. This law states that the flux of $\vec{E}$ is proportional to the total charge enclosed by the chosen volume $V$.\\
This is a direct consequence of Stokes' theorem for differential forms.\\
We choose a bounded volume $V\subset\R^3$ and integrate both sides of \eqref{eq:1stmax}
\begin{equation*}
	\iiint_V\nabla\cdot\vec{E}\dd^3x=\oiint_{\del V}\vec{E}\cdot\ver{n}\dd s=\frac{1}{\epsilon_0}\iiint_V\rho(\vec{r})\dd^3x
\end{equation*}
Defining the flux of $\vec{E}$ as $\Phi_{\del V}(\vec{E})$ we have, then
\begin{equation}
	\oiint_{\del V}\vec{E}\cdot\ver{n}\dd s=\Phi_{\del V}(\vec{E})=\frac{Q_V}{\epsilon_0}
	\label{eq:gausslaw}
\end{equation}
This is the mathematical expression of Gauss' law, where we have written
\begin{equation*}
	Q_V=\iiint_V\rho(\vec{r})\dd^3x
\end{equation*}
Which is the total charge contained inside the volume $V$.\\
This theorem is \textit{fundamental} for the solution of a myriad of electrostatic problems which would take a lot of calculations using \eqref{eq:efieldcont}. The main idea is that this can be used in conditions where there are particular symmetries of the system.
\begin{eg}[A charged sphere]
	Suppose that you have a charged sphere with radius $R$ and total charge $q$ and I want to know the electric field inside and outside the sphere. We begin by calculating the field outside using Gauss' law. Due to the radial symmetry of the problem we have that $\ver{n}=\ver{r}$ and therefore $\vec{E}=E\ver{n}$ when we choose a spherical volume.\\
	Let $\del V=S_r^2$ be our ``\textit{gaussian surface}'', a sphere of radius $r$, where the previous relation for $\vec{E}$ holds. We have that for any $r$
	\begin{equation}
		\Phi_{S^2_r}(\vec{E})=\oiint_{S^2_r}\vec{E}\cdot\ver{n}\dd s=E\oiint_{S_r^2}\dd s= 4\pi r^2E
		\label{eq:efluxsphere}
	\end{equation}
	The first part on the left of \eqref{eq:gausslaw} is already evaluated. Then we need to calculate only the right side. Noting that there is no charge outside the sphere we have an internal volumetric density of charge $\rho=q/V$. Since $V$ is a sphere we already know its volume, and the calculation it's quite easy
	\begin{equation}
		\iiint_{V_r}\rho(\vec{r})\dd^3x=\frac{q}{V}\iiint_{V_r}\dd^3x=\begin{dcases}
			q\frac{V_r}{V}&r<R\\
			q&r>R
		\end{dcases}
		\label{eq:charge}
	\end{equation}
	Where $V_r$ is the volume contained inside the gaussian sphere $S^2_r$. Remembering that $V_r=\frac{4}{3}\pi r^3$ and $V=\frac{4}{3}\pi R^3$ we have that
	\begin{equation}
		4\pi r^2E=\begin{dcases}
			\frac{q}{\epsilon_0}\left( \frac{r}{R} \right)^3&r<R\\
			\frac{q}{\epsilon_0}&r>R
		\end{dcases}
		\label{eq:almostsol}
	\end{equation}
	Dividing by $4\pi r^2$ and remembering that $\vec{E}=E\ver{n}=E\ver{r}$ we get the final solution for $\vec{E}$, both inside and outside the charged sphere
	\begin{equation}
		\vec{E}=\begin{dcases}
			\frac{q}{4\pi\epsilon_0 R^3}r\ver{r}&r<R\\
			\frac{q}{4\pi\epsilon_0 r^2}\ver{r}&r>R
		\end{dcases}
		\label{eq:eqsphere}
	\end{equation}
	Note how for $r<R$ the field grows linearly (we're adding charge increasing $r$), and it begins again falling like $r^{-2}$ after we surpass the surface of the sphere at $r=R$. Curiously (but not at random) the field for a charged sphere with constant charge $q$ is identical to the field produced by a point charge at the origin, it's like after we surpassed the surface of the sphere it collapsed all on the origin of the coordinates and became a point charge $q$ at the origin.
\end{eg}
The previous statements can be reformulated as a formal method
\begin{mtd}[Gaussian Surfaces]
	Given an electrostatic system with either spherical, cylindrical or planar symmetries. In order to solve \eqref{eq:1stmax} we need to choose an appropriate Gaussian surface $G$ which encloses a bounded volume $V$ for which $\vec{E}\propto\ver{n}_G$. In this special case, integrating the equation \eqref{eq:1stmax} and applying Stokes' theorem we have
	\begin{equation}
		\oiint_G\vec{E}\cdot\ver{n}_G\dd s=E\oiint_G\dd s=ES_G
		\label{eq:gausssurftrick}
	\end{equation}
	Where $S_G$ is the surface area of the gaussian surface. With this trick, if we call $V$ the bounded volume such that $\del V=G$ we have
	\begin{equation*}
		E=\frac{1}{\epsilon_0 S_G}\iiint_V\rho(\vec{r})\dd^3x
	\end{equation*}
	A rule of thumb for choosing $G$ is the following:
	\begin{itemize}
	\item For spherical symmetry of $E$ (like a point charge or a spherical distribution) $G$ is the sphere of radius $r$
	\item For cylindrical symmetry (like a charged cable or a charged cylinder) $G$ is the cylindrical surface of radius $r$
	\item for planar symmetry (like a charged plane) $G$ is a ``pillbox'', i.e. simply a 3D rectangle
	\end{itemize}
\end{mtd}
\subsection{The Scalar Potential}
A neat definition we can use is defining the \textit{scalar potential} $V(\vec{r})$ of the electrostatic field. As usual a potential for a vector field is defined if and only if all closed path integrals of the field are $0$ in a simply connected domain, i.e. that the curl of the field is zero in the selected domain.\\
We of course can choose this proof but it's much easier using this trick.\\
Take $V$ as a bounded domain of $\R^3$ where there is some charge distribution $\rho(\vec{r})$ inside. The general formula for the electric field then is the following
\begin{equation}
	\vec{E}=\frac{1}{4\pi\epsilon_0}\iiint_{V}^{}\rho(\vec{r}')\frac{\ver{r}}{r^2}\dd^3x'
	\label{eq:efieldpot}
\end{equation}
We immediately see that
\begin{equation*}
	\nabla\left( \frac{1}{r} \right)=-\frac{\ver{r}}{r^2}
\end{equation*}
Therefore, noting that the derivation acts only on the unprimed coordinates (i.e. it can go outside the integration without problems) we have
\begin{equation}
	\vec{E}=-\nabla\left( \frac{1}{4\pi\epsilon_0}\iiint_V\frac{\rho(\vec{r}')}{r}\dd^3x' \right)
	\label{eq:trick}
\end{equation}
By definition of potential then, we can say that $E=-\nabla V(\vec{r})$, where
\begin{equation}
	V(\vec{r})=\frac{1}{4\pi\epsilon_0}\iiint_V\frac{\rho(\vec{r}')}{r}\dd^3x'
	\label{eq:scalarpot}
\end{equation}
This is known as the \textit{scalar potential} of the electrostatic field.\\
Since by definition the curl of the gradient is always zero, we can immediately write a second constitutive equation for $\vec{E}$
\begin{equation}
	\nabla\cross\vec{E}=0
	\label{eq:3rdmaxwell}
\end{equation}
This equation is Maxwell's third equation for static fields.\\
Defining $\R^\star=\R\cup\{\pm\infty\}$ and chosen two points $a,b\in\R^\star$, we have in the language of differential forms
\begin{equation}
	\dd V=\vec{E}\cdot\dd\vec{x}
	\label{eq:extder}
\end{equation}
Therefore, with this definition, we can evaluate the work needed to move a charged particle through some path $\gamma:[a,b]\subset\R^\star\to\R^3$. We have
\begin{equation*}
	W=\int_{\gamma}^{}\vec{F}\cdot\ver{t}\dd l=q\int_\gamma\vec{E}\cdot\ver{t}\dd l=-q\int_\gamma\nabla V\cdot\ver{t}\dd l
\end{equation*}
Writing $\ver{t}\dd l=\dd\vec{x}$ we have
\begin{equation}
	W=-q\int_{\gamma}\nabla V\cdot\dd\vec{x}=-q\int_{V(a)}^{V(b)}\dd V=q\Delta V
	\label{eq:visvivaes}
\end{equation}
Therefore, $qV(\vec{r})$ can be imagined as a ``potential energy'' of the system. Via this definition, we have that the scalar potential has the following units in the SI system
\begin{equation}
	\left[ V \right]=\frac{\left[ W \right]}{\left[ q \right]}=\mathrm{\frac{J}{C}}=\mathrm{V}
	\label{eq:voltsdef}
\end{equation}
Where $\mathrm{V}$ are \textit{Volts}. With this definition
\begin{equation}
	1\ \mathrm{V}=1\ \mathrm{\frac{J}{C}}
	\label{eq:voltdef}
\end{equation}
From the definition of work we can immediately find a nice trick for evaluating the scalar potential of a distribution. Isolating the last two equalities in the first definition of work for the electric field we have
\begin{equation}
	\int_{\gamma}^{}\vec{E}\cdot\dd\vec{x}=-\int_{V(a)}^{V(b)}\dd V
	\label{eq:Vpathindep}
\end{equation}
Using the path independence of $V$ we have by direct integration
\begin{equation}
	V(b)-V(a)=-\int_{\gamma}^{}\vec{E}\cdot\dd\vec{x}
	\label{eq:trickV}
\end{equation}
Due to the definition of $V$ we know that it's defined up to a constant, and such constant can be chosen in order to have $V(a)=0$. The point $a\in\R^\star$ is known as the \textit{reference point} for the potential, and the appropriate choice depends from the charge distribution. The best choice is taking the point where the potential is 0\\
Suppose now we want to calculate the potential of a point charge in the origin. Since $\vec{E}\to0$ for $r\to\infty$ we take $a=\infty$, and therefore, since $V(\vec{r})\to0$ for $r\to\infty$ we have at some distance $b=r$
\begin{equation}
	V(r)=-\frac{q}{4\pi\epsilon_0}\int_{\infty}^{r}\frac{1}{r^2}\ver{r}\cdot\dd\vec{x}=-\frac{q}{4\pi\epsilon_0}\int_{\infty}^{r}\frac{1}{r^2}\dd r=\frac{q}{4\pi\epsilon_0}\frac{1}{r}
	\label{eq:qoriginpot}
\end{equation}
Note that by linearity of the integral, for a system of point charges we have
\begin{equation}
	V(r)=\frac{1}{4\pi\epsilon_0}\sum_{i=1}^n\frac{q_i}{r_i}
	\label{eq:systemchargespot}
\end{equation}
Note that this trick doesn't work if the charge extends to infinity since the integral would diverge, in that case the reference point will be some other $a=r_0$
\subsection{Maxwell Equations for Electrostatics and Boundary Conditions}
So far we found two main equations for the $\vec{E}$ field, these are two coupled partial differential equations known as the \textit{Maxwell equations for Electrostatics}. These equations are
\begin{equation}
	\left\{ \begin{aligned}
			\nabla\cdot\vec{E}&=\frac{\rho}{\epsilon_0}\\
			\nabla\cross\vec{E}&=\vec{0}
	\end{aligned}\right.
	\label{eq:maxwes}
\end{equation}
Or, in integral form for a bounded volume $V$ and a regular surface $\Sigma$
\begin{equation}
	\left\{ \begin{aligned}
			\oiint_{\del V}\vec{E}\cdot\ver{n}\dd s&=\frac{1}{\epsilon_0}\iiint_V\rho\dd^3x\\
			\oint_{\del\Sigma} \vec{E}\cdot\ver{t}\dd l&=0
	\end{aligned}\right.
	\label{eq:maxwesint}
\end{equation}
Inserting the definition of the potential these two equations collapse in a single equation, which is the \textit{Poisson equation} for the potential
\begin{equation}
	\nabla^2V=-\frac{\rho}{\epsilon_0}
	\label{eq:poissoneq}
\end{equation}
But, as for every partial differential equation, these make sense if and only if a boundary condition has been provided.\\
Without loss of generality we can consider an uniformly charged plane with surface density $\sigma$. We have using the Gaussian surface trick, choosing a pillbox with surface area $A$, that
\begin{equation*}
	E\oiint_G\dd s=\frac{\sigma A}{\epsilon_0}
\end{equation*}
Noting that the contribute between the 4 sides is zero, only the two faces remain and $S_G=2A$, and therefore
\begin{equation}
	E=\frac{\sigma}{2\epsilon_0}
	\label{eq:efieldplane}
\end{equation}
Since $\vec{E}\propto\ver{n}$ we have
\begin{equation}
	\vec{E}=\frac{\sigma}{2\epsilon_0}\ver{n}
	\label{eq:efieldplanevec}
\end{equation}
But the normal to the plane changes sign passing through its surface, therefore the field is discontinuous passing through its surface!\\
For the potential this is not true. By definition of potential we're checking the line integral along the tangent to the border of this Gaussian surface, which doesn't change sign when we pass through the surface. Therefore we have that $V\in C^2(V)\cup C^0(\del V)$ while the field is discontinuous on the border. These conditions are valid for every regular surface.\\
Consider that, locally, every regular surface can be considered as ``flat'' or euclidean, therefore the Gaussian pillbox trick works well.\\
Noting that the outward normal of the pillbox above and below the ``plane'' is equal to $\pm\ver{n}$ where $\ver{n}$ is the normal to this plane. Therefore, by the previous calculations we must have that passing through the surface (locally)
\begin{equation}
	\left( \vec{E}_i+\vec{E}_o \right)\cdot\ver{n}=0
	\label{eq:efieldboundarycond}
\end{equation}
I.e., the field outside this ``plane'' is opposite in sign to the field inside the ``plane``. Going back to the main general surface, via integration, we have that this result must hold generally, which emphasizes the discontinuity of the electric field.
\subsection{Energy of the Electrostatic Field}
Considering again the definition of work for a particle as $W=q\Delta V$ we can calculate it for a set of particles. Considering the interaction between particles we have that $W\propto q_iq_j$ where $i,j=1,\cdots,n$, and noting that a charge doesn't self interact, i.e. $q_iq_j=0$ for $i=j$ and that the usual multiplication between scalar is commutative, i.e. $q_iq_j=q_jq_i$ we have
\begin{equation}
	W=\frac{1}{8\pi\epsilon_0}\sum_{i=1}^n\sum_{j\ne i}^n\frac{q_iq_j}{r_{ij}^2}=\frac{1}{2}\sum_{i=1}^nq_iV(r_i)
	\label{eq:ensyspart}
\end{equation}
Passing to continuous distributions we get
\begin{equation*}
	W=\frac{1}{2}\iiint_V\rho(\vec{r})V(\vec{r})\dd^3x
\end{equation*}
Using the first Maxwell equation we have $\rho=\epsilon_0\nabla\cdot\vec{E}$, therefore
\begin{equation*}
	W=\frac{\epsilon_0}{2}\iiint_VV(\vec{r})\nabla\cdot\vec{E}\dd^3x
\end{equation*}
Integrating by parts and applying Stokes' theorem we get
\begin{equation}
	W=\frac{\epsilon_0}{2}\left( \oiint_{\del V}V\vec{E}\dd s-\iiint_V\vec{E}\cdot\nabla V\dd^3x \right)
	\label{eq:energyvolume}
\end{equation}
Noting that $\vec{E}$ extends to infinity where it becomes zero, we have on the limit $V\to\R^3$, where we use $\vec{E}=-\nabla V$, that the total energy stored in a charge distribution is
\begin{equation}
	W=\frac{\epsilon_0}{2}\iiint_{\R^3}E^2\dd^3x
	\label{eq:totalenergy}
\end{equation}
\section{Conductors}
The main real problem that somebody will encounter solving problems in electrostatics is problems with \textit{conductors}. A conductor is a rigid body for which there are free charges which can move after the application of an electric field. An example of conductor is a metallic body in the rigid body approximation.\\
One main property of conductors is that inside of it the electric field is zero.\\
Imagine taking a neutral box conductor, and then apply an electric field parallel to the sides of the box. The free charges will then move due to the action of the electrostatic force towards the field (if $q>0$) or against the field (if $q<0$). Since the conductor was neutral and charges must be conserved since they cannot pop into existence randomly, we have that the field generated by the single negative and positive charges on the surface of the conductor will be equal in magnitude and opposite in sign, therefore the total field inside is 0, even though the field outside is nonzero.\\
A second property of conductors is that the charge density inside the conductor is 0 inside. Using Gauss law and the first property of conductors we have
\begin{equation}
	\rho=\epsilon_0\nabla\cdot\vec{E}=0
	\label{eq:glinsidecond}
\end{equation}
This is always true for conductors, since as we said before $\vec{E}=\vec{0}$ inside.\\
One main explanation of this is that inside there is as much positive charge density $\rho_+$ and $\rho_-$. In fact, from Gauss' law we have
\begin{equation*}
	\rho=\rho_++\rho_-=0\implies\rho_+=\rho_-
\end{equation*}
This indicates that the charges of the conductor will then be all on the surface, and therefore the conductor is an equipotential surface. In fact
\begin{equation}
	\nabla V_{in}=-\vec{E}_{in}=0\implies V_{in}=k,\qquad k\in\R
	\label{eq:equipotentialconductor}
\end{equation}
In order to bring out other properties of the electric field in presence of conductors, we can consider the surface of separation between two materials. Consider a rectangular loop going through both materials. We have from the third Maxwell equation for electrostatics
\begin{equation}
	\oint_A\vec{E}\cdot\ver{t}\dd l=0
	\label{eq:between2cond}
\end{equation}
Since the conductor is rectangular, separating the line integral into 4 integrals, where 2 go parallel to the surface and 2 are normal to it, we have that the two normal integrals taking a clockwise path must cancel each other and therefore we have
\begin{equation}
	\oint_1\vec{E}_1\cdot\ver{t}_1\dd l+\oint_2\vec{E}_2\cdot\ver{t}_2\dd l=0
	\label{eq:parallelint}
\end{equation}
Since $\ver{t}_1=-\ver{t}_2$ we have that, locally
\begin{equation}
	\left( \vec{E}_1-\vec{E}_2 \right)\cdot\ver{t}_1=0
	\label{eq:constangcond}
\end{equation}
Therefore, the electric field tangent to the surface is continuous and therefore conserved.
Since a charged conductor has a zero electric field inside and there is no external field, then
\begin{equation*}
	\vec{E}_i=0,\quad\vec{E}_{ti}=0,\quad\vec{E}_{text}=0
\end{equation*}
But since in general a vector can always be decomposed in a tangent component to the surface and a normal component to the surface we have
\begin{equation*}
	\vec{E}_{ni}=0,\quad\vec{E}_{next}\ne0
\end{equation*}
This because the conductor is charged. This means that there is a discontinuity in the field and the field itself must be normal to the surface of the conductor due to the continuity of the tangential component of the field. If $\vec{E}_c$ is the electric field generated by a conductor we have then
\begin{equation}
	\vec{E}_c=E\ver{n}
	\label{eq:conductorfield}
\end{equation}
Consider now the potential inside and outside the conductor, $V_{ext},V_{in}$. Considering that the charges we are moving are electrons with $q=-e$, where $e$ is the fundamental charge
\begin{equation}
	e=1.6021766208(98)\cdot10^{-19}\ \mathrm{C}
	\label{eq:fundcharge}
\end{equation}
We have that the work needed to bring outside the conductor our electron will be
\begin{equation}
	\Delta U=-e\Delta V=-e\left( V_{ext}-V_{in} \right)
	\label{eq:workfunction}
\end{equation}
We define the work function as $L=\Delta U/e$ and it must obviously be positive since we're applying energy to the system in order to bring out an electron. We have
\begin{equation}
	L=V_{in}-V_{ext}>0\implies\ V_{in}>V_{ext}
	\label{eq:potcond}
\end{equation}
Due to all of these consideration, and noting that $\dd V=-\vec{E}\cdot\dd\vec{x}$ we have that the potential of a conductor will be defined as
\begin{equation}
	V_0(r)=-\int_{r}^{\infty}\vec{E}\cdot\ver{t}\dd l=-\int_{r}^{r_0}\vec{E}\cdot\ver{t}\dd l
	\label{eq:potentialcond}
\end{equation}
Where $r_0$ is the ''first`` radius immediately outside the conductor.\\
\subsection{Coulomb Theorem}
Consider now a conductor $V$ and take a small cylinder orthogonal to its surface. Considering that the charge on a conductor is only on the surface we have using Gauss' law on the differential flux of $\vec{E}$ that
\begin{equation}
	\dd\Phi(\vec{E})=\vec{E}\cdot\vec{n}\dd S=\frac{\sigma}{\epsilon_0}\dd S
	\label{eq:glcondsur}
\end{equation}
Considering the equality in terms of norms of the $\vec{E}$ field and remembering that $\vec{E}\parallel\ver{n}$ we have that
\begin{equation}
	\vec{E}=\frac{\sigma}{\epsilon_0}\ver{n}
	\label{eq:fieldconductor}
\end{equation}
You can immediately see that this field is twice the field generated by a charged infinite plane. Let's consider what's happening with some more precision.\\
In that small cylinder $\dd S$ we will have that the total external field will be composed from the contribution of the charge inside the cylinder and the one outside. The same should be for the inside, but the inside field \textit{must} be zero
\begin{equation*}
	\begin{aligned}
		\vec{E}_{ext}&=\vec{E}_{ext}^{\dd S}+\vec{E}_{ext}^{S-\dd S}\ne\vec{0}\\
		\vec{E}_{in}&=\vec{E}_{in}^{\dd S}+\vec{E}_{in}^{S-\dd S}=0
	\end{aligned}
\end{equation*}
The field $\vec{E}^{S-\dd S}$ doesn't change and it must be the same as the field generated by $\dd S$. Applying Gauss' theorem to the small surface element $\dd S$ and noting that it must be the same of a plane with surface area $\dd S$ we have
\begin{equation*}
	\vec{E}_{in}^{S-\dd S}=\vec{E}_{ext}^{S-\dd S}=-\vec{E}_{in}^{\dd S}=\frac{\sigma}{2\epsilon_0}\ver{n}
\end{equation*}
Therefore, finally
\begin{equation*}
	\vec{E}_{ext}=\frac{\sigma}{2\epsilon_0}\ver{n}+\frac{\sigma}{2\epsilon_0}\ver{n}=\frac{\sigma}{\epsilon_0}\ver{n}
\end{equation*}
Where we used again that $\vec{E}_{ext}^{\dd S}=\frac{\sigma}{2\epsilon_0}\ver{n}$
\subsection{Induced Charges}
Consider now some conductor which is empty inside. Inside the first conductor we insert another conductor charged with charge $Q$. At $t=0$ the external conductor is neutral, and therefore $Q_{ext}=0$.\\
Since charge must be conserved, we have that at $t>0$ when we insert the new conductor inside the total charge must remain neutral, therefore
\begin{equation*}
	Q_{int}+Q_{ext}=0
\end{equation*}
From Gauss' theorem, taking  a surface inside the conductor that includes inside itself the internal surface of the conductor but not the external one. For Gauss we have
\begin{equation*}
	\Phi(\vec{E})=0=\frac{Q_V}{\epsilon_0}\implies Q+Q_{in}=0
\end{equation*}
Therefore, there must be an \textit{induced charge} $Q_{in}$ on the internal surface of the conductor, such that
\begin{equation}
	Q_{in}=-Q
	\label{eq:inducedcharge}
\end{equation}
From this, substituting before, we have that on the external surface we measure the charge we added inside the conductor, via the process of charge induction
\begin{equation}
	Q_{ext}=-Q_{in}=Q
	\label{eq:chargemeas}
\end{equation}
Note that this comes directly for having charge conservation.\\
Consider now the same empty conductor but don't add any charge inside of it, but rather charge the whole conductor with some positive charge $Q$. What happens inside the hole? Is there any charge?\\
By Gauss' theorem we have, since $\vec{E}=\vec{0}$ inside the conductor, that the total charge inside the conductor is zero $Q_{in}=0$.\\
There could still be a charge balance inside, where $Q_{in}^+-Q_{in}^-=Q_{in}=0$. Supposing this true we can take a closed path that goes inside the hole. By definition of $\vec{E}$ the line integral on this path $\gamma$ must be zero. Divide the path into 1, that goes inside the hole, where there should be a field $\vec{E}$ between the two charges $Q_{in}^+$ and $Q_{in}^-$, and path 2 which is inside the conductor. Then we would have
\begin{equation*}
	\oint_\gamma\vec{E}\cdot\ver{t}\dd l=\int_1\vec{E}\cdot\ver{t}_1\dd l+\int_2\vec{E}\cdot\ver{t}_2\dd l=\int_1\vec{E}\cdot\ver{t}_1\dd l
\end{equation*}
Since the path is closed, call $D$ the surface enclosed by the path, we have
\begin{equation}
	\oint_\gamma\vec{E}\cdot\ver{t}\dd l=\oiint_D\nabla\cross\vec{E}\cdot\ver{n}\dd s=\int_1\vec{E}\cdot\ver{t}_1\dd l\implies\ \nabla\cross\vec{E}\ne\vec{0}
	\label{eq:contradictioncond}
\end{equation}
This is in clear contradiction with Maxwell's equation for electrostatics (which we have already demonstrated that they generally hold), therefore all the charge is safely distributed on the \textit{external} surface of the conductor, as we expected.
\begin{exe}[Two Charged Spheres]
	Suppose that you have two metal spheres connected by a wire. One has radius $R_1$ and the other has radius $R_2$. At $t>0$ we deposit some charge $Q$ on the system. What will be the total charge distributed on the two spheres? ($Q_1,Q_2$)\\\\
	\texttt{\textbf{S o l u t i o n}}\\\\
	The potentials on the two spheres must be equal, and we know already from previous calculations that
	\begin{equation}
		\begin{aligned}
			V_1&=\frac{1}{4\pi\epsilon_0}\frac{Q_1}{R_1}\\
			V_2&=\frac{1}{4\pi\epsilon_0}\frac{Q_2}{R_2}\\
			V_1&=V_2
		\end{aligned}
		\label{eq:spherepot}
	\end{equation}
	From the previous equation we have that
	\begin{equation}
		Q_2=\frac{R_2}{R_1}Q_1
		\label{eq:q2}
	\end{equation}
	The total charge, on the other hand, will be $Q=Q_1+Q_2$, therefore
	\begin{equation}
		Q=Q_1+\frac{R_2}{R_1}Q_1=\frac{R_1+R_2}{R_1}Q_1\implies\ Q_1=\frac{R_1}{R_1+R_2}Q
		\label{eq:totalcharge1}
	\end{equation}
	And, analogously
	\begin{equation}
		Q_2=\frac{R_2}{R_1+R_2}Q
		\label{eq:totalcharge2}
	\end{equation}
	From Gauss' theorem, if the spheres have surface charges $\sigma_i$, $i=1,2$ we must also have
	\begin{equation}
		\frac{Q_1}{R_1}=\frac{Q_2}{R_2}\implies\ \frac{4\pi R^2_1\sigma_1}{R_1}=\frac{4\pi R_2^2\sigma_2}{R_2}
		\label{eq:gausslaw}
	\end{equation}
	I.e.
	\begin{equation*}
		\sigma_1R_1=\sigma_2R_2\implies\ \sigma_1=\frac{R_2}{R_1}\sigma_2
	\end{equation*}
	Since $\frac{R_1\sigma_1}{\epsilon_0}=\frac{R_2\sigma_2}{\epsilon_0}$, we must also have that the fluxes of the fields multiplied by $R_i$ are equal, i.e. the electric fields are scaled as follows
	\begin{equation}
		E_2=\frac{R_1}{R_2}E_1
		\label{eq:efield2spheres}
	\end{equation}
	\hfill\square
\end{exe}
\subsection{Capacity}
Consider an isolated conductor on which there is some charge $Q$, distributed with density $\sigma$ on its surface, such that the conductor is equipotential. We have that for every point in the conductor, by definition
\begin{equation}
	\begin{aligned}
		V(r)&=\frac{1}{4\pi\epsilon_0}\iint_S\frac{\sigma}{r}\dd s\\
		Q&=\iint_S\sigma\dd s
	\end{aligned}
	\label{eq:potentialchargecapacity}
\end{equation}
It's clear that by this definition that if we vary $\sigma$ to a new $\sigma'=\alpha\sigma$ with $\alpha\in\R$, we also have that $V'=\alpha V,\ Q'=\alpha Q$.\\
The following rate is then called the \textit{capacity} of the conductor
\begin{equation}
	C=\frac{Q}{V}
	\label{eq:capacitu}
\end{equation}
This is clearly only dependent on the geometry of the system. The capacity is measured in \textit{Farads}, where
\begin{equation*}
	1\unit{F}=1\unit{\frac{C}{V}}
\end{equation*}
\begin{eg}[Capacity of a Spherical Conductor]
	Take now a spherical conductor with charge $Q$. We have
	\begin{equation*}
		V=\frac{Q}{4\pi\epsilon_0R}
	\end{equation*}
	Therefore
	\begin{equation}
		C=4\pi\epsilon_0R
		\label{eq:sphericalcondcap}
	\end{equation}
	This lets us redefine $\epsilon_0$ in terms of Farads. In fact
	\begin{equation*}
		\left[ \epsilon_0 \right]=\frac{\left[ C \right]}{\left[ R \right]}=\unit{\frac{F}{m}}
	\end{equation*}
	Therefore
	\begin{equation}
		\epsilon_0=8.854\unit{\frac{F}{m}}
		\label{eq:epsilon0farad}
	\end{equation}
\end{eg}
In the case that we have multiple conductors one close to the other the problem gets slightly more complex.\\
Add a charge $Q_1$ to the first conductor, which will have potential $V_1$, which will induce a charge $Q_2$ and therefore a potential $V_2$ on the second. If I change the charge to $Q_1'=\alpha Q_1$ we will have a basically identical result to the previous problem. Inverting the system and setting the charge on the second conductor $Q_2$ we will have a symmetrical system, for which we can write
\begin{equation}
	V_i=\sum_{j=1}^np_{ij}Q_j
	\label{eq:potcoef}
\end{equation}
The $p_{ij}$ are the potential coefficients, for which holds $p_{ij}=p_{ji}>0$, $p_{ii}\ge p_{ij}\ i\ne j$.\\
Due to the fact that the potential is unequivocally determined we must be able to solve the inverse problem, therefore we also know that $\det p_{ij}\ne0$, and therefore
\begin{equation}
	Q_i=\sum_{j=1}^nc_{ij}V_j
	\label{eq:capcoef}
\end{equation}
The matrix $c_{ij}$ is known as the capacitance matrix, and we have $p_{ij}=c^{-1}_{ij}$. The diagonal elements $c_{ii}$ are known as the \textit{capacity coefficients}, while the off diagonal $c_{ij},\ i\ne j$ are known as the induction coefficients.\\
For this matrix hold the following properties, known as \textit{Maxwell inequalities}
\begin{equation}
	\left\{\begin{aligned}
		c_{ij}&=c_{ji}\\
		c_{ii}&>0\\
		c_{ij}&<0\quad i\ne j\\
		\sum_{j=1}^nc_{ij}&\ge0
	\end{aligned}\right.
	\label{eq:maxwelldis}
\end{equation}
\subsection{Capacitors}
Let's take again two conductors in total induction as for our previous system of two concentric conductors where one inside is set at a charge $Q$. Grounding the external surface we get that the external shell will be at a fixed $V=0$, while the internal surface will have an induced charge $-Q$. Between these two surfaces there will be a potential difference $\Delta V$, for which it's possible to evaluate the capacitance as
\begin{equation*}
	C=\frac{Q}{\Delta V}
\end{equation*}
Writing this in terms of the potential matrix $V_i=\sum_jp_{ij}Q_j$ we have the following system of equations, where $Q_1=Q,\ Q_2=-Q$
\begin{equation*}
	\left\{ \begin{aligned}
			V_1&=p_{11}Q-p_{12}Q\\
			V_2&=p_{21}Q-p_{22}Q
	\end{aligned}\right.
\end{equation*}
Subtracting the second from the first we have
\begin{equation*}
	\Delta V=\left( p_{11}+p_{12}-2p_{12} \right)Q
\end{equation*}
Therefore
\begin{equation}
	C=\frac{Q}{\Delta V}=\frac{1}{p_{11}+p_{12}-2p_{21}}
	\label{eq:capacitypotentialmatrix}
\end{equation}
Or in terms of the capacitance matrix $c_{ij}$
\begin{equation}
	C=\frac{\det(c_{ij})}{c_{11}+c_{12}-2c_{12}}
	\label{eq:capacitancecapmatrix}
\end{equation}
Finding the capacitance using these matrices tho is a quite long calculations, therefore we directly use the line integral of the $\vec{E}$ field for determining it, therefore, since
\begin{equation*}
	\Delta V_{12}=\int_{2}^{1}\vec{E}\cdot\dd\vec{x}
\end{equation*}
\begin{eg}[Spherical Capacitor]
	Consider now a spherical capacitor for which the outer shell is grounded, we have
	\begin{equation*}
		\vec{E}=\frac{Q}{4\pi\epsilon_0}\frac{\ver{r}}{r^2}
	\end{equation*}
	Therefore
	\begin{equation*}
		\Delta V=\frac{Q}{4\pi\epsilon_0}\int_{r_2}^{r_1}\frac{1}{r^2}\dd r=\frac{Q}{4\pi\epsilon_0}\left( \frac{1}{r_1}-\frac{1}{r_2} \right)
	\end{equation*}
	Therefore
	\begin{equation}
		C_s=\frac{4\pi\epsilon_0r_1r_2}{r_2-r_1}
		\label{eq:sphcap}
	\end{equation}
\end{eg}
\begin{eg}[Cylindrical Capacitor]
	For a cylindrical capacitor made of two conducting cylinders of radius $R_1$ and $R_2$ and length $l>>R_2$ and total charge $\lambda l$ we have that the electric field is
	\begin{equation*}
		\begin{aligned}
			2\pi lrE&=\frac{\lambda l}{\epsilon_0}\\
			\vec{E}&=\frac{\lambda}{2\pi\epsilon_0}\frac{\ver{r}}{r}
		\end{aligned}
	\end{equation*}
	Therefore
	\begin{equation*}
		\Delta V=\int_{1}^{2}\vec{E}\cdot\dd\vec{x}=\frac{\lambda}{2\pi\epsilon_0}\log\left( \frac{R_2}{R_1} \right)
	\end{equation*}
	Therefore, since $Q=\lambda l$, we have
	\begin{equation}
		C_c=\frac{2\pi\epsilon_0}{\log\left( \frac{R_2}{R_1} \right)}
		\label{eq:cylcap}
	\end{equation}
\end{eg}
\begin{eg}[Parallel Plane Capacitor]
	For two parallel plane conductors for which $d<<\sqrt{S}$ where $S$ is the surface area of the plane we have that
	\begin{equation*}
		E=\frac{\sigma}{\epsilon_0},\ Q=S\sigma
	\end{equation*}
	Therefore
	\begin{equation*}
		\Delta V=\frac{\sigma}{\epsilon_0}d
	\end{equation*}
	Where $d$ is the distance between the plates, and therefore
	\begin{equation}
		C=\frac{\epsilon_0S}{d}
		\label{eq:parallelplane}
	\end{equation}
\end{eg}
\subsection{Forces on a Conductor}
Consider a charged conductor with surface area $S$. Considering a small element $\dd S$ we have that the external field generated by the remaining surface is
\begin{equation*}
	\vec{E}_{ext}^{S-\dd S}=\frac{\sigma}{2\epsilon_0}\ver{n}
\end{equation*}
The total charge in $\dd S$ is $\sigma\dd S$, and therefore the (infinitesimal) force on the area element $\dd S$ is, by definition of electrostatic force
\begin{equation}
	\dd\vec{F}=\sigma\vec{E}^{S-\dd S}\dd S=\frac{\sigma^2}{2\epsilon_0}\ver{n}\dd S=\frac{1}{2}\epsilon_0E^2\ver{n}\dd S=u\ver{n}\dd S
	\label{eq:forceonds}
\end{equation}
Where we identified the energy density of the field $u$ as
\begin{equation*}
	u=\frac{1}{2}\epsilon_0E^2
\end{equation*}
Deriving everything by $\dd S$, we have that the \textit{electrostatic pressure} $\vec{p}$ on the infinitesimal element of the surface of the conductor is
\begin{equation}
	\vec{p}=\dv{\vec{F}}{S}=u
	\label{eq:elpressure}
\end{equation}
Consider now a virtual displacement of the external surface of the conductor, where we move it by $\delta r$ orthogonally to the previous surface, then, the (virtual) work necessary for such displacement is
\begin{equation*}
	\delta L=\delta\vec{F}_e\cdot\delta\vec{r}=\delta U
\end{equation*}
Where we used that $\delta L=\delta U$, and $\vec{F}_e$ as the ''extraction force``. Since $\vec{F}_e=-\vec{F}$ we have that
\begin{equation*}
	\delta F_r=-\frac{\delta U}{\delta r}
\end{equation*}
But, by definition
\begin{equation*}
	\delta U=-\frac{1}{2}\epsilon_0E^2\delta r\dd S
\end{equation*}
Therefore, as before
\begin{equation*}
	\delta F_r=u\dd S
\end{equation*}
For constant charge, we might think to apply this to a charged parallel plate capacitor, for which we know that the infinitesimal work needed to charge it, i.e. to move the charges from infinity towards our capacitor, is
\begin{equation*}
	\dd W=V\dd q=\frac{q}{C}\dd q\implies\ W=\frac{1}{2}\frac{Q^2}{C}
\end{equation*}
For a parallel plate capacitor therefore
\begin{equation*}
	U(x)=\frac{1}{2}\frac{Q^2x}{\epsilon_0S}
\end{equation*}
Therefore
\begin{equation*}
	F=-\pdv{U}{x}=-\frac{1}{2}\frac{Q^2}{\epsilon_0S}
\end{equation*}
This force is attractive (obvious from the system).\\
What if $V=cost.$ but the charge isn't constant? We know that
\begin{equation*}
	V=\frac{Q}{C}
\end{equation*}
And since both $C,V$ are constants (one depends only on the geometry and the other is set constant by the system) $Q$ can be the only one to have changed.\\
This means that there is some generator that charges up the capacitor, with work
\begin{equation*}
	\delta W_g=V\delta Q=V^2\delta C
\end{equation*}
Where we have $Q=VC$. From our previous relations we have
\begin{equation*}
	\delta W_{ext}=F_{ext}\delta x,\qquad\delta U=\delta W_g+\delta W_{ext}
\end{equation*}
Therefore, since $F_{ext}=-F$ we have that
\begin{equation*}
	\delta U=\delta W_g-\delta W
\end{equation*}
And, for the generator
\begin{equation*}
	\delta(CV^2)=\delta W+\delta\left( \frac{1}{2}CV^2 \right)
\end{equation*}
Finally
\begin{equation*}
	\delta W=\delta U=F\delta x
\end{equation*}
Remembering that $U=CV^2/2$ and $C=S\epsilon_0/x$ we have through derivation that
\begin{equation*}
	F=-\frac{1}{2}\frac{S\epsilon_0 V^2}{x^2}=-\frac{1}{2}\frac{C^2V^2}{S\epsilon_0}=-\frac{1}{2}\frac{Q^2}{S\epsilon_0}
\end{equation*}
Which is the same result as before.\\
It's clear that for a charged conductor then the force is the mechanical moment of the system. It can be derived using the virtual work theorem, noting that $\delta L_{ext}=\delta U=-\delta L$, therefore
\begin{equation*}
	\delta L=\vec{F}\cdot\delta\vec{x}+\vec{L}\cdot\delta\vec{\theta}=-\delta U
\end{equation*}
Where, in the limit $\delta x,\delta \theta\to\dd x,\dd\theta$
\begin{equation*}
	\begin{aligned}
		F_x&=-\pdv{U}{x}\\
		L_\theta&=-\pdv{U}{\theta}
	\end{aligned}
\end{equation*}
\end{document}
