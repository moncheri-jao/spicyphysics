\documentclass[../electromagnetism.tex]{subfiles}
\begin{document}
\section{RC Filters}
\subsection{RC Circuit in DC}
Let's begin to consider what happens when currents aren't anymore constant in time but instead are variable. The simplest possible circuit that we can imagine with variable currents (but with a continuous source) is the following one
\begin{figure}[H]
	\centering
	\begin{circuitikz}
		\draw (0, 0) node[cute spdt down, rotate=180](sw){};
		\draw (sw.out 1) -- ++(0, -4) node[ground](gr){};
		\draw (sw.in) to[R, l=$R$, v=$V_R$] ++(2, 0) to[C, l=$C$, i=$I_C(t)$, v=$V_C(t)$] ++(0, -4) |- (gr);
		\draw (sw.out 2) -- ++(-4, 0) to[dcvsource, l=$V$, i=$I$] ++(0, -4) |- (gr);
	\end{circuitikz}
	\caption{Diagram of a circuit which charges a capacitor with DC current and then lets it discharge freely to ground}
	\label{fig:chargedisC.ac}
\end{figure}
If we consider two time frames, one when the switch is on the other configuration and charges the capacitor, till a time $t_0$, when the capacitor is fully charged, and then gets switched and the charged capacitor discharges on the ground, we have, during the charge of the capacitor
\begin{equation*}
	V=V_R+V_C=RI+\frac{Q}{C}=R\dv{Q}{t}+\frac{C}{C}
\end{equation*}
Where we used the following identities
\begin{equation*}
	\begin{paligned}
		C&= \frac{Q}{V_C}\\
		I&= \dv{Q}{t}
	\end{paligned}
\end{equation*}
Solving for the current we have
\begin{equation*}
	RC\dv{Q}{t}=CV-Q
\end{equation*}
And solving this ordinary differential equation between $t=0$ and $t=t_0$ we get
\begin{equation*}
	log\left( \frac{CV-Q}{CV-Q(0)} \right)=-\frac{t}{RC}
\end{equation*}
Which after solving for $Q$ becomes, noting that at $t=0$ we have $Q(0)=0$
\begin{equation}
	Q(t)=CV\left( 1-e^{-\frac{t}{RC}} \right)
	\label{eq:chargeC.ac}
\end{equation}
We define the \textit{relaxation time of the circuit} $\tau=RC$, and after substituting the solution into the definition of the capacitance and solving for $V_C$ we have
\begin{equation}
	V_C(t)=V\left( 1-e^{-\frac{t}{\tau}} \right)
	\label{eq:vccdischarge.ac}
\end{equation}
If we solve for $V_R$, noting that $V_R=RI=R\dot{Q}$ we get that the voltage at the poles of the resistor at a time $t$ is
\begin{equation}
	V_R(t)=\frac{RCV}{\tau}e^{-\frac{t}{\tau}}=Ve^{-\frac{t}{\tau}}
	\label{eq:chargecaprv.ac}
\end{equation}
The circuit is said to be stable when $t\gtrsim4\tau$, so that we can assume the capacitor as fully charged.\\
We can now analyze the case when the capacitor is discharging, i.e. when the switch is flipped and the resistance and capacitor are isolated, and the circuit is practically the following
\begin{figure}[H]
	\centering
	\begin{circuitikz}
		\draw (0, 0) node[ground](gr){};
		\draw (gr) -- ++(4, 0) to[C, l=$C$, i=$I_C(t)$, v=$V_C(t)$] ++(0,4) to[R, l=$R$, v=$V_R(t)$] ++(-4, 0) |- (gr);
	\end{circuitikz}
	\caption{Diagram of the discharge of the capacitor $C$}
	\label{fig:dischargeconf.ac}
\end{figure}
Here, since we're directly connected to ground, we must have $V=0$, thus
\begin{equation*}
	I(t)R+V_C(t)=0
\end{equation*}
Thus, through substitution we have
\begin{equation*}
	R\dv{Q}{t}=-\frac{Q(t)}{C}\implies Q(t)=Q(0)e^{-\frac{t}{RC}}
\end{equation*}
But $Q(0)=CV$m and thus
\begin{equation}
	\begin{paligned}
		Q(t)&= CVe^{-\frac{t}{\tau}}\\
		V_C(t)&=\frac{Q(t)}{C}Ve^{-\frac{t}{\tau}}\\
	\end{paligned}
	\label{eq:chargecondfinal.ac}
\end{equation}
\subsection{RC Circuit in AC, Low Pass Filters}
Consider now the same circuit using an alternate voltage source $V(t)$. Consider the special case where this source is an harmonic source, as
\begin{equation*}
	V(t)=V_0\cos\left( \omega t \right)
\end{equation*}
The circuit diagram is the following
\begin{figure}[H]
	\centering
	\begin{circuitikz}
		\draw (0, 0) to[sV=$V$, v=$V(t)$] ++(0, 3) to[R, l=$R$] ++(4, 0) to[C, l=$C$, v=$V_C(t)$] ++(0, -3) -- (0, 0); 
	\end{circuitikz}
	\caption{Alternate current version of the resistor-capacitor circuit treated in the previous section}
	\label{fig:accapacitordis.ac}
\end{figure}
As we have already saw before, we have
\begin{equation*}
	\begin{paligned}
		V_C&= \frac{Q}{C}\\
		V_R&= R\dv{Q}{t}\\
	\end{paligned}
\end{equation*}
Thus, the \textit{characteristic differential equation of the circuit} is 
\begin{equation*}
	V(t)-R\dv{Q}{t}-\frac{Q}{C}=0
\end{equation*}
Or, noting that
\begin{equation*}
	\frac{Q}{C}=V_C\implies Q=CV_C\implies R\dv{Q}{t}=RC\dv{V_C}{t}
\end{equation*}
We have
\begin{equation}
	V(t)-RC\dv{V_C(t)}{t}-V_C(t)=0
	\label{eq:rcchargechareq.ac}
\end{equation}
Substituting the functional expression of $V(t)$, we get a not-so-difficult first order differential equation
\begin{equation}
	RC\dv{V_C}{t}+V_C(t)=V_0\cos(\omega t)
	\label{eq:eqtosolvercdis.ac}
\end{equation}
We solve it by searching a solution via the similarity method, i.e. supposing that $V_{C}(T)\propto\cos(\omega t)$. Said $V_{0C}$ the proportionality constant we get, adding a generic phase
\begin{equation*}
	\dv{V_C}{t}=-\omega V_{0C}\sin\left( \omega t+\varphi \right)
\end{equation*}
This is better expressed using phasor notation, which is built using Euler's identity. Keeping in mind that we actually measure the real part of this complex function, i.e. $V_C^{(M)}(t)=\real\left\{ V_{C}(t) \right\}$, we get by deriving and plugging back into the differential equation
\begin{equation*}
	V_C(t)+RC\dv{V_C}{t}=V_{0C}e^{i\omega t+i\varphi}+i\omega RCV_{0C}e^{i\omega t+i\varphi}=V_0e^{i\omega t}
\end{equation*}
Solving, we have
\begin{equation*}
	\left( 1+i\omega RC \right)V_{0C}e^{i\varphi}=V_0\implies V_{0C}e^{i\varphi}=\frac{1-i\omega RC}{1+\left( RC\omega \right)^2}V_0
\end{equation*}
And therefore, substituting $\tau=RC$
\begin{equation}
	V_{0C}(t)e^{i\varphi}=\left( \frac{1}{1+\left( \tau\omega \right)^2}-\frac{i\omega \tau}{1+\left( \tau\omega \right)^2} \right)V_0
	\label{eq:solutionrcac.ac}
\end{equation}
Or, in terms of measured voltages
\begin{equation}
	V_{C}^{(M)}=\frac{V_0}{1+\omega^2\tau^2}\cos\left( \omega r+\arctan\left( -\omega\tau \right) \right)
	\label{eq:measuredvoltrc.ac}
\end{equation}
Note that this solution is strongly dependent on the frequency $\omega$ of the input voltage, in fact, we specifically have, ignoring the phase $\varphi=\arctan\left( -\omega\tau \right)$
\begin{equation}
	\begin{paligned}
		\lim_{\omega\to0}V_{0C}(\omega)&= V_0\\
		\lim_{\omega\to\infty}V_{0C}(\omega)&= 0\\
	\end{paligned}
	\label{eq:filterbehaviorrc.ac}
\end{equation}
This behavior clearly shows that the voltage is nonzero only when $\omega<\omega_R$, with $\omega_R$ being the resonance frequency. This type of circuit is widely known as a \textit{low pass filter}, i.e. a circuit which permits the passage of voltage only to a limit \textit{resonant} frequency.\\
We now might want to see how does the current behave in this circuit. We have
\begin{equation*}
	I(t)=\dv{Q}{t}\implies I(t)=C\dv{V_C}{t}=i\omega V_{0C}e^{i\omega t +i\varphi}=i\omega CV_{C}(t)
\end{equation*}
Thus
\begin{equation}
	\begin{paligned}
		V_C(t)e^{i\varphi_V}&= \frac{V_0}{1+i\omega\tau}e^{i\omega t}\\
		I(t)e^{i\varphi_I}&= \frac{i\omega CV_0}{1+i\omega\tau}e^{i\omega t}
	\end{paligned}
	\label{eq:current+voltrc.ac}
\end{equation}
we can evaluate easily the phase of the current, and we have
\begin{equation*}
	\varphi_I=\arctan\left( \frac{1}{\omega\tau} \right)
\end{equation*}
Thus it's clear that the phases of current and voltage are different, and precisely we have that since
\begin{equation*}
	\tan\left( \varphi_I-\varphi_V \right)=\frac{\tan\varphi_I-\tan\varphi_V}{1+\tan\varphi_V\tan\varphi_I}=\frac{\frac{1}{\omega\tau}+\omega\tau}{1-\frac{\omega\tau}{\omega\tau}}=\infty
\end{equation*}
Thus $\varphi_I-\varphi_V=\frac{\pi}{2}$
\subsubsection{Generalized Ohm Law}
From what we saw before, we can imagine to compile a new \textit{generalized Ohm law}, which works also with different components and variable currents.\\
From what we saw before, we have for a resistor-capacitor circuit
\begin{equation*}
	I(t)=i\omega CV_C(t)
\end{equation*}
From $V=RI$ we can define the \textit{capacitor impedance} $Z_C$ as follows
\begin{equation}
	V_C(t)=Z_CI(t)\implies Z_C=\frac{1}{i\omega C}
	\label{eq:capacitorimped.gol}
\end{equation}
Let's consider now an inductor $L$. From Lenz's law we have
\begin{equation*}
	V_L(t)=L\dv{I_L}{t}=i\omega LI_L
\end{equation*}
Thus, again from $V=RI$ we get
\begin{equation}
	Z_L=i\omega L
	\label{eq:inductorimpedance}
\end{equation}
Thanks to this considerations we can say that for circuits containing only resistors, capacitors and inductors in alternate current we have $V=ZI$, where
\begin{equation}
	\begin{paligned}
		Z_R&= R\\
		Z_{C}&= \frac{1}{i\omega C}\\
		Z_{L}&= i\omega L
	\end{paligned}
	\label{eq:impedances.gol}
\end{equation}
Consider now the classic voltage divisor, where instead of having resistor we install a generic component which can be either a resistor, a capacitor or an inductor, connected as follows
\begin{figure}[H]
	\centering
	\begin{circuitikz}
		\draw (0, 0) to[sV, l=$V_{in}(t)$] ++(0, 3) -- ++(4, 0) to[generic, l=$Z_1$] ++(0, -2) node[left](o1){} to[generic, l=$Z_2$] ++(0, -2) node[](o2){} -| (0, 0);
		\draw (o1) -- ++(2, 0) to[open, o-o, v^=$V_{out}$] ++(0, -2) -| (o2);
	\end{circuitikz}
	\caption{A generic voltage-divider like circuit in alternate current}
	\label{fig:vdiv.gol}
\end{figure}
Since we already know that the impedances $Z$ work exactly as do the resistances, we get
\begin{equation}
	V_{out}(t)=\frac{Z_2}{Z_1+Z_2}V_{in}(t)
	\label{eq:acvoltdiv.gol}
\end{equation}
If we suppose $Z_1$ as a normal resistor and $Z_2$ as some capacitor, using the previous formulas we get
\begin{equation}
	V_{out}(t)=\frac{\frac{1}{i\omega C}}{R+\frac{1}{i\omega C}}V_{in}(t)=\frac{1}{1+i\omega RC}V_{in}(t)
	\label{eq:voltagepartitorc.gol}
\end{equation}
Note that this is \textit{exactly} the RC circuit we solved before at length.
\subsection{CR Circuit, High Pass Filters}
The previous section can be used to immediately solve the CR circuit, since it has the exact same shape as this voltage divider, as in the following diagram
\begin{figure}[H]
	\centering
	\begin{circuitikz}
		\draw (0, 0) to[sV, l=$V_{in}(t)$] ++(0, 3) -- ++(4, 0) to[C, l=$C$, v=$V_C(t)$] ++(0, -2) node[left](o1){} to[R, l=$R$, v=$V_R(t)$] ++(0, -2) node[](o2){} -| (0, 0);
		\draw (o1) -- ++(2, 0) to[open, o-o, v^=$V_{out}$] ++(0, -2) -| (o2);
	\end{circuitikz}
	\caption{A CR circuit in an alternate current regime}
	\label{fig:crcirc.ac}
\end{figure}
From what we wrote before, using the generalized Ohms law we find immediately that
\begin{equation*}
	V_{out}(t)=\frac{R}{R+Z_C}V_{in}(t)=\frac{i\omega RC}{i\omega RC+1}V_{in}(t)
\end{equation*}
We find immediately that the phase is
\begin{equation*}
	\varphi_V=\arctan\left( \frac{\imaginary V_{out}(t)}{\real V_{out}(t)} \right)=\arctan\left( \frac{1}{\omega RC} \right)
\end{equation*}
Considering also the ratio of voltage between the input and output 
\begin{equation*}
	\frac{V_{out}}{V_{in}}=\frac{\omega RC}{\sqrt{1+\left( \omega RC \right)^2}}
\end{equation*}
It's clear that for low frequencies, specifically for $\omega<<\omega_c=\tau^{-1}$ the ratio between these voltages goes quickly to zero. This behavior is typical of the so-called \textit{high pass filters}
\section{RL Filters}
Thanks to what we have defined before, finding the behavior of RL and LR circuits is almost immediate. Consider firstly a LR circuit, as in the following diagram
\begin{figure}[H]
	\centering
	\begin{circuitikz}
		\draw (0, 0) to[sV, l=$V_{in}(t)$] ++(0, 3) -- ++(4, 0) to[L, l=$L$, v=$V_L(t)$] ++(0, -2) node[left](o1){} to[R, l=$R$, v=$V_R(t)$] ++(0, -2) node[](o2){} -| (0, 0);
		\draw (o1) -- ++(2, 0) to[open, o-o, v^=$V_{out}$] ++(0, -2) -| (o2);
	\end{circuitikz}
	\caption{A LR circuit}
	\label{fig:lrfilter.ac}
\end{figure}
The solution is almost immediate, in fact we have
\begin{equation}
	V_{out}(t)=\frac{Z_R}{Z_L+Z_R}V_{in}(t)=\frac{R}{i\omega L+R}V_{in}(t)=\frac{1}{1+i\omega\tau}V_{in}(t)
	\label{eq:voutlrfilter.ac}
\end{equation}
Where we defined the relaxation time of the circuit $\tau$ as 
\begin{equation}
	\tau_{RL}=\frac{R}{L}
	\label{eq:relaxtimerl.ac}
\end{equation}
The input-output voltage ratio is then
\begin{equation}
	\frac{V_{out}}{V_{in}}=\frac{1}{1+i\omega\tau}
	\label{eq:vinoutrl.ac}
\end{equation}
Which clearly indicates this circuit as a \textit{low pass filter}.\\
For a RL circuit, as in the following diagram instead
\begin{figure}[H]
	\centering
	\begin{circuitikz}
		\draw (0, 0) to[sV, l=$V_{in}(t)$] ++(0, 3) -- ++(4, 0) to[R, l=$R$, v=$V_R(t)$] ++(0, -2) node[left](o1){} to[L, l=$L$, v=$V_L(t)$] ++(0, -2) node[](o2){} -| (0, 0);
		\draw (o1) -- ++(2, 0) to[open, o-o, v^=$V_{out}$] ++(0, -2) -| (o2);
	\end{circuitikz}
	\caption{A LR circuit}
	\label{fig:lrfilter.ac}
\end{figure}
As before, we have
\begin{equation}
	V_{out}=\frac{Z_L}{Z_L+Z_R}V_{in}(t)=\frac{i\omega L}{i\omega L+R}V_{in}(t)
	\label{eq:voutrl.ac}
\end{equation}
The ratio of outgoing voltage is then clearly dependent on $\omega$ and thus this circuit is a \textit{high pass filter}
\section{Integrator and Derivator Circuits}
A special behavior of filters is to output a voltage which is proportional to either the derivative or the integral of the input voltage.\\
\subsection{Integrator Circuits}
We begin by considering a RC circuit. We must have that
\begin{equation*}
	V_{in}(t)=V_{C}(t)+V_R(t)
\end{equation*}
But, if
\begin{equation*}
	\abs{V_C(t)}<<\abs{V_R(t)}
\end{equation*}
Then 
\begin{equation*}
	V_{in}(t)\approx V_{R}(t)=RI(t)\implies I(t)=\frac{1}{R}V_{in}(t)
\end{equation*}
Thus, the outgoing voltage, $V_{out}(t)=V_C(t)$ is
\begin{equation}
	V_{out}(t)=\frac{1}{C}\int_{0}^{t}I(t)\dd^{}{t}=\frac{1}{RC}\int_{0}^{t}V_{in}(t)\dd^{}{t}=\frac{1}{\tau}\int_{0}^{t}V_{in}(t)\dd^{}{t}
	\label{eq:outgoingrc.int}
\end{equation}
Due to this, RC circuits are \textit{integrators}.\\
Another example of integrator circuit is the LR circuit. As before, we must have
\begin{equation*}
	V_{in}(t)=V_{R}(t)+V_{L}(t)
\end{equation*}
When $\abs{V_R(t)}<<V_{L}(t)$, we have that $V_{out}=V_R$
\begin{equation*}
	V_{in}(t)\approx V_L(t)=L\dv{I}{t}
\end{equation*}
And therefore
\begin{equation}
	V_{R}(t)=RI(t)=\frac{R}{L}\int_{0}^{t}V_{in}(t)\dd^{}{t}+RI(0)=\frac{1}{\tau}\int_{0}^{t}V_{in}(t)\dd^{}{t}+RI(0)
	\label{eq:outgoinglr.int}
\end{equation}
\subsection{Derivator Circuits}
Consider now the previous circuits with components inverted. In the first case we will have the CR circuit, where, if 
\begin{equation*}
	\abs{V_R(t)}<<\abs{V_C(t)}
\end{equation*}
We have that the input voltage is approximately equal to the capacitor voltage, thus
\begin{equation*}
	V_{in}(t)\approx V_C(t)=\frac{1}{C}\int_{0}^{t}I(t)\dd^{}{t}
\end{equation*}
This clearly implies that
\begin{equation*}
	\dv{V_{in}}{t}=\frac{1}{C}I(t)
\end{equation*}
Which, consequently, implies that the outgoing voltage will be proportional to this derivative, in fact, since $V_{out}\approx V_R$ with these conditions
\begin{equation}
	V_R(t)=RI(t)=RC\dv{V_{in}}{t}=\tau\dv{V_{in}}{t}
	\label{eq:croutput.der}
\end{equation}
The evaluation is similar in the RL circuit, where, when
\begin{equation*}
	\abs{V_L(t)}<<\abs{V_R(t)}\implies V_{in}(t)\approx V_R, \qquad V_{out}\approx V_L
\end{equation*}
we have
\begin{equation*}
	V_{in}(t)=RI(t)\implies I(t)=\frac{1}{R}V_{in}(t)
\end{equation*}
Which, consequently implies
\begin{equation}
	V_{L}(t)=L\dv{I}{t}=\frac{L}{R}\dv{V_{in}}{t}=\tau\dv{V_{in}}{t}
	\label{eq:lroutput.der}
\end{equation}
Thus RL circuits behave as derivator circuits, as do CR circuits.\\
In general we have that:
\begin{itemize}
\item Low pass filters (RC, LR circuits) behave as voltage integrators
\item High pass filters (CR, RL circuits) behave as voltage derivators
\end{itemize}
The more is the frequency in the band of frequencies passed by the filter, the more the circuit behaves like an integrator or a derivator
\section{RLC Circuits}
\subsection{RLC Circuit in Series}
A special kind of filter is the RLC circuit, described in the following diagram
\begin{figure}[H]
	\centering
	\begin{circuitikz}
		\draw (0, 0) to[sV, l=$V(t)$, i=$I(t)$] ++(0, 4) to[L, l=$L$, v=$V_L(t)$, v=$V_L(t)$] ++(5, 0) to[R, l=$R$, v=$V_R(t)$] ++(0, -2) to[C, v=$V_C(t)$, l=$C$] ++(0, -2) -| (0, 0);
	\end{circuitikz}
	\caption{RLC Filter diagram}
	\label{fig:rlcdiagram.ac}
\end{figure}
Remembering that the voltages are
\begin{equation*}
	\begin{paligned}
		V_R(t)&= RI(t)\\
		V_L(t)&= L\dv{I}{t}\\
		V_C(t)&= \frac{1}{C}\int_{0}^{t}I(t)\dd^{}{t}
	\end{paligned}
\end{equation*}
Thus, due to Kirchhoff's mesh law, we have that the voltage $V(t)$ must be equal to the sum of these three, and deriving with respect to $t$ in order to change the integro-differential equation into a second order differential equation, we have
\begin{equation}
	\dv{V}{t}=L\dv[2]{I}{t}+R\dv{I}{t}+\frac{1}{C}I(t)
	\label{eq:rlccircode.ac}
\end{equation}
We can solve this equation with the usual techniques of ordinary differential equation theory, thus searching for the sum of an homogeneous solution and a particular solution. For the homogeneous equation we have
\begin{equation}
	\begin{paligned}
		L\dv[2]{I}{t}+R\dv{I}{t}+\frac{1}{C}I(t)&= 0\\
		L\lambda^2+R\lambda+\frac{1}{C}\lambda&= 0
	\end{paligned}
	\label{eq:rlchomo.ac}
\end{equation}
Solving the equation and the characteristic polynomials we reach that
\begin{equation*}
	\lambda=-\frac{R}{2L}\pm\sqrt{\left( \frac{R}{2L} \right)^2-\frac{I}{LC}}
\end{equation*}
Which implies
\begin{equation}
	I(t)=I_{\pm}e^{-\frac{Rt}{2L}\pm t\sqrt{\left( \frac{R}{2L} \right)^2-\frac{1}{LC}}}
	\label{eq:currentrlc.ac}
\end{equation}
We simplify everything by substitution with the following parameters
\begin{equation*}
	\begin{paligned}
		a&= \frac{R}{2L}\\
		b&= \sqrt{\left( \frac{R}{2L} \right)^2-\frac{1}{LC}}
	\end{paligned}
\end{equation*}
We can rewrite the current as the simple function
\begin{equation}
	I(t)=I_+e^{-(a-b)t}+I_-e^{-(a+b)t}
	\label{eq:simpleirlc.ac}
\end{equation}
We get the following cases
\begin{equation}
	\begin{dcases}
		\left( \frac{R}{2L} \right)^2>\frac{1}{LC}& b\in\R\\
		\left( \frac{R}{2L} \right)^2>\frac{1}{LC}& b=0\\
		\left( \frac{R}{2L} \right)^2>\frac{1}{LC}& b\in\Cf
	\end{dcases}
	\label{eq:cases}
\end{equation}
We can now begin to evaluate the transient response of the circuit
\subsubsection{Overdamped Response}
The first case that we will consider is when $b\in\R$, $b\ne 0$, the circuit with these conditions is said to be \textit{overdamped}. For physical reasons we can say immediately that $I(0)=0$, thus, imposing Cauchy conditions to the current solution and the characteristic equation of the circuit we have
\begin{equation*}
	\begin{paligned}
		I(0)=I_{+}+I_{-}= 0\\
		RI(0)+L\left.\dv{I}{t}\right|_{t=0}+\frac{q_0}{C}&= 0
	\end{paligned}
\end{equation*}
Deriving the homogeneous solution we have
\begin{equation*}
	\begin{aligned}
		\dv{I}{t}&= I_+\left( b-a \right)e^{-(a-b)t}-I_{-}(a+b)e^{-(a+b)t}\\
		\left.\dv{I}{t}\right|_{t=0}&= I_{+}(b-a)-I_{-}(a+b)=2bI
	\end{aligned}
\end{equation*}
Where we considered that $I_{-}=-I_{+}=I_0$. Inserting it into the equation we get
\begin{equation*}
	2LbI_0=-\frac{q_0}{C}\implies I_0=-\frac{q_0}{2bLC}
\end{equation*}
Indicating the \textit{natural frequency} of the circuit $\omega_0$ as
\begin{equation}
	\omega_0=\sqrt{\frac{1}{LC}}
	\label{eq:natfreq.rlc}
\end{equation}
We have finally, remembering that $I(t)=I_0f(t)$
\begin{equation*}
	I(t)=-\frac{q_0\omega_0^2}{2b}e^{-at}\left( e^{bt}-e^{-bt} \right)
\end{equation*}
Substituting the exponentials with the hyperbolic sine function we get
\begin{equation}
	I(t)=-\frac{q_0\omega_0^2}{b}e^{-at}\sinh(bt)=-\frac{q_0\omega_0^2}{\sqrt{\left( \frac{R}{2L} \right)^2-\frac{1}{LC}}}e^{-\frac{R}{2L}t}\sinh\left( t\sqrt{\left( \frac{R}{2L} \right)^2-\frac{1}{LC}} \right)
\label{eq:overdamped.rlc}
\end{equation}
\subsubsection{Critically Damped Response}
The second case is when $b=0$, i.e. when the circuit is \textit{critically damped}. Going back to the definition of $a, b$ we have that $b=0$ implies
\begin{equation*}
	\left( \frac{R}{2L} \right)^2=\frac{1}{LC}
\end{equation*}
Thus, considered the characteristic polynomial of the differential equation
\begin{equation*}
	L\lambda^2+R\lambda+\frac{1}{C}\lambda=0
\end{equation*}
Which, has solutions 
\begin{equation*}
	\lambda_{12}=a\pm b
\end{equation*}
Thus, being $b=0$ with this circuit response, the solution to the polynomial is a degenerated single solution with an algebraic multiplicity of 2, $\lambda=a$, thus the homogeneous solution for the current is
\begin{equation}
	I(t)=I_1e^{-\frac{R}{2L}t}+tI_2e^{-\frac{R}{2L}t}=\left( I_1+tI_2 \right)e^{-at}
	\label{eq:critdampIho.rlc}
\end{equation}
Imposing the initial condition $I(0)=0$ we then get
\begin{equation*}
	I(0)=I_1=0\implies I(t)=tI_2e^{-at}\qquad I_2=I_0
\end{equation*}
And
\begin{equation*}
	\left.\dv{I}{t}\right|_{t=0}=\left.\left( 1-at \right)I_0e^{-at}\right|_{t=0}=I_0
\end{equation*}
We have then 
\begin{equation*}
	LI_0=-\frac{q_0}{C}\implies I_0=-\frac{q_0}{LC}=-q_0\omega_0^2
\end{equation*}
Thus, the current function in a critically damped RLC is
\begin{equation}
	I(t)=I_0te^{-at}=-q_0\omega_0^2te^{-at}=-q_0\omega_0^2te^{-\frac{R}{2L}t}
	\label{eq:criticaldamp.rlc}
\end{equation}
\subsubsection{Damped Oscillatory Response}
The last case is the \textit{damped oscillatory regime}, i.e. when
\begin{equation*}
	\left( \frac{R}{2L} \right)^2<\omega_0^2\implies \lambda_{\pm}=-a\pm ib
\end{equation*}
We have an oscillatory solution to the current.\\
Imposing the Cauchy condition $I(0)=0$ we have
\begin{equation*}
	I(0)=I_{+}+I_{-}=0\implies I_{+}=-I_{-}
\end{equation*}
Thus, since
\begin{equation*}
	I(t)=\left( I_{+}e^{ibt}+I_{-}e^{-ibt} \right)e^{-at}
\end{equation*}
We get
\begin{equation*}
	I(t)=I_0e^{-at}\left( e^{ibt}-e^{-ibt}\right)=2i I_0e^{-at}\sin(bt)
\end{equation*}
Noting that
\begin{equation*}
	I_0=-\frac{q_0}{2bLC}=-\frac{q_0\omega_0^2}{2b}
\end{equation*}
We get the final solution for the current in an RLC circuit in damped oscillatory conditions
\begin{equation}
	I(t)=-\frac{i\omega_0^2q_0}{b}e^{-at}\sin\left( bt \right)=-\frac{i\omega_0^2q_0}{\sqrt{\left( \frac{R}{2L} \right)^2-\frac{1}{LC}}}e^{-\frac{R}{2L}t}\sin\left( t\sqrt{\left( \frac{R}{2L} \right)^2-\frac{1}{LC}} \right)
	\label{eq:oscillatorydampedI.rlc}
\end{equation}
Note that although the current is harmonic in this configuration, there's still Joule dissipation on the resistor $R$.
\subsection{General Solution in AC Regimes}
We can now evaluate the general solution to the circuit equation with a sinusoidal voltage source. The equation we need to solve is
\begin{equation*}
	\dv{V}{t}=L\dv[2]{I}{t}+R\dv{I}{t}+\frac{I}{C}
\end{equation*}
Where
\begin{equation*}
	\begin{paligned}
		V(t)&= V_0e^{i\omega t+i\varphi_V}\\
		I(t)&= I_0e^{i\omega t+i\varphi_I}
	\end{paligned}
\end{equation*}
Substituting into the equation we get
\begin{equation*}
	i\omega V_0e^{i\varphi_V}=-\omega^2LI_0e^{i\varphi_I}+i\omega Re^{i\varphi_I}+\frac{I_0}{\omega_0C}e^{i\varphi_I}
\end{equation*}
Rearranging the equation and multiplying both sides by $e^{-i\varphi_I}$ we have
\begin{equation}
	V_0e^{i\left( \varphi_V-\varphi_I \right)}=\left[ R+\left(i\omega L-\frac{1}{i\omega C}  \right) \right]I_0=Z_{RLC}I_0
	\label{eq:solutionrlc.rlc}
\end{equation}
We get the values for $I_0$ and $\varphi=\varphi_V-\varphi_I$
\begin{equation}
	\begin{paligned}
		I_0&= \frac{V}{\abs{Z_{RLC}}}=\frac{V_0}{\sqrt{R^2+\left( \omega L-\frac{1}{\omega C} \right)^2}}V_0\\
		\varphi&= \arctan\left[ \frac{1}{R}\left( \omega L-\frac{1}{\omega C} \right) \right]
	\end{paligned}
	\label{eq:rlcfinal.rlc}
\end{equation}
Note how $Z_{RLC}$ is exactly what we'd get from the generalized ohm law, and that also $Z_{RLC}\in\Cf$ in general.\\
Also $Z_{RLC}\in\R$ only in the special resonant case of $\omega=\omega_0$, where $Z_{RLC}=R$ and $\varphi=0$
\subsection{Quality Factor}
\begin{dfn}[Quality Factor]
In order to understand better the behavior of the RLC circuit we can define a parameter, known as the \textit{quality factor} as follows
\begin{equation}
	Q_0=\frac{\omega_0 L}{R}=\frac{1}{R}\sqrt{\frac{L}{C}}
	\label{eq:qualityfactor.rlc}
\end{equation}
\end{dfn}
Thus, the current we evaluated previously can be rewritten as follows
\begin{equation}
	I_0=\frac{V_0}{\sqrt{1+Q_0^2\left( \frac{\omega^2-\omega_0^2}{\omega\omega_0} \right)^2}}
	\label{eq:i0quality.rlc}
\end{equation}
It's clear that only some frequencies will pass through the filter, defining the RLC circuit as a \textit{band pass filter}. Note that the closer is the quality factor to infinity, the more selective the filter is\\
\begin{dfn}[Bandwidth]
	We define the \textit{bandwidth} $\Delta\omega$ of the RLC band pass filter as the distance between the points of the frequency response where
	\begin{equation}
		I_{\Delta\omega}=\frac{1}{\sqrt{2}}I_{max}
		\label{eq:bandwidthcond.rlc}
	\end{equation}
	It's clear then that the bandwidth depends on the quality factor of the circuit
\end{dfn}
In order to evaluate the bandwidth of the filter we begin by finding where $I/I_0$ is equal to $1/\sqrt{2}$ Thus, we have
\begin{equation*}
	\frac{I}{I_0}=\frac{1}{\sqrt{2}}=\frac{1}{\sqrt{1+Q_0^2\left( \frac{\omega^2-\omega_0^2}{\omega\omega_0} \right)^2}}
\end{equation*}
Clearly we must solve 
\begin{equation*}
	Q_0^2\left( \frac{\omega^2-\omega_0^2}{\omega\omega_0} \right)^2=1
\end{equation*}
Thus, we have
\begin{equation*}
	\frac{\omega^2-\omega_0^2}{\omega\omega_0}=\pm\frac{1}{Q_0}\implies\omega^2-\omega_0^2\mp\frac{\omega\omega_0}{Q_0}=0
\end{equation*}
Solving for $\omega$ in order to find the intersections we get
\begin{equation}
	\omega=\pm\frac{\omega_0}{2Q_0}\pm\sqrt{\left( \frac{\omega_0}{2Q_0} \right)^2+\omega_0}
	\label{eq:freqsolbandwidth.rlc}
\end{equation}
Imposing only positive solutions, since $\omega>0$ \textit{always} we get that the bandwidth is 
\begin{equation}
	\Delta\omega=\omega_2-\omega_1=\frac{\omega_0}{Q_0}=\frac{R}{L}
	\label{eq:bandwidth.rlc}
\end{equation}
For commercial filters we have that $Q_0\propto10^2$. For getting higher quality factors expensive components are needed, like inductors composed by superconductors
\subsection{Ohm's Law and RLC Circuits}
\subsubsection{RLC in Series}
We can use Ohm's law in order to evaluate RLC circuits and find all the parameters we need. We start with an RLC circuit in series, with a resistor $R$ as the circuit load, as in the following diagram
\begin{figure}[H]
	\centering
	\begin{circuitikz}
		\draw (0, 0) to[sV, l=$V_G(t)$] (0, 2) to[R, l=$R_G$] (0, 4) to[R, l=$R_L$] (2, 4) to[L, l=$L$] (4, 4) to[C, l=$C$] (6, 4) -- (7, 4) to[open, o-o, v^=$V_{out}(t)$] (7, 0) -| (0, 0);
		\draw (6, 4) to[R, l=$R$] (6, 0);
	\end{circuitikz}
	\caption{RLC circuit with a real generator and a real inductor, every component is connected in series to a load $R$}
	\label{fig:rlcseriesohm.rlc}
\end{figure}
Since everything is in series we have that the equivalent impedance of the circuit is $Z_{eq}$ is
\begin{equation*}
	Z_{eq}=Z_G+Z_L+Z_C
\end{equation*}
Where $Z_G$ and $Z_L$ are respectively the \textit{total} impedances of the generator and inductor. Thus
\begin{equation}
	Z_{eq}=R_G+\left( R_L+i\omega L \right)+\frac{1}{i\omega C}
	\label{eq:zeqseries.rlc}
\end{equation}
Thanks to $V=ZI$ we have
\begin{equation}
	I(t)=\frac{V_{out}(t)}{Z}=\frac{V_G(t)}{R_G+R_L+i\left( \omega L-\frac{1}{\omega C} \right)}
	\label{eq:currentseriesrlc.rlc}
\end{equation}
Writing $R_0=R_L+R_G$ we can evaluate from the previous expression the quality factor, as
\begin{equation*}
	Q_s=\frac{\omega_0 L}{R_0}=\frac{1}{R_G+R_L}\sqrt{\frac{L}{C}}
\end{equation*}
Looking back to the current expression, we can also write the voltage on the inductor and capacitor, where
\begin{equation*}
	\begin{paligned}
		V_C(t)&= Z_CI(t)=\frac{Z_C}{Z_{eq}}V_G(t)=\frac{1}{i\omega C}\frac{V_G(t)}{R_G+R_L+i\left( \omega L+\frac{1}{\omega C} \right)}\\
		V_L(t)&= Z_LI(t)=\frac{Z_L}{Z_{eq}}V_G(t)=i\omega L\frac{V_G(t)}{R_G+R_L+i\left( \omega L+\frac{1}{\omega C} \right)}
	\end{paligned}
\end{equation*}
Thus, evaluating the ratio between the measured output and input voltage, we have  that
\begin{equation*}
	\begin{paligned}
		\frac{V_C^M}{V_G^M}&= \frac{1}{\omega C}\frac{1}{\sqrt{R^2}+\left( \omega L-\frac{1}{\omega C} \right)^2}\\
		\frac{V_L^M}{V_G^M}&= \frac{\omega L}{\sqrt{R^2+\left( \omega L-\frac{1}{\omega C} \right)^2}}
	\end{paligned}
\end{equation*}
When at resonance, i.e. when $\omega=\omega_0$, then we have
\begin{equation*}
	\frac{V_C^M}{V_G^M}=\frac{V_L^M}{V_G^M}=Q_s\implies V_C^M=V_L^M=Q_sV_G^M
\end{equation*}
Being $Q>1$, we also have that both $V_C^M, V_C^M>V_G$ and we are actually measuring a higher voltage on the two components.\\
Remember that $\varphi_C=-\varphi_L$, therefore \textit{there is no current flow on the load}.
\subsubsection{RLC in Parallel}
We now consider the dual version of the RLC circuit in series, the RLC circuit in parallel, which has the following diagram
\begin{figure}[H]
	\centering
	\begin{circuitikz}
		\draw (0, 4) to[sV, l=$V_G(t)$, i=$I(t)$] (0, 0);
		\draw (0, 4) -- (8, 4) to[open, o-o, v^=$V_{out}(t)$] (8, 0) -- (0, 0);
		\draw (3, 4) to[L, l=$L$, i=$I_L(t)$] (3, 0);
		\draw (5, 4) to[C, l=$C$, i=$I_C(t)$] (5, 0);
		\draw (7, 4) to[R, l=$R$, i=$I_R(t)$] (7, 0);
	\end{circuitikz}
	\caption{RLC circuit in parallel}
	\label{fig:RLCparallel.rlc}
\end{figure}
The evaluation is completely analogue, we just gotta remember that
\begin{equation}
	\frac{1}{Z_{eq}}=\frac{1}{Z_R}+\frac{1}{Z_C}+\frac{1}{Z_L}=\frac{1}{R}+i\left( \omega C-\frac{1}{\omega L} \right)
\end{equation}
Which therefore implies
\begin{equation}
	I(t)=\frac{V_G(t)}{Z_{eq}}=\left[ \frac{1}{R}ì\left( \omega C-\frac{1}{\omega L} \right) \right]V_G(t)
\end{equation}
And, thus
\begin{equation*}
	I_0^M=V_G^M\sqrt{\frac{1}{R^2}+\left( \omega C-\frac{1}{\omega L} \right)^2}
\end{equation*}
And the phase of the current will be
\begin{equation}
	\varphi=\arctan\left[ R\left( \omega C-\frac{1}{\omega L} \right) \right]
\end{equation}
\section{Transfer Functions}
Thanks to the circuits being linear, for which the superposition theorem holds, we can imagine to write the output of two circuits combined using some function.
\begin{dfn}[Transfer Function]
	Given two circuits $C_1$ and $C_2$, said $V_{i1}$, $V_{o1}$ and $V_{i2}$, $V_{o2}$ the input and output voltages of the two circuits, we can define a \textit{transfer function} of the two as
	\begin{equation}
		\begin{aligned}
			T_{1}(\omega)&= \frac{V_{i1}}{V_{o1}}\\
			T_{2}(\omega)&= \frac{V_{i2}}{V_{o2}}
		\end{aligned}
		\label{eq:transferfuncdef.trf}
	\end{equation}
	These functions completely describe the behavior of the two circuits.\\
	Note that when we connect the two circuits
	\begin{equation*}
		T_{12}\ne T_1T_2
	\end{equation*}
	Since the connection of the second circuit to the first induces a modification of the response of the circuits at the connection point. If this modification is $V_x(\omega)$, we can tho write
	\begin{equation}
		T_{12}=\frac{V_{o2}}{V_{x}}\frac{V_x}{V_{i1}}
		\label{eq:transferfunc2.trf}
	\end{equation}
\end{dfn}
This can be in general be computed by using the formula of the voltage divider. Suppose to have two circuits connected one after the other. If the equivalent impedance of the first circuit is $Z_{i1}$, the input voltage $V_{i1}$, with equivalent output voltage $V_{o1}$ and output impedance $Z_{o1}$ and the second circuit has an equivalent output voltage $V_{o2}$ with equivalent impedance $Z_{o2}$, we get that
	\begin{equation*}
		V_{i2}=\frac{Z_{i2}}{Z_{i2}+Z_{o1}}V_{o1}
	\end{equation*}
	Thus
	\begin{equation*}
		T_{12}=\frac{V_{o2}}{V_{i1}}=\frac{V_{o2}}{V_{i2}}\frac{V_{i2}}{V_{i1}}=\frac{V_{o2}}{V_{i2}}\frac{Z_{i2}}{Z_{i2}+Z_{o1}}\frac{V_{o1}}{V_{i1}}
	\end{equation*}
	Substituting the first equation into the second we get
	\begin{equation}
		T_{12}=\frac{Z_{i2}}{Z_{i2}+Z_{o1}}T_2T_1
		\label{eq:transferfunc2circ.trf}
	\end{equation}
Let's see this in practice by connecting an RC circuit to a CR circuit as follows
\begin{figure}[H]
	\centering
	\begin{circuitikz}
		\draw (-1, 0) to[open, o-o, v^=$V_{i1}$] (-1, 4) to[R, l=$R_1$] (2, 4) -- (4, 4) to[open, *-*] (4, 0) -- (-1, 0);
		\draw (2, 4) to[C, l=$C_1$] (2, 0);
		\draw (4, 0) to[open, v_=$V_x$] (4, 4) to[C, l=$C_2$] (6, 4) -- (8,4) to[open, o-o, v^=$V_{o2}$] (8, 0) -- (4, 0);
		\draw (6, 4) to[R, l=$R_2$] (6, 0);
	\end{circuitikz}
	\caption{Cascade connection of a RC and CR circuit}
	\label{fig:rccr.trf}
\end{figure}
As for before we have
\begin{equation*}
	V_x=\frac{Z_{i2}}{Z_{i1}+Z_{i2}}V_{i1}
\end{equation*}
Where $Z_{i2}$ is the equivalent impedance of $C_2, R_2$ and $C_1$. Thus
\begin{equation*}
	Z_{i2}=\frac{1}{i\omega C_1}+R_2+\frac{1}{i\omega C_2}
\end{equation*}
Thus, re-inputting it into the previous equation we have
\begin{equation*}
	V_x=\frac{V_{i1}}{1+i\omega R_1C_1+\frac{R_1}{R_2+\frac{1}{i\omega C_2}}}
\end{equation*}
But we also have
\begin{equation*}
	V_{o2}=\frac{R_2}{R_2+Z_{C2}}V_{x}=\frac{i\omega C_2R_2}{1+i\omega C_2R_2}V_x
\end{equation*}
Thus
\begin{equation*}
	V_{o2}=\frac{i\omega C_2R_2}{1+i\omega C_2R_2}\frac{V_{i1}}{1+i\omega R_1C_1+\frac{i\omega C_2R_1}{1+i\omega C_2R_2}}
\end{equation*}
Which then implies
\begin{equation}
	T(\omega)=\frac{i\omega C_2R_2}{\left( 1+i\omega C_2R_2 \right)\left( 1+i\omega R_1C_1 \right)+i\omega C_2R_1}
	\label{eq:trfrccr.tfr}
\end{equation}
\section{Dissipated Power in AC}
Consider a simple capacitor connected to an ideal voltage generator in an alternate current regime
\begin{figure}[H]
	\centering
	\begin{circuitikz}
		\draw (0, 0) to[sV, l=$V(t)$] ++(0, 3) -- ++(4, 0) to[C, l=$C$, i=$I(t)$] ++(0, -3) -- (0, 0);
	\end{circuitikz}
	\caption{Simple circuit in AC regime}
	\label{fig:simpleac.pow}
\end{figure}
The dissipated power in this sinusoidal regime is as usual 
\begin{equation}
	P(t)=V(t)I(t)
	\label{eq:powerac.pow}
\end{equation}
Thus, using Ohm's law
\begin{equation*}
	V(t)=ZI(t)=\frac{I(t)}{i\omega C}\implies I(t)=i\omega CV(t)
\end{equation*}
Note that in this special case, being the current a perfectly imaginary number we will have $\varphi_I=\frac{\pi}{2}$.\\
Since $V(t)$ is sinusoidal, we have that $P(t)$ can be positive or negative depending on which phase of the cycle we are in. When $P(t)>0$ we are giving power to the capacitor, i.e. charging the capacitor, and vice versa when it's negative.\\
In general we must then have, if the current and voltage are in phase
\begin{equation}
	\expval{P}=\frac{1}{T}\int_{0}^{T}V(t)I(t)\dd^{}{t}=\frac{\omega}{2\pi}\int_{0}^{\frac{\omega}{2\pi}}V_0I_0\sin\left( \omega t \right)\cos\left( \omega t \right)\dd^{}{t}=0
	\label{eq:inphase.pow}
\end{equation}
But, when current and voltage \textit{are not} in phase, we have
\begin{equation*}
	\cos\left( \omega t+\varphi_I \right)\cos\left( \omega t \right)=\cos^2\left( \omega t \right)\cos\left( \varphi_I \right)+\sin\left( \omega t \right)\cos\left( \omega t \right)\sin\left( \varphi_I \right)
\end{equation*}
Thus
\begin{equation*}
	\expval{P}=\frac{V_0I_0}{T}\cos\left( \varphi_I \right)\int_{0}^{T}\cos^2\left( \omega t \right)\dd^{}{t}
\end{equation*}
Integrating in one period we have
\begin{equation}
	\expval{P}=\frac{V_0I_0}{2}\cos\left( \varphi_I \right)
	\label{eq:exppownphase.pow}
\end{equation}
If we define $I_{eff}=I_0/\sqrt{2}$, $V_{eff}=V_0/\sqrt{2}$ as the \textit{effective} current and voltage, we have
\begin{equation}
	\expval{P}=V_{eff}I_{eff}\cos\left( \varphi_I \right)
	\label{eq:peff.pow}
\end{equation}
The cosine of the current phase is known as the \textit{power factor}.\\
For circuits which are purely active the real part of the impedance is zero, and with it the power factor, thus
\begin{equation*}
	\expval{P}_{active}=0
\end{equation*}
Note tho that being $I(t)\ne0$ inside the generator, there will be dissipated power in $R$ thanks to Joule dissipation
\subsection{Quality Factor as a Measure of Dissipated Energy}
Consider the definition of the quality factor $Q_0$. We have
\begin{equation*}
	Q_0=\omega_0\frac{L}{R}
\end{equation*}
We can write, at resonance $\omega_0=\frac{2\pi}{T_0}$, thus with some algebraic play we have
\begin{equation*}
	Q_0=\frac{2\pi}{T}\frac{L}{R}\frac{I_0^2/2}{I_0^2/2}=\frac{2\pi}{T}\frac{2}{RI_0^2}\frac{LI^2_0}{2}
\end{equation*}
We immediately recognize the two following quantities
\begin{equation*}
	\begin{aligned}
		\frac{1}{2}LI_0^2&= E_{M}\\
		\frac{1}{2}RI_0^2&= P_J
	\end{aligned}
\end{equation*}
I.e. the magnetic energy inside the inductor, and the dissipated power in the resistor via Joule effect. Thus
\begin{equation}
	Q_0=\frac{2\pi}{T}\frac{E_M}{P_J}=2\pi\frac{E_M}{E_J}
	\label{eq:energylossratio.qfe}
\end{equation}
Thus, the quality factor $Q_0$ can be seen as $2\pi$ times the energy inside the resonator (the inductor) and the dissipated energy in the resistor.
\section{Diodes}
\subsection{Semiconductor Diodes}
In electronics we can find circuit components which are not linear. These nonlinear components have as a main factor that $I\not\propto V$, thus, components for which Ohm's law doesn't work.\\
One special kind of nonlinear component is the \textit{diode}. The most common types of diodes use Silica or Germanium component, both tetravalent semi-metals.\\
In order to ease the conduction in these elements, their crystalline structure is \textit{doped} with impurities e.g., for silica
\begin{itemize}
\item Phosphorus doping, adding one valence electron
\item Gallium doping, removing one valence electron, i.e. adding one \textit{hole} 
\end{itemize}
This doping elements are added in very small parts, usually 1 part per million, but still manage to increase the conductivity of the silica structure by an order of $10^2$.\\
In diodes, in order to obtain these particular structures, a \textit{junction} is created between two doped semiconductors. This junction is known as a \textit{NP} or \textit{PN} junction, representing the connection of:
\begin{itemize}
\item N crystals, doped mainly with P, As, An, i.e. doped with pentavalent atoms
\item P crystals, doped mainly with B, Ga, In, i.e. doped with trivalent atoms
\end{itemize}
The connection between these crystals inside a diode creates an electric dipole field between the two junctions, thus generating an electric potential $\varphi$.\\
The extra electrons in the N crystal manage to jump the potential, going towards the P crystal if and only if $E_{e^-}>\varphi$.\\
These electrons are in thermal equilibrium, thus we can use Boltzmann statistics in order to evaluate them.\\
We want to find the number of electrons with energy $\dd E$. Indicating $\beta=\frac{1}{kT}$ we have
\begin{equation*}
	\dd N=e^{-\beta E}\dd R
\end{equation*}
Thus
\begin{equation}
	N\left( E_{e^-}>\varphi \right)=C\int_{\varphi}^{\infty}e^{-\beta E}\dd^{}{E}=\frac{C}{\beta}e^{-\beta\varphi}
	\label{eq:electronsjumpn.dio}
\end{equation}
The probability of having electrons with energies greater than $\varphi$ that actually manage to jump is then, if $N_T$ is the total number of electrons
\begin{equation}
	P\left( jump \right)=\frac{N\left( E_{e^-}>\varphi \right)}{N_T}=e^{-\beta\varphi}
	\label{eq:jumpprob.dio}
\end{equation}
And thus, the current between the junction is $I_{NP}\propto P\left( jump \right)$.\\
At equilibrium we will have $I_{NP}=I_{PN}$. 
\subsection{Voltage Rectifiers}
We now begin to evaluate diodes inside of circuits. Consider a diode connected to some battery, which will add a new electric field $\vec{E}'$. If the battery has a voltage $V$, we will have, inside the junctions
\begin{equation}
	E_{e^-}'=\varphi-e\abs{V}
	\label{eq:endiodirectpol.dio}
\end{equation}
Thus, the diode will be \textit{directly polarized}, and
\begin{equation}
	\begin{aligned}
		I_{NP}&= Ae^{-\beta\left( \varphi-e\abs{V} \right)}\\
		I_{PN}&= Ae^{-\beta\varphi}
	\end{aligned}
	\label{eq:directpolcurrs.dio}
\end{equation}
The total current between the junctions is then
\begin{equation}
	I_{e^-}=I_{NP}-I_{PN}=Ae^{-\beta\varphi}\left( e^{\beta e\abs{V}}-1 \right)
	\label{eq:totcurr}
\end{equation}
This direct polarization can inverted, thus obtaining for the circuit
\begin{equation}
	I=I_0\left( e^{-\beta eV}-1 \right)
	\label{eq:diodecurr.dio}
\end{equation}
This can be used for building \textit{rectifiers}.\\
There are two main kinds of rectifiers, \textit{half-wave rectifiers}, and \textit{bridge rectifiers}. A half-wave rectifier is built as follows
\begin{figure}[H]
	\centering
	\begin{circuitikz}
		\draw (0, 0) to[sV, l=$V_{in}(t)$] (0, 3) to[D] (5, 3) to[open, o-o, v^=$V_{out}(t)$] (5, 0) -- (0, 0);
		\draw (4, 3) to[R, l=$R$] (4, 0);
	\end{circuitikz}
	\caption{Half-wave rectifier}
	\label{fig:hwrect.dio}
\end{figure}
In this circuit we will have
\begin{equation}
	\begin{aligned}
		V_{in}(t)&\propto\sin\left( \omega t \right)\\
		V_{out}(t)&\propto\begin{dcases}
			\sin(\omega t)&V_{in}(t)>0\\
			0&V_{out}(t)<0
		\end{dcases}
	\end{aligned}
	\label{eq:hwrectvolt.dio}
\end{equation}
Thus, as output, we will get only the positive half-wave of the input voltage.\\
A bridge rectifier instead creates with a good approximation DC current from an AC source, and they're built as follows
\begin{figure}[H]
	\centering
	\begin{circuitikz}
		\draw (0, 0) to[D, l=$4$] (-2, 2) to[D, l=$1$] (0, 4) node[left](b+){};
		\draw (0, 0) to[D, l_=$3$] (2, 2) to[D, l_=$2$] (0, 4);
		\draw (-2, 2) to[sV, l=$V_{in}(t)$] (2, 2);
		\draw (0, 4) -- (6, 4) to[open, o-o, v^=$V_{out}$] (6, 0) -- (0, 0);
		\draw (5, 4) to[R, l=$R$] (5, 0);
	\end{circuitikz}
	\caption{Bridge full-wave rectifier}
	\label{fig:voltbridge.dio}
\end{figure}
In this circuit, we have diodes 1 and 3 in \textit{inverse polarization}, thus working as PN components, while diodes 2 and 4 are directly polarized, i.e. NP components.\\
As for the voltage, we will have
\begin{equation}
	V_{out}\propto V_{in}(t)
	\label{eq:bridgeout.dio}
\end{equation}
\subsection{Zener Diodes}
For common diodes, exists a voltage $V_{B}$, also known as \textit{breakdown voltage}, for which there is \textit{full ionization} of the crystal junction, thus destroying the component.\\
A special type of diode built in order to sustain $V>V_B$ is the \textit{Zener diode}, which is used usually to stabilize DC voltages, in circuits like the following
\begin{figure}[H]
	\centering
	\begin{circuitikz}
		\draw (0, 0) to[dcvsource, l=$V$] (0, 3) to[R, l=$R$] (5, 3) to [open, o-o, v^=$V_{B}$] (5, 0) -- (0, 0);
		\draw (4, 0) to[zzD, l=$Z_D$] (4, 3);
	\end{circuitikz}
	\caption{DC Stabilizer circuit with a Zener diode $Z_D$}
	\label{fig:zenerstab.dio}
\end{figure}
Zener diodes are also used to limit sinusoidal voltages, with a \textit{clipping effect} at $V_{-}=-0.6$ V and $V_{+}=V_B$
\section{Transmission Lines}
\subsection{Electromagnetic Kirchhoff Laws}
The need to transport voltages and currents between long distances, creates a major problem.\\
Since relativity holds, the signal never travels faster than the speed of light $c$, adding a delay and a phase shift, which are negligible if and only if $\omega<< 1$ GHz.\\
These kinds of circuits are known as \textit{transmission lines}, and are indicated in circuits as follows
\begin{figure}[H]
	\centering
	\begin{circuitikz}
		\draw (0, 0) to[R, l=$Z_G$] (0, 2) to[sV, l=$V_{in}(t)$] (0, 4) -- (2, 2) to[TL, l=$Z_0$] (3, 2) -- (5, 4) to[generic, l=$Z_L$] (5, 0) -- (3, 2); 
		\draw (0, 0) -- (2, 2);
	\end{circuitikz}
	\caption{Diagram of a transmission line $Z_0$ connecting a real generator to a generic load $Z_L$}
	\label{fig:trasline.tra}
\end{figure}
The component $Z_0$, known as a \textit{transmission line}, can be schematized as follows
\begin{figure}[H]
	\centering
	\begin{circuitikz}
		\node[above left] at (0, 0) {$B$};
		\node[above left] at (0, 3) {$A$};
		\node[above right] at (9, 3) {$A$};
		\node[above right] at (9, 0) {$B$};
		\draw (0, 0) to[open, *-*] (0, 3) to[R, l=$R_A$] (2, 3) to[L, l=$L_A$] (4, 3) -- (9, 3) to[open, *-*, v^=$V_{AB}$] (9, 0);
		\draw (0, 0) to[R, l=$R_B$] (2, 0) to[L, l=$L_B$] (4, 0) -- (9, 0);
		\draw (6, 0) to[C, l=$C$] (6, 3);
		\draw (8, 0) to[R, l=$G$] (8, 3);
		\draw[<->, dashed] (0, -0.5)  -- (9, -0.5);
		\node[below] at (4.5, -0.5) {$\dd x$};
	\end{circuitikz}
	\caption{Approximation of a piece of transmission line long $\dd x$}
	\label{fig:dxtrans.tra}
\end{figure}
Using mesh calculus we can write
\begin{equation}
	\begin{paligned}
		\dd V_A&= -R_AI(x, t)\dd x-L_A\pdv{I}{t}\dd x\\
		\dd V_B&= R_BI(x, t)\dd x+L_B\pdv{I}{t}\dd x
	\end{paligned}
	\label{eq:voltdiff.tra}
\end{equation}
Thus
\begin{equation*}
	\pdv{V}{x}=-\left( R_B+R_A \right)I(x, t)-(L_A+L_B)\pdv{I}{t}
\end{equation*}
We define 
\begin{equation}
	\begin{paligned}
		R_u\dd x&= R_A+R_B\\
		L_u\dd x&= L_A+L_B
	\end{paligned}
	\label{eq:rulu.tra}
\end{equation}
And thus, we get a set of coupled differential equations for voltage and currents
\begin{equation}
	\begin{paligned}
		\pdv{V}{x}&= -\left( R_u+L_u\pdv{x}\right)I(x, t)\\
		\pdv{I}{x}&= -\left( G_u+C_u\pdv{t}\right)V(x, t)
	\end{paligned}
	\label{eq:teleq.tra}
\end{equation}
Where we indicated with $G_u$ the admittance of the line and with $C_u$ the capacitance\\
The operators inside the parentheses are known as the \textit{unitary impedance operator} $\hat{Z}_u$ and the \textit{unitary admittance operator} $\hat{Y}_u$
\subsection{Telegrapher's Equations}
In general, transmissions lines are used with sinusoidal voltage sources.\\
From the system of equations we can write, by deriving with respect to $x$ both equations
\begin{equation*}
	\begin{paligned}
		\pdv[2]{V}{x}&= \hat{Z}_u\hat{Y}_uV(x, t)\\
		\pdv[2]{I}{x}&= \hat{Z}_u\hat{Y}_uI(x, t)
	\end{paligned}
\end{equation*}
And considering that since $V\propto e^{i\omega t}$ we have 
\begin{equation*}
	\begin{paligned}
		\pdv{V}{t}&= i\omega V(x, t)\\
		\pdv{I}{t}&= i\omega I(x, t)
	\end{paligned}
\end{equation*}
And using the approximation 
\begin{equation*}
	\begin{paligned}
		\hat{Z}_u&\approx R+i\omega L\\
		\hat{Y}_u&\approx G+i\omega C
	\end{paligned}
\end{equation*}
We can write the equations as two second order uncoupled partial differential equations, known as the \textit{Telegraphist equations}
\begin{equation}
	\begin{paligned}
		\pdv[2]{V}{x}&= \gamma^2V(x, t)\\
		\pdv[2]{I}{x}&= \gamma^2I(x, t)
	\end{paligned}
	\label{eq:telegraphist.tra}
\end{equation}
The solution of these equations can be found by searching for the following solution
\begin{equation*}
	V(x)=A_1e^{-\gamma x}+A_2e^{\gamma x}, \quad A_i=a_ie^{i\varphi_i}, \quad Z_0=z_0e^{i\psi}
\end{equation*}
Where we omitted the time dependency of the voltage, it being $e^{i\omega t}$.\\
We define also
\begin{equation*}
	\gamma=\alpha+i\beta=\sqrt{R+i\omega L}\sqrt{G+i\omega C}=\sqrt{\hat{Z}_u\hat{Y}_u}
\end{equation*}
With this definition of $\gamma$ and the two complex amplitudes $A_i$ we have
\begin{equation*}
	V(x)=a_1e^{-i\left( \beta-\varphi_1 \right)}e^{-\alpha x}+a_2e^{i\left(\beta+\varphi_2 \right)}
\end{equation*}
Multiplying both sides by the time dependence and finding the measured voltage along the line, we get
\begin{equation}
	V_M(x, t)=\real\left\{ V(x, t) \right\}=a_1e^{-\alpha x}\cos\left( \omega t-\beta x+\varphi_1 \right)+a_2e^{\alpha x}\cos\left( \omega t+\beta x+\varphi_2 \right)
	\label{eq:measvolt.tra}
\end{equation}
This is clearly a wavefunction, thus, taken the solution at constant phase we have
\begin{equation}
	\omega-\beta\pdv{x}{t}=0\implies\pdv{x}{t}=\frac{\omega}{\beta}=u
	\label{eq:phasevel.tra}
\end{equation}
Which is the \textit{phase velocity} of the voltage wave inside the transmission line. Note that by definition, we have that the phase velocity is the imaginary part of $\gamma$, i.e. 
\begin{equation*}
	u=\imaginary\left\{ \gamma \right\}
\end{equation*}
\subsubsection{Non Dissipative Solution}
If we consider the ideal case of a non-dissipating transmission line, i.e. $R_u=G_u=0$, the solution is clearly way simpler, but especially we get
\begin{equation}
	\gamma=i\omega\sqrt{LC}=\frac{i\omega}{\tau}
	\label{eq:nondissgamma.tra}
\end{equation}
Where $\tau$ is again the relaxation time for the LC oscillator in the line. Since also $\beta=\omega\tau$ we have that the phase velocity of the wave will be the \textit{inverse} of this relaxation time
\begin{equation}
	u_{ND}=\frac{1}{\tau}
	\label{eq:nondissphasevel.tra}
\end{equation}
\subsection{Coaxial Cables}     
A great example of transmission line used every day in laboratories and houses is the \textit{coaxial cable}.\\
A coaxial cable can be imagined as a conductive cable covered by an insulating cladding.\\
Considering the moving currents inside the conductor, we have thanks to Maxwell's equations
\begin{equation*}
	\begin{paligned}
		\nabla\cdot\vec{B}&= 0\\
		\nabla\times\vec{B}&= \mu\vec{J}
	\end{paligned}
\end{equation*}
From the second equation we get that the magnetic induction field around the conductive section of the cable is
\begin{equation*}
	\oint_{\del C}\vec{B}\cdot\ver{t}\dd l=2\pi rB=\mu I\implies B(r)=\frac{\mu}{2\pi r}I
\end{equation*}
And thus, thanks to the first equation of the two
\begin{equation*}
	\oiint_{C}\vec{B}\cdot\vec{n}\dd^2s=\frac{\mu lI}{2\pi}\log\left( \frac{b}{a} \right)
\end{equation*}
Where $a, b$ are the internal and external radii of the insulating cladding and $l$ the length of the cable.\\
Note that also, by definition
\begin{equation}
	L_u=\frac{L}{l}=\frac{1}{l}\frac{\Phi}{LI}=\frac{\mu}{2\pi}\log\left( \frac{b}{a} \right)
\end{equation}
For the capacitance of the coaxial cable it's even easier. Using the formula for the capacitance of a cylinder we have
\begin{equation}
	C_u=\frac{C}{l}=\frac{2\pi\epsilon}{\log\left( \frac{b}{a} \right)}
	\label{eq:unitcapac.coax}
\end{equation}
Then, $\gamma$ is easily calculable as
\begin{equation}
	\gamma=\sqrt{L_uC_u}=\sqrt{\mu\epsilon}=\frac{n}{c}
	\label{eq:gamma.coax}
\end{equation}
%\section{Three-phase Generators}
\end{document}
