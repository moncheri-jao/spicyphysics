\documentclass[../electromagnetism.tex]{subfiles}
\begin{document}
\section{Variable Current Circuits}
\subsection{RC Circuit in DC}
Let's begin to consider what happens when currents aren't anymore constant in time but instead are variable. The simplest possible circuit that we can imagine with variable currents (but with a continuous source) is the following one
\begin{figure}[H]
	\centering
	\begin{circuitikz}
		\draw (0, 0) node[cute spdt down, rotate=180](sw){};
		\draw (sw.out 1) -- ++(0, -4) node[ground](gr){};
		\draw (sw.in) to[R, l=$R$, v=$V_R$] ++(2, 0) to[C, l=$C$, i=$I_C(t)$, v=$V_C(t)$] ++(0, -4) |- (gr);
		\draw (sw.out 2) -- ++(-4, 0) to[dcvsource, l=$V$, i=$I$] ++(0, -4) |- (gr);
	\end{circuitikz}
	\caption{Diagram of a circuit which charges a capacitor with DC current and then lets it discharge freely to ground}
	\label{fig:chargedisC.ac}
\end{figure}
If we consider two time frames, one when the switch is on the other configuration and charges the capacitor, till a time $t_0$, when the capacitor is fully charged, and then gets switched and the charged capacitor discharges on the ground, we have, during the charge of the capacitor
\begin{equation*}
	V=V_R+V_C=RI+\frac{Q}{C}=R\dv{Q}{t}+\frac{C}{C}
\end{equation*}
Where we used the following identities
\begin{equation*}
	\begin{paligned}
		C&= \frac{Q}{V_C}\\
		I&= \dv{Q}{t}
	\end{paligned}
\end{equation*}
Solving for the current we have
\begin{equation*}
	RC\dv{Q}{t}=CV-Q
\end{equation*}
And solving this ordinary differential equation between $t=0$ and $t=t_0$ we get
\begin{equation*}
	log\left( \frac{CV-Q}{CV-Q(0)} \right)=-\frac{t}{RC}
\end{equation*}
Which after solving for $Q$ becomes, noting that at $t=0$ we have $Q(0)=0$
\begin{equation}
	Q(t)=CV\left( 1-e^{-\frac{t}{RC}} \right)
	\label{eq:chargeC.ac}
\end{equation}
We define the \textit{relaxation time of the circuit} $\tau=RC$, and after substituting the solution into the definition of the capacitance and solving for $V_C$ we have
\begin{equation}
	V_C(t)=V\left( 1-e^{-\frac{t}{\tau}} \right)
	\label{eq:vccdischarge.ac}
\end{equation}
If we solve for $V_R$, noting that $V_R=RI=R\dot{Q}$ we get that the voltage at the poles of the resistor at a time $t$ is
\begin{equation}
	V_R(t)=\frac{RCV}{\tau}e^{-\frac{t}{\tau}}=Ve^{-\frac{t}{\tau}}
	\label{eq:chargecaprv.ac}
\end{equation}
The circuit is said to be stable when $t\gtrsim4\tau$, so that we can assume the capacitor as fully charged.\\
We can now analyze the case when the capacitor is discharging, i.e. when the switch is flipped and the resistance and capacitor are isolated, and the circuit is practically the following
\begin{figure}[H]
	\centering
	\begin{circuitikz}
		\draw (0, 0) node[ground](gr){};
		\draw (gr) -- ++(4, 0) to[C, l=$C$, i=$I_C(t)$, v=$V_C(t)$] ++(0,4) to[R, l=$R$, v=$V_R(t)$] ++(-4, 0) |- (gr);
	\end{circuitikz}
	\caption{Diagram of the discharge of the capacitor $C$}
	\label{fig:dischargeconf.ac}
\end{figure}
Here, since we're directly connected to ground, we must have $V=0$, thus
\begin{equation*}
	I(t)R+V_C(t)=0
\end{equation*}
Thus, through substitution we have
\begin{equation*}
	R\dv{Q}{t}=-\frac{Q(t)}{C}\implies Q(t)=Q(0)e^{-\frac{t}{RC}}
\end{equation*}
But $Q(0)=CV$m and thus
\begin{equation}
	\begin{paligned}
		Q(t)&= CVe^{-\frac{t}{\tau}}\\
		V_C(t)&=\frac{Q(t)}{C}Ve^{-\frac{t}{\tau}}\\
	\end{paligned}
	\label{eq:chargecondfinal.ac}
\end{equation}
\subsection{RC Circuit in AC}
Consider now the same circuit using an alternate voltage source $V(t)$. Consider the special case where this source is an harmonic source, as
\begin{equation*}
	V(t)=V_0\cos\left( \omega t \right)
\end{equation*}
The circuit diagram is the following
\begin{figure}[H]
	\centering
	\begin{circuitikz}
		\draw (0, 0) to[sV=$V$, v=$V(t)$] ++(0, 3) to[R, l=$R$] ++(4, 0) to[C, l=$C$, v=$V_C(t)$] ++(0, -3) -- (0, 0); 
	\end{circuitikz}
	\caption{Alternate current version of the resistor-capacitor circuit treated in the previous section}
	\label{fig:accapacitordis.ac}
\end{figure}
As we have already saw before, we have
\begin{equation*}
	\begin{paligned}
		V_C&= \frac{Q}{C}\\
		V_R&= R\dv{Q}{t}\\
	\end{paligned}
\end{equation*}
Thus, the \textit{characteristic differential equation of the circuit} is 
\begin{equation*}
	V(t)-R\dv{Q}{t}-\frac{Q}{C}=0
\end{equation*}
Or, noting that
\begin{equation*}
	\frac{Q}{C}=V_C\implies Q=CV_C\implies R\dv{Q}{t}=RC\dv{V_C}{t}
\end{equation*}
We have
\begin{equation}
	V(t)-RC\dv{V_C(t)}{t}-V_C(t)=0
	\label{eq:rcchargechareq.ac}
\end{equation}
Substituting the functional expression of $V(t)$, we get a not-so-difficult first order differential equation
\begin{equation}
	RC\dv{V_C}{t}+V_C(t)=V_0\cos(\omega t)
	\label{eq:eqtosolvercdis.ac}
\end{equation}
We solve it by searching a solution via the similarity method, i.e. supposing that $V_{C}(T)\propto\cos(\omega t)$. Said $V_{0C}$ the proportionality constant we get, adding a generic phase
\begin{equation*}
	\dv{V_C}{t}=-\omega V_{0C}\sin\left( \omega t+\varphi \right)
\end{equation*}
This is better expressed using phasor notation, which is built using Euler's identity. Keeping in mind that we actually measure the real part of this complex function, i.e. $V_C^{(M)}(t)=\real\left\{ V_{C}(t) \right\}$, we get by deriving and plugging back into the differential equation
\begin{equation*}
	V_C(t)+RC\dv{V_C}{t}=V_{0C}e^{i\omega t+i\varphi}+i\omega RCV_{0C}e^{i\omega t+i\varphi}=V_0e^{i\omega t}
\end{equation*}
Solving, we have
\begin{equation*}
	\left( 1+i\omega RC \right)V_{0C}e^{i\varphi}=V_0\implies V_{0C}e^{i\varphi}=\frac{1-i\omega RC}{1+\left( RC\omega \right)^2}V_0
\end{equation*}
And therefore, substituting $\tau=RC$
\begin{equation}
	V_{0C}(t)e^{i\varphi}=\left( \frac{1}{1+\left( \tau\omega \right)^2}-\frac{i\omega \tau}{1+\left( \tau\omega \right)^2} \right)V_0
	\label{eq:solutionrcac.ac}
\end{equation}
Or, in terms of measured voltages
\begin{equation}
	V_{C}^{(M)}=\frac{V_0}{1+\omega^2\tau^2}\cos\left( \omega r+\arctan\left( -\omega\tau \right) \right)
	\label{eq:measuredvoltrc.ac}
\end{equation}
Note that this solution is strongly dependent on the frequency $\omega$ of the input voltage, in fact, we specifically have, ignoring the phase $\varphi=\arctan\left( -\omega\tau \right)$
\begin{equation}
	\begin{paligned}
		\lim_{\omega\to0}V_{0C}(\omega)&= V_0\\
		\lim_{\omega\to\infty}V_{0C}(\omega)&= 0\\
	\end{paligned}
	\label{eq:filterbehaviorrc.ac}
\end{equation}
This behavior clearly shows that the voltage is nonzero only when $\omega<\omega_R$, with $\omega_R$ being the resonance frequency. This type of circuit is widely known as a \textit{low pass filter}, i.e. a circuit which permits the passage of voltage only to a limit \textit{resonant} frequency <++>
%%TODO corrente in RC AC come inizio GOL
\subsection{Generalized Ohm Law}
\subsection{Transfer Functions}
\section{RC, RL Circuits}
\subsection{Low Pass Filters}
\subsection{High Pass Filters}
\subsection{Integrator and Derivator Circuits}
\section{RLC Circuits}
\subsection{Overdamping}
\subsection{Critical Damping}
\subsection{Oscillatory Damping}
\section{Power in AC Circuits}
\section{Three-phase Generators}
\section{Diodes}
\subsection{Semiconductor Diodes}
\subsection{Zener Diodes}
\section{Transmission Lines}
\subsection{Electromagnetic Kirchhoff Laws}
\subsection{Telegrapher's Equations}
\subsection{Non-dissipative Solution}
\section{Coaxial Cables}
\end{document}
