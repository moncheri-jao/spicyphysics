\documentclass[../electromagnetism.tex]{subfiles}
\begin{document}
\section{Interference}
Suppose that we have a single point-like source $S$ which emits an electromagnetic wave $\vec{E}$, which passes through two point apertures $S_1, S_2$, and then converges again to a point $P$ on which we put some detector.\\
The starting wave is described as the sum of two single waves coming out the pinholes via the superposition principle, i.e.
\begin{equation*}
	\vec{E}=\vec{E}_{(1)}+\vec{E}_{(2)}
\end{equation*}
Where, in general, the two wavelets can be described generally as usual
\begin{equation}
	\begin{aligned}
		\vec{E}_{(1)}&= \vec{E}_1e^{i\vec{k}_1\cdot\vec{r}-i\omega t+i\phi_1}\\
		\vec{E}_{(2)}&= \vec{E}_2e^{i\vec{k}_2\cdot\vec{r}-i\omega t+i\phi_2}
	\end{aligned}
	\label{eq:afterpin.int}
\end{equation}
Due to the constraint of the system we have that $\vec{k}_1=\vec{k}_2$, therefore the sum of the two gives the general field, which in general depends only on the phase difference of the two wavelets.\\
These two wavelets are said to be \textit{mutually coherent}, if the phase difference between the two is constant, i.e.
\begin{equation}
	\Delta\phi=\phi_1-\phi_2=\text{ const}
	\label{eq:coherencedef.int}
\end{equation}
This definition comes up directly when we evaluate the irradiance of the measured field $\vec{E}$. In fact, we have
\begin{equation*}
	I=\sqrt{\frac{\mu}{\epsilon}}E^2=\sqrt{\frac{\epsilon}{\mu}}\left[ E_1^2+E_2^2+2\real\left\{ \vec{E}_{(1)}\cdot\vec{E}_{(2)} \right\} \right]
\end{equation*}
Remembering that $\sqrt{\epsilon/\mu}=nZ_0^{-1}$, and rewriting $nE_i^2/Z_0=I_i$ we have that the total intensity measured is (in the general case with two different waves)
\begin{equation}
	I=I_1+I_2+2\sqrt{\frac{\mu}{\epsilon}I_1I_2}\cos\left( \vec{k}_1\cdot\vec{r}+\vec{k}_2\cdot\vec{r}+\phi_1-\phi_2 \right)
	\label{eq:<+label+>}
\end{equation}<++>
\subsection{Young Experiment}
\subsection{Michelson Interferometer}
\section{Partial Coherence}
\subsection{Correlation}
\subsubsection{Visibility}
\subsection{Coherence Time}
\subsection{Coherence Length}
\section{Coherence and Fourier Calculus}
\subsection{Line Width}
\subsection{Power Spectra}
\subsubsection{Wiener-Khinchin Theorem}
\section{Multiple Beam Interference}
\subsection{Finesse and Airy Functions}
\subsection{Fabry-Perot Instruments}
\subsubsection{Resolution Power of Fabry-Perot Instruments}
\subsubsection{Free Spectral Range}
\end{document}
