\documentclass[../electromagnetism.tex]{subfiles}
\begin{document}
\section{Interference}
Suppose that we have a single point-like source $S$ which emits an electromagnetic wave $\vec{E}$, which passes through two point apertures $S_1, S_2$, and then converges again to a point $P$ on which we put some detector.\\
The starting wave is described as the sum of two single waves coming out the pinholes via the superposition principle, i.e.
\begin{equation*}
	\vec{E}=\vec{E}_{(1)}+\vec{E}_{(2)}
\end{equation*}
Where, in general, the two wavelets can be described generally as usual
\begin{equation}
	\begin{aligned}
		\vec{E}_{(1)}&= \vec{E}_1e^{i\vec{k}_1\cdot\vec{r}-i\omega t+i\phi_1}\\
		\vec{E}_{(2)}&= \vec{E}_2e^{i\vec{k}_2\cdot\vec{r}-i\omega t+i\phi_2}
	\end{aligned}
	\label{eq:afterpin.int}
\end{equation}
Due to the constraint of the system we have that $\vec{k}_1=\vec{k}_2$, therefore the sum of the two gives the general field, which in general depends only on the phase difference of the two wavelets.\\
These two wavelets are said to be \textit{mutually coherent}, if the phase difference between the two is constant, i.e.
\begin{equation}
	\Delta\phi=\phi_1-\phi_2=\text{ const}
	\label{eq:coherencedef.int}
\end{equation}
This definition comes up directly when we evaluate the irradiance of the measured field $\vec{E}$. In fact, we have
\begin{equation*}
	I=\sqrt{\frac{\mu}{\epsilon}}E^2=\sqrt{\frac{\epsilon}{\mu}}\left[ E_1^2+E_2^2+2\real\left\{ \vec{E}_{(1)}\cdot\vec{E}_{(2)} \right\} \right]
\end{equation*}
Remembering that $\sqrt{\epsilon/\mu}=nZ_0^{-1}$, and rewriting $nE_i^2/Z_0=I_i$ we have that the total intensity measured is (in the general case with two different waves)
\begin{equation}
	I=I_1+I_2+2\sqrt{\frac{\mu}{\epsilon}I_1I_2}\cos\left( \vec{k}_1\cdot\vec{r}-\vec{k}_2\cdot\vec{r}+\phi_1-\phi_2 \right)
	\label{eq:intwavesI.int}
\end{equation}
The last term, which depends on the root of the product of the intensities of the single waves, is called the \textit{interference term}. This term is the only of the three that depends on the ``angle'' $\theta$, defined as
\begin{equation}
	\Delta(\vec{r})=(\vec{k}_1-\vec{k}_2)\cdot\vec{r}+\phi_1-\phi_2
	\label{eq:Deltar.int}
\end{equation}
This last interference term is clearly dependent on the polarization of the two waves. In fact, it can be exactly zero when the scalar product of the two fields of the single waves is zero. I.e. when their polarization is orthogonal.\\
In laboratories, for obvious reasons we never measure the instantaneous intensity, but an average. Also, the interfering waves must not have the same frequencies and can come from different sources, therefore the total intensity we measure will instead be described by the following equation
\begin{equation}
	I=\expval{I}=\lim_{T\to\infty}\frac{1}{T}\sqrt{\frac{\epsilon}{\mu}}\int_{0}^{T}\left( \vec{E}_{(1)}+\vec{E}_{(2)} \right)\cdot\adj{(\vec{E}_{(1)}+\vec{E}_{(2)})}\dd^{}{t}
	\label{eq:iexpval.int}
\end{equation}
The interference term $I_{int}$ is therefore defined as follows
\begin{equation}
	I_{int}=2\vec{E}_1\cdot\vec{E}_2\sqrt{\frac{\epsilon}{\mu}}\lim_{T\to\infty}\frac{1}{T}\int_{0}^{T}\cos\left( \left( \vec{k}_1-\vec{k}_2 \right)\cdot\vec{r}-(\omega_1-\omega_2)t +\phi_1-\phi_2\right)\dd^{}{t}
	\label{eq:inttermreal.int}
\end{equation}
The integral at the end is zero only in a handful of cases, in fact
\begin{equation}
	I_{int}=0\implies\begin{cases}
		\vec{E}_1\perp\vec{E}_2\\
		\omega_1\ne\omega_2\\
		\phi_1-\phi_2\ne\text{ const}
	\end{cases}
	\label{eq:casesnoint.int}
\end{equation}
\subsection{Double Slit Interferometer}
The first experiment with interference was prepared by Thomas Young in the early 1800s. This experimental setup, better known as the \textit{double slit experiment} is set up, ideally, as a point source which emits a single wave; this wave then comes through two slits and then a detecting screen shows the interference pattern.\\
In general we can say that the two sections of the experiment (one where lays the source, and one where lays the screen) have different refraction indexes $n_1\, n_2$, for better emphasizing this we write $\vec{k}=n\vec{k}_0$.
Since the electromagnetic wave comes from a single source, we have to impose that the two wavelets coming out from the slits into the screen have the same color, i.e. $\omega_1=\omega_2=\omega$, therefore, the interference term will be
\begin{equation*}
	I_{int}=2\vec{E}_1\cdot\vec{E}_2\sqrt{\frac{\epsilon}{\mu}}\lim_{T\to\infty}\frac{1}{T}\int_{0}^{T}\cos\left( \vec{k}_0\left( n_1\vec{r}_1-n_2\vec{r}_2 \right)+\Delta\phi \right)\dd^{}{t}
\end{equation*}
Due to the different position of the two slits, and since their distance from the measuring point on the screen is not necessarily equal, we have that the two single wavelets will be described by two different $\vec{r}$ vectors.\\
Now, since the two incoming wavelets are not necessarily parallel, we can write the following
\begin{equation*}
	\vec{E}_1\cdot\vec{E}_2=\sqrt{I_1I_2}\cos\delta
\end{equation*}
And therefore, in general, the intensity will be
\begin{equation}
	I=I_1+I_2+2\sqrt{I_1I_2}\cos\delta\cos\Delta
	\label{eq:youngintesity.yint}
\end{equation}
Where we omitted the time average (as we will do going forward from now, for notational ease).\\
Now, due to the origin of the two waves, we know that $I_1, I_2$ don't depend on time and are constants, but it's clear that depending on the position of the measuring point on the screen that $I_{min}\le I\le I_{max}$, all determined by the last cosine factor of the interference term.\\
Since $-1\le\cos\Delta\le1$ we will have
\begin{equation}
	I=\begin{dcases}
		I_{max}&\Delta=2m\pi\\
		I_{min}&\Delta=(2m+1)\pi
	\end{dcases}\qquad m\in\Z
	\label{eq:imaximinyoung.yint}
\end{equation}
This result can be reproduced also thinking in strictly trigonometric terms. Said $L$ the distance from the slits to the screens and said $d$ the distance from the half-point between the slits and the measuring point on the screen, we have, called $x'$ the measuring point, that, firstly
\begin{equation*}
	I=I_0+2I_0\cos\Delta=I\left( 1+2\cos\Delta \right)=4I\cos^2\left( \frac{\Delta}{2} \right)
\end{equation*}
And then that
\begin{equation}
	\frac{\Delta}{2}\approxeq\frac{\pi d}{\lambda_0}\frac{x'}{L}
	\label{eq:delta12approx.yint}
\end{equation}
I.e.
\begin{equation*}
	I_t=4I_0\cos^2\left( \frac{\pi d}{\lambda_0}\frac{x'}{L} \right)
\end{equation*}
Then, the maxima and minima of the intensity measured on the screen can be described all in terms of wavelength and distance from the screen, i.e.
\begin{equation}
	I=\begin{dcases}
		I_{max}&\Delta=\frac{m\lambda_0}{d}\\
		I_{min}&\Delta=\frac{\lambda_0}{2d}(2m+1)
	\end{dcases}\qquad m\in\Z
	\label{eq:itot.yint}
\end{equation}
\subsection{Michelson-Morley Interferometer}
Another experiment demonstrating the interference of electromagnetic waves was made in the later years of the 1800s by Michelson and Morley. This experiment was of critical importance also in other branches, like mechanics, in fact it proved that without doubt there was no ether in space, but rather gave the foundation to the special relativistic Lorentz transformations.\\
This experiment is composed by a point-like source, which passes through a \textit{beam splitter}, i.e. an optically active object which divides an incoming beam into two separate beams.\\
This beam splitter is oriented in a way such that the two outgoing beams are orthogonal between each other, and after a \textit{different} distance for each ($d_1, d_2$) they meet again at the beam splitter, which aligns them back up again and sends them to a receiver.\\
Said $d_i$ the optical path traveled by the two waves we have that
\begin{equation*}
	2d=2(d_1-d_2)
\end{equation*}
Then, the total intensity at the receiver will be
\begin{equation}
	I=4I_0\cos^2\left( \frac{\pi}{\lambda}2d \right)
	\label{eq:michelsoninttot.mint}
\end{equation}
Then, as we did before with the double slit interferometer, we will have by finding the maximum and the minimum of the cosine
\begin{equation}
	I=\begin{dcases}
		I_{max}&2d=m\lambda\\
		I_{min}&2d=\left( m+\frac{1}{2} \right)\lambda
	\end{dcases}\qquad m\in\Z
	\label{eq:michelsonmaxint.mint}
\end{equation}
This kind of interferometer is still widely in use in the world of modern physics, in fact it's the same kind of interferometer used in LIGO and VIRGO experiments (albeit much more modern). These interferometers use arms long in the order of kilometers for detecting the slightest change in the interference pattern given by the oscillation of the optical path length at the passage of a gravitational wave. Here the point source chosen is a high-power laser.\\
Another usage of the Michelson-Morley interferometer will see the light in the future with the launch of the LISA constellation of satellites, which will use huge distances for detecting the faintest gravitational waves.
\section{Partial Coherence}
\subsection{Correlation}
It's time to consider the most generic case possible, and the nature of interference itself.\\
Consider two beams coming from a point-like source $S$, with the same frequency and polarization. Chosen a point R on which the two beams rejoin after two different paths $l_1, l_2$, we know for a fact (since electromagnetic waves travel at a constant speed $u=c/n$), that one of the two beams will arrive at a later time at the point.\\
Said $l_1<l_2$, $t$ being the time needed to cross the path long $l_1$ and $t+\tau$ being the time needed to cross the path $l_2$, we have
\begin{equation}
	\begin{aligned}
		\vec{E}_{(1)}(t)&= \vec{E}_1e^{i\vec{k}\cdot\vec{r}-i\omega t+i\phi_1(t)}\\
		\vec{E}_{(2)}(t+\tau)&= \vec{E}_2e^{i\vec{k}\cdot\vec{r}-i\omega t+i\phi_2(t+\tau)}
	\end{aligned}
	\label{eq:pcorre1e2.pcorr}
\end{equation}
Where we consider the randomness of phase by noting that it's time dependent.\\
By definition of what we said so far, then 
\begin{equation}
	\tau=\frac{1}{u}\abs{l_2-l_1}=\frac{\Delta l}{u}
	\label{eq:taudef.pcorr}
\end{equation}
And, therefore the intensity at the point R will then be defined by the usual equation
\begin{equation*}
	I=I_1+I_2+2Z\expval{\real\left\{ \vec{E}_1(t)\cdot\adj{\vec{E}}_2(t+\tau) \right\}}
\end{equation*}
\begin{dfn}[Correlation Function]
	From the last equation it's possible to define a new integral function, which we will call the \textit{correlation function} or the \textit{mutual correlation}. This function is defined as a function of $\tau$, as follows
	\begin{equation}
		\Gamma_{12}(\tau)=\lim_{T\to\infty}\frac{1}{T}\sqrt{\frac{\epsilon}{\mu}}\int_{0}^{T}\vec{E}_1(t)\adj{\vec{E}}_2(t+\tau)\dd^{}{t}
		\label{eq:correlationfunc.pcorr}
	\end{equation}
	From the definition, it's clear that
	\begin{equation*}
		\Gamma_{12}(\tau)=\sqrt{\frac{\epsilon}{\mu}}\expval{\vec{E}_{1}(t)\adj{\vec{E}}_{2}(t+\tau)}
	\end{equation*}
	We also define the \textit{autocorrelation function} as
	\begin{equation}
		\Gamma(\tau)=\Gamma_{ii}(\tau)=\sqrt{\frac{\epsilon}{\mu}}\expval{\vec{E}_i(t)\adj{\vec{E}}_i(t+\tau)}
		\label{eq:autocorrelation.pcorr}
	\end{equation}
	Also, by substitution, we can say that
	\begin{equation*}
		\Gamma_{12}(\tau)=\frac{1}{2}I_{int}
	\end{equation*}
	And also that
	\begin{equation*}
		\begin{aligned}
			\Gamma_{11}(0)&= I_1\\
			\Gamma_{22}(0)&= I_2
		\end{aligned}
	\end{equation*}
	Another useful function derived by the correlation function is the normalized version, known as the \textit{degree of correlation} $\gamma_{ij}(\tau)$.\\
	It's defined as follows
	\begin{equation}
		\gamma_{ij}(\tau)=\frac{\Gamma_{ij}(\tau)}{\sqrt{\Gamma_{ii}(0)\Gamma_{jj}(0)}}
		\label{eq:degreeofcorr.pcorr}
	\end{equation}
\end{dfn}
From the previous definitions it's possible then to see how the interaction term depends on the correlation between the two waves, and specifically, on the phase difference between the two. Being $\gamma$ a complex function, we can write
\begin{equation*}
	\real\left\{ \gamma_{ij}(\tau) \right\}=\abs{\gamma}_{ij}(\tau)\cos\left( \Delta\phi_{ij} \right)
\end{equation*}
Then, the interaction term can be defined in the most general way as follows
\begin{equation}
	I_{int}=2\sqrt{I_1I_2}\abs{\gamma_{12}(\tau)}\cos\left( \Delta\phi_{12} \right)	
	\label{eq:intterm-autocorr.pcorr}
\end{equation}
The definition of the autocorrelation as a normalized term of measure of the correlation of two waves, gives rise to the definitions of \textit{perfect incoherence}, \textit{partial coherence} and \textit{perfect coherence}, respectively when $\abs{\gamma_{12}}=1$, $\abs{\gamma_{12}}=0$ and $0<\abs{\gamma_{12}}<1$.\\
Therefore, the range of interference is exactly in the set
\begin{equation}
	I_{int}\in\left[ -2\sqrt{I_1I_2}\abs{\gamma_{12}},\sqrt{I_1I_2}\abs{\gamma_{12}} \right]
	\label{eq:interfrange.pcorr}
\end{equation}
\begin{dfn}[Visibility]
	Another useful definition is the \textit{visibility of fringes} $\mathcal{V}$, a constant comprised between $0,1$, defined as the ratio of the difference of intensity between the peaks and the shadows of the interference pattern and the sum of the two intensities
	\begin{equation}
		V=\frac{I_{max}-I_{min}}{I_{max}+I_{min}}=\frac{2\sqrt{I_1I_2}\abs{\gamma_{12}(\tau)}}{I_1+I_2}
		\label{eq:visibility.pcorr}
	\end{equation}
	Note also that if $I_1=I_2=I_0$, clearly
	\begin{equation*}
		\mathcal{V}=\abs{\gamma_{12}(\tau)}
	\end{equation*}
	Which implies that the maximum visibility will be obtained when the two waves are in the regime of \textit{total coherence}.
\end{dfn}
\subsection{Coherence Time and Coherence Length}
From what we have seen before, the degree of correlation is strictly tied to the signal and its properties, but especially to its phase.\\
Consider a quasimonochromatic wave ($\Delta\omega\approx0$). The field is then described as follows
\begin{equation}
	\vec{E}(\vec{r},t)=\vec{E}_0(\vec{r})e^{-i\omega t+i\phi(t)}
	\label{eq:quasimonochromaticwave.ctm}
\end{equation}
The phase function $\phi(t)$ is as we said before a random function of time. Physically, we can see this function as a composition of multiple Heaviside step functions, or, more clearly, it describes intervals of coherence ($\phi(t)=\text{ const}$) and instants of decoherence.\\
Said $\tau_0$ the \textit{coherence time}, i.e. the time needed for $\phi(t)$ to change from one constant value to the other, we can begin to analyze the behavior of $\gamma(\tau)$ in different occasions.\\
Since
\begin{equation*}
	\begin{aligned}
		\vec{E}(\vec{r},t)&= \vec{E}_0(\vec{r})e^{-i\omega t}e^{i\phi(t)}\\
		\vec{E}(\vec{r},t+\tau)&= \vec{E}_0(\vec{r})e^{-i\omega(t+\tau)}e^{i\phi(t+\tau)}
	\end{aligned}
\end{equation*}
The scalar product of $\vec{E}(t)$ and $\vec{E}(t+\tau)$ is then (omitting the spatial dependence, since it does not interfere with our calculations)
\begin{equation*}
	\vec{E}(t)\cdot\adj{\vec{E}}(t+\tau)=E_0^2e^{-i\omega\tau}e^{i\left( \phi(t)-\phi(t+\tau) \right)}
\end{equation*}
Therefore, the degree of correlation depends only on the difference of the two phases
\begin{equation*}
	\gamma(\tau)=\frac{1}{\expval{E^2}}\expval{\vec{E}(t)\cdot\adj{\vec{E}}(t+\tau)}=e^{-i\omega\tau}\expval{e^{i(\phi(t))-\phi(\tau)}}
\end{equation*}
Since we defined $\phi(t)$ as a periodic (random) step function with a period of $\tau_0$, the expected value is
\begin{equation}
	\expval{\phi(t)-\phi(t+\tau)}=\begin{dcases}
		0&\tau>\tau_0\ \vee\ 0<t<\tau_0-\tau\\
		\Delta\phi&\tau_0-\tau<t<\tau_0
	\end{dcases}
	\label{eq:expvalphase.ctm}
\end{equation}
Or, in common words, it's zero if we're evaluating the coherence when $\phi(t+\tau)$ has changed already to another random value, or vice-versa when $\phi(t)$ has not yet reached the coherence time $\tau_0$. It's equal to a constant value $\Delta\phi$ only and only when we're checking in an interval which is not greater than the coherence time.\\
Considering only the first interval of coherence, then 
\begin{equation*}
	\gamma(\tau)=\frac{e^{-i\omega\tau}}{\tau_0}\left[ \int_{0}^{\tau_0-\tau}\dd^{}{t}+e^{i\Delta\phi}\int_{0}^{\tau_0}\dd^{}{t} \right]=\left( \frac{\tau_0-\tau}{\tau_0}+\frac{\tau}{\tau_0}e^{i\Delta\phi} \right)e^{-i\omega\tau}
\end{equation*}
Or, evaluating the integrals we have, in general, for a single cycle (or, in common terms, in a single \textit{wave train}) the coherence is strictly tied to the coherence time $\tau_0$, with the equation
\begin{equation}
	\gamma(\tau)=\begin{dcases}
		\frac{\tau_0-\tau}{\tau_0}e^{-i\omega\tau}&\tau<\tau_0\\
		0&\tau>\tau_0
	\end{dcases}
	\label{eq:correlationfuncwave.ctm}
\end{equation}
Note that, if we take the evaluation of the integral and check it's expectation value for $T\to\infty$, the expected value is zero, due to the random variation of phase $\Delta\phi$. Therefore, a wave will tend to decoherence for big periods.\\
Note also that, since we're not considering two different waves, if there are no attenuation phenomena, the absolute value of the degree of correlation is the visibility of fringes, i.e. 
\begin{equation}
	\mathcal{V}=1-\frac{\tau}{\tau_0}
	\label{eq:vis.ctm}
\end{equation}
Considered everything and evaluated the real part of what we found before we have
\begin{equation}
	I=I_1+I_2+2\sqrt{I_1I_2}\real\left\{ \frac{\tau_0-\tau}{\tau_0}e^{-i\omega\tau} \right\}\quad\tau<\tau_0
	\label{eq:intundertcoherence.ctm}
\end{equation}
Which, if we develop the last operation on the right becomes
\begin{equation}
	I=I_1+I_2+2\sqrt{I_1I_2}\left( 1-\frac{\tau}{\tau_0} \right)\cos(\omega\tau)\quad\tau<\tau_0
	\label{eq:intcoherencecomplete.ctm}
\end{equation}
Noting that on the interference term we have the real part of the degree of self-correlation, we can say with ease that the interference pattern will be present only when $\tau$ is less than the coherence time $\tau_0$, therefore indicating that peak-shadow patterns only appear with coherent light.\\
The parameters \textit{coherence length} and \textit{coherence time} are two intrinsic parameters of the wave, which indicate single coherent packets of light ,or \textit{wave trains}, where the first can be understood experimentally as the length of the wave train.\\
Note that also, if $\Delta d>l_0=c\tau_0$, $\Delta\tau>\tau_0$, i.e. if the wave is non-coherent, then $I_{int}=0$, and the visibility of patterns is null ($\mathcal{V}=0$).
\section{Coherence and Fourier Calculus}
\subsection{Line Width and Power Spectrum}
When dealing with electromagnetic waves it's important to note that in nature monochromatic sources \textit{do not} exist in general.\\
Every single source that emits electromagnetic waves at some frequency $\omega_0$, has \textit{always} some spread around the emission frequency called the \textit{line width} $\Delta\omega$, given from dispersion.\\
This line width is strictly tied to the coherence time $\tau_0$ of the source.\\
Consider now a generic wave train $f(t)$ with coherence time $\tau_0$. Its time dependence is generally described by a complex exponential in the following way
\begin{equation}
	f(t)=\begin{dcases}
		e^{-i\omega_0t}&-\frac{\tau_0}{2}<t<\frac{\tau_0}{2}\\
		0&\abs{t}\ge\frac{\tau_0}{2}
	\end{dcases}
	\label{eq:trainwave.cfc}
\end{equation}
The study of this single wave train is way simpler in the $\omega$ space. Applying the Fourier transformation to our wave train we get
\begin{equation*}
	\hat{f}(\omega)=\fopr{f}(\omega)=\frac{1}{i(\omega-\omega_0)\sqrt{2\pi}}\left[ e^{i(\omega-\omega_0)\frac{\tau_0}{2}}-e^{-i(\omega-\omega_0)\frac{\tau_0}{2}} \right]
\end{equation*}
Noting the sine on the right, we have finally
\begin{equation}
	\hat{f}(\omega)=\sqrt{\frac{2}{\pi}}\frac{\sin\left[ (\omega-\omega_0)\frac{\tau_0}{2} \right]}{\omega-\omega_0}\qquad\abs{t}<\frac{\tau_0}{2}
	\label{eq:ftrainwave.cfc}
\end{equation}
From this, we define the \textit{power spectrum} $\hat{F}(\omega)$ of the wave in $\omega$ space as
\begin{equation*}
	\hat{F}(\omega)=\abs{\hat{f}(\omega)}^2
\end{equation*}
Which, in for this train wave is
\begin{equation}
	\hat{F}(\omega)=\frac{2}{\pi}\frac{\sin^2\left[ (\omega-\omega_0)\frac{\tau_0}{2} \right]}{\omega-\omega_0}
	\label{eq:trainwaveps.cfc}
\end{equation}
Now, in order to find the line width, we search for the zeroes of the power spectrum, which in this case are
\begin{equation*}
	\sin\left[ (\omega_k-\omega_0)\frac{\tau_0}{2} \right]=0\implies\omega_k=\frac{2k\pi}{\tau_0}+\omega_0
\end{equation*}
The distance between two consecutive orders of shadows ($\omega_k, \omega_{k+1}$), we have
\begin{equation}
	\Delta\omega=\frac{2\pi}{\tau_0}\implies\Delta\nu=\frac{1}{\tau_0}
	\label{eq:linewidth.cfc}
\end{equation}
I.e., the line width is \textit{exactly} the inverse of the coherence time. Note that a perfectly coherent (ideal) electromagnetic wave must have an infinite coherence time, therefore the line width must be zero. This is obtained only if the power spectrum is a delta distribution around the emission frequency $\omega_0$
\begin{equation*}
	\hat{F}(\omega)=\delta\left( \omega-\omega_0 \right)
\end{equation*}
This result, gives us a way to experimentally measure the average coherence time and length of a wave. By definition we have
\begin{equation}
	\expval{\tau_0}=\frac{1}{\Delta\nu}\qquad\expval{l_0}=c\expval{\tau_0}=\frac{c}{\Delta\nu}
	\label{eq:expectedcoherencetimelenght.cfc}
\end{equation}
Since $\Delta\nu/\nu=\Delta\lambda/\lambda$ we have
\begin{equation*}
	\Delta\nu=\frac{\Delta\lambda}{c\lambda^2}
\end{equation*}
I.e., if we pass from frequencies to wavelengths, we can estimate the coherence length with the following expression
\begin{equation}
	\expval{l_0}=\frac{\lambda^2}{\Delta\lambda}
	\label{eq:coherencelength.cfc}
\end{equation}
\subsubsection{Power Spectra and Interference}
Given some wave train, how can we find the interference pattern from the power spectrum? This comes easily as a result of the following theorem
\begin{thm}[Wiener-Khinchin]
	Given an electromagnetic wave with power spectrum $G(\omega)=\abs{g(\omega)}^2$, the autocorrelation function is given by
	\begin{equation}
		\Gamma(\tau)=\mathcal{\hat{F}}^{-1}\left[ G(\omega) \right](\tau)
		\label{eq:wienerkhnstatement.cfc}
	\end{equation}
	I.e., the autocorrelation function of a wave is the inverse Fourier transform of the power spectrum
\end{thm}
\begin{proof}
	Said $E(t)$ the inverse Fourier transform of $g(\omega)$, we have
	\begin{equation*}
		\Gamma(\tau)=\int_{\R}^{}E(t)\cc{E}(t+\tau)\dd^{}{t}=\frac{1}{2\pi}\int_{\R}^{}\left[ \int_{\R}^{}g(\omega)e^{-i\omega t}\dd^{}{\omega} \right]\cc{\left[ \int_{\R}^{}g(\omega')e^{-i\omega'(t+\tau)}\dd^{}{\omega'} \right]}\dd^{}{t}
	\end{equation*}
	Due to the independence of the variables, using Fubini's theorem on the exchange of integrals we can rewrite everything as follows
	\begin{equation*}
		\Gamma(\tau)=\frac{1}{2\pi}\iiint_{\R^3}g(\omega)\cc{g(\omega')}e^{-i(\omega-\omega')t}e^{i\omega'\tau}\dd\omega\dd\omega'\dd t
	\end{equation*}
	Integrating the first exponential with respect to $t$ it transforms exactly to a delta distribution, and therefore
	\begin{equation*}
		\Gamma(\tau)=\frac{1}{2\pi}\iint_{\R^2}g(\omega)\cc{g(\omega')}\delta(\omega-\omega')e^{i\omega'\tau}\dd\omega\dd\omega'
	\end{equation*}
	Applying the delta distribution on the integral in $\omega'$ we have then
	\begin{equation*}
		\Gamma(\tau)=\frac{1}{2\pi}\int_{\R}^{}g(\omega)\cc{g(\omega)}e^{i\omega\tau}\dd^{}{\omega}
	\end{equation*}
	Using $z\cc{z}=\abs{z}^2$ we obtain the power spectrum, and the theorem is proven
	\begin{equation*}
		\Gamma(\tau)=\frac{1}{2\pi}\int_{\R}^{}G(\omega)e^{i\omega\tau}\dd^{}{\omega}=\afopr{G(\omega)}(\tau)
	\end{equation*}
\end{proof}
\section{Multiple Beam Interference}
\subsection{Laser Cavities}
So far we treated only the interference of a single wave with itself and of two different waves (also the same wave split in two) interfering between each other.\\
Now we will treat the case of multiple beams (or a single beam interfering with itself) interfering with each other. Experimentally it can be shown (with instruments like an etalon or a Fabry-Perot interferometer) that multiple beams interfering with each other \textit{will} show an interference pattern on a given screen.\\
Consider a material thick $d$ with a refraction index $n_2$ and a single wave with amplitude $E_0$ incoming from outside the material at some angle $\theta$, where the outside has refraction index $n_1$. The result, as we know already, will be a reflection and a refraction.\\
Said $t_1, r_1$ the Fresnel coefficient for $n_1\to n_2$ and $t_2, r_2$ the Fresnel coefficients $n_2\to n_2$ we have that, at the first reflection we will get a reflected beam $E_{r,1}$ with amplitude $r_1E_0$ and transmitted wave with amplitude $E_1=t_1E_0$. The transmitted wave will reach again the boundary of the material (on the other side) and the process will be repeated with a wave with amplitude $E_1$.\\
At the end we will have a series of transmissions and reflections, with $n-$th term 
\begin{equation}
	\begin{aligned}
		E_{T, n}&= t_1t_2r^{2n}_2E_0\\
		E_{R, n}&= t_1t_2r^{2n+1}_2E_0
	\end{aligned}
	\label{eq:nthterm.mbi}
\end{equation}
On each reflection, there will be a phase shift equal to the wavenumber times the optical path traveled. For one process (\textit{two} reflections) we have that, if the reflection angle \textit{inside} the medium is $\theta$, the phase shift will be
\begin{equation}
	\delta=2kd\cos\theta=\frac{4n_2\pi}{\lambda_0}d\cos\theta
	\label{eq:phaseshift.mbi}
\end{equation}
Where $\lambda_0$ is the vacuum wavelength. Considering \textit{every} reflection ($n\to\infty$) together with the phase shift we have that the final amplitude of the series of reflected waves from the material
\begin{equation}
	E_R=r_1E_0+t_1t_2r_2E_0e^{i\delta}\sum_{j=0}^{\infty}r^{2n}_2e^{in\delta}
	\label{eq:reflectionseries.mbi}
\end{equation}
The infinite sum on the right is a convergent geometric sum, which gives the following final result for the amplitude of the last wave of the reflected series
\begin{equation*}
	E_R=E_0\left[ r_1+\frac{t_1t_2r_2e^{i\delta}}{1-r_2^2e^{i\delta}} \right]
\end{equation*}
It's provable that $r_1=-r_2$ and $t_1t_2=1-r_1^2$, then, rearranging the term on the right we have
\begin{equation*}
	E_R=E_0\left[ r_1-\frac{t_1t_2r_1e^{i\delta}}{1-r_1^2e^{i\delta}} \right]=E_0\left[ \frac{r_1\left( 1-r_1^2e^{i\delta} \right)-t_1t_2r_1e^{i\delta}}{1-r_1^2e^{i\delta}} \right]=E_0r_1\left[ \frac{1-(r_1^2+t_1t_2)e^{i\delta}}{1-r_1^2e^{i\delta}} \right]
\end{equation*}
Using $t_1t_2=T_1$ and $r_1^2=R_1$ and that $r_1^2+t_1t_2=1$ we can write that the final amplitude of the reflected series of waves is
\begin{equation}
	E_R=E_0r_1\frac{1-e^{i\delta}}{1-Re^{i\delta}}
	\label{eq:refseriesamp.mbi}
\end{equation}
Or, using $R=\abs{E_R/E_0}^2$, we could also write that the reflection coefficient of the slab is
\begin{equation*}
	R=\abs{\frac{E_R}{E_0}}^2=R_1\frac{\abs{1-e^{i\delta}}^2}{\abs{1-Re^{i\delta}}}^2
\end{equation*}
Which, using $\abs{z}^2=z\cc{z}$ as usual for complex numbers, becomes
\begin{equation*}
	R=R_1\frac{(1-e^{i\delta})(1-e^{-i\delta})}{(1-Re^{i\delta})(1-Re^{-i\delta})}=2R\frac{1-\cos\delta}{1+R^2-2R\cos\delta}
\end{equation*}
Further simplifying, using
\begin{equation*}
	1-\cos\delta=2\sin^2\left( \frac{\delta}{2} \right)
\end{equation*}
We have
\begin{equation*}
	R=\frac{4R\sin\left( \frac{\delta}{2} \right)}{1+R^2-2R-4R\sin^2\left( \frac{\delta}{2} \right)}=\frac{4R}{(1-R)^2}\frac{\sin^2\left( \frac{\delta}{2} \right)}{1+\frac{4R}{(1-R)^2}\sin^2\left( \frac{\delta}{2} \right)}
\end{equation*}
Which, defined the \textit{Finesse coefficient} as
\begin{equation}
	F=\frac{4R}{(1-R)^2}
	\label{eq:finesse.mbi}
\end{equation}
Becomes
\begin{equation}
	R=\frac{F\sin^2\left( \frac{\delta}{2} \right)}{1+F\sin^2\left( \frac{\delta}{2} \right)}
	\label{eq:refcoefffinesse.mbi}
\end{equation}
The function on the right (divided by $F$) is known as the \textit{Airy function}.\\
It's possible to find easily also the transmission coefficient of the slab $T$, using
\begin{equation}
	T=1-R=\frac{1}{1+F\sin^2\left( \frac{\delta}{2} \right)}, \qquad F=\frac{4(1-T)}{T^4}
	\label{eq:transcoefffinesse.mbi}
\end{equation}
The peaks and the shadows of the image on the screen can be calculated by optimization calculus, using the function
\begin{equation*}
	I_R(\delta)=I_0F\frac{1}{1+F\sin^2\left( \frac{\delta}{2} \right)}=RI_0
\end{equation*}
Deriving with respect to delta then
\begin{equation*}
	\pdv{I_R}{\delta}=I_0F\frac{\sin\left( \frac{\delta}{2} \right)\cos\left( \frac{\delta}{2} \right)}{\left( 1+F\sin^2\left( \frac{\delta}{2} \right)^2 \right)}=0
\end{equation*}
Which implies that the maxima and minima of the intensity are at the following values of $\delta$
\begin{equation}
	\begin{dcases}
		\delta_m=2m\pi&max\left\{ R \right\}\\
		\delta_m'=(2m+1)\pi&\max\left\{ T \right\}
	\end{dcases}
	\label{eq:maximaminimaI.mbi}
\end{equation}
Remembering that $\delta$ is a function of the wavelength and optical path, we have
\begin{equation*}
	\delta(k)=2kd\cos(\theta), \quad\delta(\lambda)=\frac{4\pi d}{\lambda}\cos\theta, \quad\delta(\omega)=\frac{2\omega n}{c}d\cos\theta
\end{equation*}
Noting the dependence on the frequency of $\delta$, it's clear that the slab is useful for distinguishing between waves with different frequencies, as different frequencies will have different peaks, precisely, noting that at different \textit{peak orders}, defined by the whole number $m\in\Z$
\begin{equation*}
	\begin{aligned}
		\omega_m=\frac{m\pi c}{dn}
	\end{aligned}
\end{equation*}
And that, developing the transmission intensity around these peaks, i.e. at $\omega-\omega_m$, we have
\begin{equation}
	I_T=I_0T^2\frac{1}{T^2+4(1-T)\sin^2\left( \frac{dm}{\pi c}(\omega-\omega_m) \right)}\approxeq I_0\frac{\frac{\sigma^2}{4}}{\frac{\sigma^2}{4}+(\omega-\omega_m)^2}
	\label{eq:transintdifffreq.mbi}
\end{equation}
Where $\sigma$ is a constant which depends only on the properties of the material
\begin{equation*}
	\frac{\sigma^2}{4}=\frac{4T^2c^2}{d^2n^2}\frac{1}{4(1-T)}
\end{equation*}
This is \textit{exactly} as if we studied the behavior of a cavity. Experimentally it can be interpreted as a \textit{laser cavity}.\\
Summing for each transmission, and using the cavity limits $0\le m\le N-1$ noting that what we found is exactly the power spectrum of the cavity, from the Wiener-Khinchin theorem that the self correlation of the beam is simply the Fourier transform of the power spectrum, Therefore
\begin{equation*}
	\gamma(\tau)=c\sum_{m=0}^{N-1}\fopr{\frac{\frac{\sigma^2}{4}}{\frac{\sigma^2}{4}+(\omega-\omega_m)^2}}(\tau)=c\sum^{N-1}_{m=0}e^{\frac{\abs{\sigma}\tau}{2}}e^{i\omega_m\tau}
\end{equation*}
Which, after summation, gives
\begin{equation}
	\gamma(\tau)=ce^{\frac{\abs{\sigma}\tau}{2}}\frac{1-e^{iN\omega_m\tau}}{1-e^{i\omega_m\tau}}\quad c\in\Cf
	\label{eq:selfcorrlasercavity.mbi}
\end{equation}
The visibility of fringes is simply $\mathcal{V}=\abs{\gamma}=\sqrt{\gamma\cc{\gamma}}$. Evaluating the parenthesis multiplications and writing the correct trigonometric functions we have (inglobating a $\sqrt{2}$ in a constant $k$
\begin{equation}
	\mathcal{V}(\tau)=ke^{\frac{\abs{\sigma}\tau}{2}}\sqrt{\frac{1-\cos(N\omega_m\tau)}{1-\cos(\omega_m\tau)}}=ke^{\frac{\abs{\sigma}\tau}{2}}\abs{\frac{\sin\left( \frac{N\omega_m\tau}{2} \right)}{\sin\left( \frac{\omega_m\tau}{2} \right)}}
	\label{eq:visnonnorm.mbi}
\end{equation}
Normalizing everything, and using $\abs{\gamma(0)}=1$, we have $k=N^{-1}$, which gives finally
\begin{equation}
	\mathcal{V}(\tau)=\frac{e^{\frac{\abs{\sigma}\tau}{2}}}{N}\abs{\sin\left( \frac{N\omega_m\tau}{2} \right)\csc\left( \frac{\omega_m\tau}{2} \right)}
	\label{eq:visibilitylasercavity.mbi}
\end{equation}
\subsection{Fabry-Perot Instruments}
Fabry-Perot interferometers utilize the results obtained from the study of multi-beam interference with broad sources of light. They're usually used to determine the frequency of waves.\\
Their construction is similar to a laser cavity, where the slab gets substituted by two semi-transparent mirrors which replicate the cavity. In this case, since there is air inside these mirrors, we have
\begin{equation*}
	r_1=r_2=r, \quad t_1=t_2=t
\end{equation*}
As before, the path difference between the $n$-th and the $n+1$-th reflection is 
\begin{equation*}
	d=2nd\cos\theta
\end{equation*}
Therefore, the phase difference in a single double reflection is
\begin{equation*}
	\delta=\frac{4\pi d}{\lambda}\cos\theta
\end{equation*}
Hence, the final outgoing reflected and trasmitted amplitude are
\begin{equation}
	\left\{ \begin{aligned}
			E_R&= rE_0+t^2E_0\sum_{n=0}^{\infty}r^{2n+1}e^{in\delta}\\
			E_T&= t^2E_0\sum_{n=0}^{\infty}r^{2n}e^{in\delta}
	\end{aligned}\right.
	\label{eq:eretout.fpi}
\end{equation}
Noting that $r^2<1$ and that $r^2=R, \ t^2=T$, we have
\begin{equation}
	\begin{aligned}
		E_T&= \frac{t^2E_0}{1-Re^{i\delta}}\\
		I_T&= \frac{T^2I_0}{\abs{1-Re^{i\delta}}^2}
	\end{aligned}
	\label{eq:etitout.fpi}
\end{equation}
As with the laser cavity we have that $\abs{1-Re^{i\delta}}^2=(1-R)^2(1+F\sin^2(\delta/2))$, with $F$ being the finesse of the instrument, defined in \eqref{eq:finesse.mbi}.\\
Considered also the absorbment of some of the intensity, i.e. noting that $R+T+A=1$, with $A$ being the absorption coefficient, and considering also that $r\in\Cf$ will bring a phase shift itself, precisely, for each reflection some value $\delta_r/2\in[0, 2\pi)$ we get that, writing $\Delta=\delta+\delta_r$ that firstly
\begin{equation*}
	\frac{T^2}{(1-R)^2}=\left( \frac{1-R-A}{1-R} \right)^2
\end{equation*}
And therefore, for our realistic interferometer
\begin{equation}
	I_T=I_0\left( 1-\frac{A}{1-R} \right)^2\frac{1}{1+F\sin^2\left( \frac{\Delta}{2} \right)}
	\label{eq:realfinesse.fpi}
\end{equation}
As usual, we find the maxima of the Airy function, which correspond to
\begin{equation}
	\Delta=2N\pi=\frac{4\pi d}{\lambda}\cos\theta-\delta_r
	\label{eq:maxima.fpi}
\end{equation}
The integer $N\in\N$ is known as the \textit{order of interference} of the beams, which indicates the optical distance difference of two beams with different $\lambda$.\\
At this maxima we have
\begin{equation}
	I_{max}=\frac{T^2I_0}{(1-R)^2}=I_0\left( 1-\frac{A}{1-R} \right)^2
	\label{eq:peakint.fpi}
\end{equation}
These instruments are used to measure the wavelength of a source with high precision. A real Fabry-Perot instrument takes the light coming from a broad source of light, collimates it towards the two mirrors described before and then with another lense collimates the outgoing rays to a single point in the measuring screen. There are two configurations for a Fabry-Perot instrument, one is the \textit{etalon} in which the mirrors are fixed in position, and another, known as the interferometer, where the mirrors can be moved in order to change the phase difference $\delta$ obtained by the multiple reflections.
\subsubsection{Resolution Power of Fabry-Perot Instruments}
In order to actually measure the wavelenghts (or frequencies) of the broad source in question, we gotta understand what it means to \textit{resolve} two lines in the interference pattern obtained.\\
For a simpler evaluation, consider $A=\delta_r=0$, and call $\Gamma$ the line width at half height at the peak. Consider two lines, one at order $m$ and one at order $m+1$.\\
In order to call the two peaks resolved, we employ \textit{Taylor's criterion}, which states that two lines are resolved if they, at maximum, intersecate at the half-height point, where the intensity is $I_0/2$. Therefore if $I=I_0/2$ at the half height point, we have then, using \eqref{eq:peakint.fpi}
\begin{equation*}
	\delta=2\pi m+\frac{\Gamma}{2}=\Delta
\end{equation*}
Where we moved from the peak of exactly half line width, bringing ourselves at the intersection point of the two lines.\\
Applying the aforementioned criterion, we get then
\begin{equation*}
	\frac{I_0}{2}=\frac{T^2I_0}{1+F\sin^2\left( m\pi+\frac{\Gamma}{4} \right)}\implies\frac{1}{2}=\left( 1-\frac{A}{1-R} \right)^2\frac{1}{1+F\sin^2\left( m\pi+\frac{\Gamma}{4} \right)}
\end{equation*}
Imposing $A=0$, and noting that $m\pi+\Gamma/4<<1$ we have, after some algebraic juggling
\begin{equation*}
	1=F\sin^2\left( m\pi+\frac{\Gamma}{4} \right)\approx \frac{F\Gamma^2}{16}
\end{equation*}
Solving for gamma we get that a line will be resolved (for Taylor), \textit{if and only if} its line width is exactly equal to the following
\begin{equation}
	\Gamma=\frac{4}{\sqrt{F}}
	\label{eq:linewidthres.fpi}
\end{equation}
I.e., if the maximas are at a distance $d_m>\Gamma$. It's clear that, due to the definition of finesse, the minimum resolution distance is tied only to the instrument and not to the properties of the wave
\subsubsection{Spectral Resolution}
The ``goodness'' of a measure with a Fabry-Perot instrument is evaluated with what is known as \textit{Resolving Power}, $RP$. This value is a pure number defined by the ratio of the measured wavelength (or frequency) of the measured wave with the smallest resolvable interval of wavelength (frequency) as
\begin{equation}
	RP=\frac{\lambda}{\Delta\lambda_{min}}=\frac{\nu}{\Delta\nu_{min}}=\frac{\omega}{\Delta\omega_{min}}
	\label{eq:rp.fpi}
\end{equation}
It's clear that if the instrument can measure a smaller interval, then the resolution power will be greater.\\
We can tie this value with what we wrote before noting that, close to the maximum of the line
\begin{equation*}
	\delta=\frac{4\pi d}{\lambda}\cos\theta=2\pi m\implies\frac{2d}{\lambda}\cos\theta=m\qquad m\in\Z
\end{equation*}
Or, in terms of only wavelengths 
\begin{equation*}
	m\lambda=2d\cos\theta
\end{equation*}
Differentiating, we can also say that
\begin{equation*}
	\begin{dcases}
		m\Delta\lambda=2d\sin\theta\Delta\theta\\
		\Delta\delta=\frac{4\pi d}{\lambda}\sin\theta\Delta\theta
	\end{dcases}
\end{equation*}
Or, solving
\begin{equation*}
	2\pi\Delta\delta=2\pi m\frac{\Delta\lambda}{\lambda}=\frac{4}{\sqrt{F}}
\end{equation*}
Therefore, we have then
\begin{equation}
	\frac{\Delta\lambda}{\lambda}=\frac{\pi}{2m}\frac{1}{\sqrt{F}}
	\label{eq:deltalonl.fpi}
\end{equation}
Which, inserted into the formula for the $RP$, we have that
\begin{equation}
	RP=\frac{\pi}{2}m\sqrt{F}
	\label{eq:rp2.fpi}
\end{equation}
Therefore, the resolving power is also linearly dependent to the order of interference.\\
Another important part of spectral analysis of waves using Fabry-Perot instruments is the distance between two maxima. This is commonly known as the \textit{Free Spectral Range} of the instrument.\\
By definition, we have $\delta_m=2\pi m$
\begin{equation}
	\begin{aligned}
		\delta_{m+1}-\delta_m&= 2\pi\\
		4\pi d\left( \frac{1}{\lambda_{m+1}}-\frac{1}{\lambda_m} \right)\cos\theta&= 2\pi
	\end{aligned}
	\label{eq:fsrdef.fpi}
\end{equation}
The free spectral range, or $FSR$ is then defined
\begin{equation*}
	FSR=\frac{1}{\frac{1}{\lambda_{m+1}-\lambda_m}}=\frac{\lambda^2}{2d\cos\theta}
\end{equation*}
Since we're talking about maxima, we have that $m\lambda=2d\cos\theta$, which inserted into the previous equation gives
\begin{equation}
	FSR=\frac{\lambda}{m}
	\label{eq:fsr.fpi}
\end{equation}
Therefore, the FSR gets smaller with higher orders
Defining also the \textit{reflecting finesse} as
\begin{equation}
	\mathcal{F}=\frac{\pi}{2}\sqrt{F}
	\label{eq:reffinesse.fpi}
\end{equation}
We can also redefine the RP as
\begin{equation}
	RP=m\mathcal{F}=m\pi\frac{\sqrt{R}}{1-R}
	\label{eq:rprfin.fpi}
\end{equation}
Therefore tying closely the resolving power of the instrument to its physical properties.\\
In general, then, given $m\in\Z$ the interference order of the wave in study, we can summarize what we wrote in maths as follows
\begin{itemize}
\item The resolving power (RP) is tied to the physical properties of the instrument and grows with $m$
\item The distance between peaks (FSR) gets smaller with greater $m$
\end{itemize}
It's clear that even if the order of interference grows and with it the resolving power, the distance between peaks will reduce, and there will be a point where the peaks won't be resolved anymore. This fact sigillates etalons to a single specific group of measures, while interferometers can be used with a wider range of tests.\\
\end{document}
