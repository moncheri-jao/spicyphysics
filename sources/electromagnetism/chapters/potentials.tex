\documentclass[../electromagnetism.tex]{subfiles}
\begin{document}
\section{Maxwell's Equation for Potentials}
As we have seen already, Maxwell's equations are the following
\begin{equation}
	\left\{ \begin{aligned}
			\del_iE^i&=\frac{\rho}{\epsilon_0}\\
			\cpr{i}{j}{k}\del^jE^k&=-\pdv{B^i}{t}\\
			\del_iB^i&=0\\
			\cpr{i}{j}{k}\del^jB^k&=\mu_0J^i+\frac{1}{c^2}\pdv{E^i}{t}
	\end{aligned}\right.
	\label{eq:maxwellpotentialchapter}
\end{equation}
If we wanted to write the potential formulation of this equation, we must know that in general, the potentials might be different. In fact, Coulomb's law and Biot-Savart only work in the static case, i.e. where $\del_tE=\del_tB=0$.\\
Using that $\del_iB^i=0$ from the third equation we can say for sure that
\begin{equation*}
	\cpr{i}{j}{k}\del^jA^k=B^i
\end{equation*}
And inserting it in the second we get
\begin{equation*}
	\begin{aligned}
		\cpr{i}{j}{k}\del^jE^k&=-\pdv{t}\cpr{i}{j}{k}\del^jA^k\\
		\cpr{i}{j}{k}\del^j\left( E^k+\pdv{A^k}{t} \right)&=0
	\end{aligned}
\end{equation*}
The second line immediately tells us that the vector field inside must be the gradient of some scalar field!\\
Using $\del^iV=-E^i$ then we can rewrite the electric potential as a sum of the time variation of the vector potential and the scalar potential, which gives us
\begin{equation}
	E^i(x^i,t)=-\pdv{V}{x_i}-\pdv{A^i}{t}
	\label{eq:electric-dynamicpotential}
\end{equation}
Now that we have the potentials for the dynamic case we know that the Poisson equation for the electric field then becomes
\begin{equation*}
	\del_iE^i=-\del_i\del^iV-\pdv{t}\del_iA^i=\frac{\rho}{\epsilon_0}
\end{equation*}
Or, written better
\begin{equation}
	\del_i\del^iV+\pdv{t}\del_iA^i=-\frac{\rho}{\epsilon_0}
	\label{eq:poissondynamiceq-V}
\end{equation}
Also, for the equivalent vectorial Poisson equation for the $A^i$ field
\begin{equation*}
	\cpr{i}{j}{k}\cpr{k}{l}{m}\del^lA^m=\mu_0J^i-\frac{1}{c^2}\pdv{t}\left( \pdv{V}{x_i}-\frac{1}{c^2}\pdv{A^i}{t} \right)
\end{equation*}
Rewriting the first double cross product as
\begin{equation*}
	\del^i\del_jA^j-\del_j\del^jA^i
\end{equation*}
And bringing the time derivative of the gradient of V to the left, while grouping it inside the $\del^i$ operator (it's linear), and bringing with it the second time derivative on $A^i$ we have (note also that I changed sign on both sides)
\begin{equation}
	\del_j\del^jA^i-\pdv{x_i}\left( \frac{1}{c^2}\pdv{V}{t}+\del_jA^j \right)-\frac{1}{c^2}\pdv[2]{A^i}{t}=-\mu_0J^i
	\label{eq:poissondynamiceq-A}
\end{equation}
Both together give us the most general possible way to formulate Maxwell equations for potentials, which reduce to two coupled non-homogeneous second order PDEs 
\begin{equation}
	\left\{ \begin{aligned}
			&\del_j\del^jV+\pdv{t}\del_jA^j=-\frac{\rho}{\epsilon_0}\\
			&\del_j\del^jA^i-\frac{1}{c^2}\pdv[2]{A^i}{t}-\pdv{x_i}\left( \frac{1}{c^2}\pdv{V}{t}+\del_jA^j \right)=-\mu_0J^i
	\end{aligned}\right.
	\label{eq:mweqpot}
\end{equation}
\subsection{Gauge Freedom}
What we've learned before about electromagnetic potentials is that \textit{they're gauge-modifiable}. Depending on what we really need we might choose between any given gauge, since Maxwell's equation are gauge-invariant.\\
The first gauge we will use is the most common one, it's useful when dealing with magnetostatics or when we really need to find V. It's \textit{Coulomb's gauge}.\\
Here we set the divergence of $A^i$ to zero, and the first equation of \eqref{eq:mweqpot} reduces back to a Poisson's equation. The second simplifies a bit, but it's not easy to solve\ldots
\begin{equation}
	\left\{ \begin{aligned}
			&\del_i\del^iV=-\frac{\rho}{\epsilon_0}\\
			&\del_j\del^jA^i-\frac{1}{c^2}\pdv[2]{A^i}{t}-\frac{1}{c^2}\pdv[2]{V}{x_i}{t}=-\mu_0J^i
	\end{aligned}\right.
	\label{eq:mweqpotcoulombg}
\end{equation}
Another thing to note here is that V \emph{cannot} be observable, moving charges change $\rho$ which changes $V$ istantaneously, it's not Lorentz invariant.\\
The second most important gauge we can define it's Lorenz's\footnote{Lorenz, not Lorentz, apparently} gauge, which defines the divergence of A as follows
\begin{equation}
	\del_iA^i=\frac{1}{c^2}\pdv{V}{t}
	\label{eq:lorenzgauge-mwpot}
\end{equation}
Then, reinserting it back to \eqref{eq:mweqpot} we get by immediate substitution two uncoupled non-homogeneous wave equations
\begin{equation}
	\left\{\begin{aligned}
		\del_j\del^jV-\frac{1}{c^2}\pdv[2]{V}{t}&=-\frac{\rho}{\epsilon_0}\\
		\del_j\del^jA^i-\frac{1}{c^2}\pdv[2]{A^i}{t}&=-\mu_0J^i\\
	\end{aligned}\right.
	\label{eq:relwaveeq}
\end{equation}
This is a \textit{relativistic wave equation} with sources $\rho$ and $J^i$.\\
Note that if we define the four-gradient as the 4-vector composed by the following components ($\mu=0,\cdots,3$)
\begin{equation}
	\del_\mu=\left( \frac{1}{c}\del_t,-\del_i \right)
	\label{eq:4-gradient-mwpot}
\end{equation}
We have that, formally
\begin{equation*}
	\del_\mu\del^\mu=\frac{1}{c^2}\del^2_t-\del_i\del^i=\square
\end{equation*}
Where the box operator is known as the \textit{D'Alambertian}, which is the equivalent of the Laplacian in 4 spacetime dimensions. Therefore we can also write
\begin{equation}
	\left\{\begin{aligned}
		\square V&= \del^\mu\del_\mu V=\frac{\rho}{\epsilon_0}\\
		\square A^i&= \del^\mu\del_\mu A^i=\mu_0J^i
	\end{aligned}\right.
	\label{eq:relativisticwaveeq-mwpot}
\end{equation}
\section{Retarded Potentials}
Using \eqref{eq:relativisticwaveeq-mwpot} and setting the time derivatives as 0, we get back Poisson's equations for both potentials, for which we know already the general solution for a volume $V$.
\begin{equation*}
	\begin{aligned}
		V(x^j)&=\frac{1}{4\pi\epsilon_0}\int_{V}^{}\frac{\rho(\tilde{x}^j)}{r}\dd^3{\tilde{x}}\\
		A^i(x^j)&=\frac{\mu_0}{4\pi}\int_{V}^{}\frac{J^i(\tilde{x}^j)}{r}\dd^3{\tilde{x}}
	\end{aligned}
\end{equation*}
We can say, from the previous equations, that the interaction travels at speed $c$, therefore we might imagine the time progression of the interaction as ``retarded in time'' by a factor of $r/c$. We then define the \textit{retarded time} as
\begin{equation}
	t_r=t-\frac{r}{c}
	\label{eq:retardedtime}
\end{equation}
We therefore can imagine a solution to those equations as 
\begin{equation}
	\begin{aligned}
		V(x^j,t_r)&=\frac{1}{4\pi\epsilon_0}\int_{V}^{}\frac{\rho(\tilde{x}^j,t_r)}{r}\dd^3{\tilde{x}}\\
		A^i(x^j,t_r)&=\frac{\mu_0}{4\pi}\int_{V}^{}\frac{J^i(\tilde{x}^j,t_r)}{r}\dd^3{\tilde{x}}
	\end{aligned}
	\label{eq:retardedsolution}
\end{equation}
Note that we could imagine this solution only due to the mathematical shape of the equation, it cannot be done the same way for the fields.\\
Now let's check if this idea we had is a solution for the relativistic equations. Noting that
\begin{equation*}
	\del_it_r=-\frac{\hat{x}^i}{c}
\end{equation*}
We have, after using the chain rule
\begin{equation*}
	\pdv{\rho}{x^i}=-\frac{1}{c}\pdv{\rho}{t}\hat{x}^i
\end{equation*}
And therefore
\begin{equation*}
	\del_iV=-\frac{1}{4\pi\epsilon_0}\int_{V}^{}\frac{1}{c}\pdv{\rho}{t}\frac{\hat{x}^i}{r}-\rho\frac{\hat{x}^i}{r^2}\dd^3{x}
\end{equation*}
Applying again the del operator we get
\begin{equation*}
	\del^i\del_iV=-\frac{1}{4\pi\epsilon_0}\int_{V}^{}\frac{1}{c}\left( \frac{\hat{x}^i}{r}\pdv[2]{\rho}{t}{x^i}+\pdv{\rho}{t}\frac{\del_i\hat{x}^i}{r}+\pdv{\rho}{t}\hat{x}^i\del_i\left( \frac{1}{r} \right) \right) +\left( \frac{\hat{x}^i}{r^2}\pdv{\rho}{x^i}+\rho\pdv{x^i}\left( \frac{\hat{x}^i}{r^2} \right) \right)\dd^3{x}
\end{equation*}
But
\begin{equation*}
	\pdv[2]{\rho}{t}{x^i}=-\frac{1}{c}\pdv[2]{\rho}{t}\hat{x}^i
\end{equation*}
And
\begin{equation*}
	\del_i\left( \frac{\hat{x}^i}{r} \right)=\frac{1}{r^2},\qquad\del_i\left( \frac{\hat{x}^i}{r^2} \right)=4\pi\delta^3(x^i)
\end{equation*}
Therefore, finally
\begin{equation}
	\del^i\del_iV=\frac{1}{4\pi\epsilon_0c^2}\int_{V}^{}\frac{1}{r}\pdv[2]{\rho}{t}\dd^{3}{x}-\frac{\rho}{\epsilon_0}\delta^3(x^i)
	\label{eq:lablas}
\end{equation}
Seeing immediately on the right the time second time derivative with respect to $ct$ of $V$, bringing it to the left and playing with minuses we get again the awaited Maxwell equation.
\begin{equation*}
	\square V=\frac{\rho}{\epsilon_0}
\end{equation*}
The calculation for $A^i$ is completely analogous. Note that we could also have chosen an \textit{advanced time} $t_a$ defined as
\begin{equation*}
	t_a=t+\frac{r}{c}
\end{equation*}
Everything comes back to the two Maxwell equations, but the physical sense gets lost since the potentials we found don't respect causality, they sense the change \emph{before} it actually happens in the chosen reference frame.
\section{Jefimenko's Equations}
Given the two retarded potentials defined in \eqref{eq:retardedsolution}, we could imagine to determine the electric and magnetic field generated by both. Since the retarded potentials, as we have shown, solve \textit{generally} Maxwell's equations \eqref{eq:relativisticwaveeq-mwpot}, the fields will also solve generally Maxwell's equations for the fields.\\
We begin by finding $E^i$. We know that
\begin{equation*}
	E^i=-\del^iV-\del_tA^i
\end{equation*}
Therefore
\begin{equation}
	E^i=-\frac{1}{4\pi\epsilon_0}\int_{V}^{}\del^i\left( \frac{\rho}{r} \right)\dd^{3}{x}-\frac{\mu_0}{4\pi}\int_{V}^{}\frac{1}{r}\pdv{J^i}{t}\dd^{3}{x}
	\label{eq:Efieldimenko}
\end{equation}
From the previous calculations, we already know that
\begin{equation*}
	\del^i\left( \frac{\rho}{r} \right)=-\frac{1}{c}\pdv{\rho}{t}\hat{x}^i-\rho\frac{\hat{x}^i}{r^2}
\end{equation*}
And we get easily the first Jefimenko equation for the $E$ field
\begin{equation*}
	E^i(x^j,t)=\frac{1}{4\pi\epsilon_0}\int_{V}^{}\left( \frac{1}{c}\pdv{\rho}{t}+\frac{\rho}{r} \right)\frac{\hat{x}^i}{r}\dd^{3}{x}-\frac{\mu_0}{4\pi}\int_{V}^{}\frac{1}{r}\pdv{J^i}{t}\dd^{3}{x}
\end{equation*}
Using $\epsilon_0=(\mu_0c^2)^{-1}$ we can group everything in a clearer equation
\begin{equation}
	E^i(x^j,t)=\frac{1}{4\pi\epsilon_0}\int_{V}^{}\left( \frac{1}{c}\pdv{\rho}{t}+\frac{\rho}{r} \right)\frac{\hat{x}^i}{r}-\frac{1}{rc^2}\pdv{J^i}{t}\dd^{3}{x}
	\label{eq:Efimenko}
\end{equation}
For $B^i$ the calculations are slightly harder due to the presence of the curl, but with some discipline are doable. We have that
\begin{equation*}
	B^i=\cpr{i}{j}{k}\del^jA^k=\frac{\mu_0}{4\pi}\int_{V}^{}\frac{1}{r}\cpr{i}{j}{k}\del^jJ^k+\cpr{i}{j}{k}J^i\del^k\left( \frac{1}{r} \right)\dd^{3}{x}
\end{equation*}
But, by definition we have
\begin{equation*}
	\del_iJ^k=\left( \del_tJ^k\del_k \right)t_r=-\frac{1}{c}\pdv{J^k}{t}\pdv{r}{x^k}=\frac{1}{c}\pdv{J^k}{t}\hat{x}_k
\end{equation*}
So, the cross product is simply
\begin{equation*}
	\cpr{i}{j}{k}\del^jJ^k=\frac{1}{c}\cpr{i}{j}{k}\del_tJ^j\hat{x}^k
\end{equation*}
The second part instead comes immediately from the gradient of $r^{-1}$, and we have 
\begin{equation}
	B^i(x^j,t)=\frac{\mu_0}{4\pi}\int_{V}^{}\frac{1}{r}\cpr{i}{j}{k}\left( \frac{J^j}{r}+\frac{1}{c}\pdv{J^j}{t} \right)\hat{x}^k\dd^{3}{x}
	\label{eq:Bfimenko}
\end{equation}
Both Jefimenko equations grouped are, therefore (and finally)
\begin{equation}
	\left\{	\begin{aligned}
		E^i(x^k,t)&=\frac{1}{4\pi\epsilon_0}\int_{V}^{}\frac{1}{r}\left( \frac{1}{c}\pdv{\rho}{t}\hat{x}^i+\rho\frac{\hat{x}^i}{r}-\frac{1}{c^2}\pdv{J^i}{t}\right)\dd^{3}{\tilde{x}}\\
		B^i(x^k,t)&=\frac{\mu_0}{4\pi}\int_{V}^{}\frac{1}{r}\cpr{i}{j}{k}\left( \frac{J^j}{r}+\frac{1}{c}\pdv{J^j}{t} \right)\hat{x}^k\dd^{3}{\tilde{x}}
	\end{aligned}\right.
	\label{eq:Jefimenko}
\end{equation}
%%TODO Liénard-Wiechert potentials, point charge potentials
\end{document}
