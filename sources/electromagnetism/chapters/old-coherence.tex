\documentclass[../electromagnetism.tex]{subfiles}
\begin{document}
\section{Coherent Waves}
\subsection{Interference}
Consider two interacting waves with electric fields $E_{(1)}^i$ and $E_{(2)}^i$. In general these two waves are solutions of the Maxwell wave equations, therefore, the most general shape they can have is the following:
\begin{equation}
	\left\{ \begin{aligned}
			E_{(1)}^i=E_1^ie^{ik_1^ir_i-i\omega t+i\phi_1}\\
			E_{(2)}^i=E_2^ie^{ik_2^ir_i-i\omega t+i\phi_2}\\
	\end{aligned}\right.
	\label{eq:twogenwavesint}
\end{equation}
Where $\phi_1,\phi_2\in[0,2\pi]$ are two general phase factors.\\
\begin{dfn}[Mutually Coherent Waves]
	Given two general waves, they're said to be \emph{coherent} if and only if, given the phases $\phi_1$ and $\phi_2$, we have
	\begin{equation}
		\phi_1-\phi_2=k
		\label{eq:coherencedef}
	\end{equation}
	Where $k\in\R$ is a constant real number.
\end{dfn}
Getting back to our interference pattern, we have that the total field that will be measured must be a linear superposition of these two general field. Therefore, the measured total intensity $I$ is, (considering two linearly polarized waves, for ease of calculation))
\begin{equation*}
	I=\norm{E}^2=\left( E_{(1)}^i+E_{(2)}^i \right)\cc{\left( E^{(1)}_i+E^{(2)}_i \right)}=E_1^2+E_2^2+2\real{\left(E_{(1)}^i\cc{E^{(2)}_i}\right)}
\end{equation*}
Evaluating the third addendum on the previous expression, using the general wave solution \eqref{eq:twogenwavesint}, we get
\begin{equation*}
	2\real{\left(E_{(1)}^i\cc{E_i^{(2)}}\right)}=2E^i_1E_i^2\real{e^{i(k_1^i-k_2^i)r_i+i(\phi_1-\phi_2)}}
\end{equation*}
Evaluating the right hand side of the previous equation, we have that the intensity will in general be calculated (for linearly polarized waves) as
\begin{equation}
	I=I_1+I_2+2E^i_1E_i^2\cos\left[ \left( k_1^i-k_2^i \right)r_i+i(\phi_1-\phi_2) \right]
	\label{eq:intensityinterference}
\end{equation}
Where $I_1$ and $I_2$ are the partial intensities of the single waves. The third term, is known as the \textit{interference term}, which, since it's comprised between $[-1,1]$, makes the intensity oscillate between $I_1+I_2\le I \le I_1+I_2$ depending on
\begin{equation}
	\theta(r^i)=i(k_1^i-k_2^i)r_i+i(\phi_1-\phi_2)
	\label{eq:cosinterference}
\end{equation}
The result are interference fringes. It's also clear that this can happen only and only if the waves are mutually coherent. If they weren't $\phi_1-\phi_2$ would vary randomly, and, by definition
\begin{equation*}
	\expval{\cos\left( \theta(r^i) \right)}=0
\end{equation*}
Which implies that the interference term is in general null, and $I=I_1+I_2$ always.\\
Interference fringes can be obtained also through spatial variations, since the argument of the cosine depends both on the phase difference and the distance $r$.\\
Interference fringes also cannot appear in the case of two orthogonally polarized waves, because by definition their scalar product is zero, eliminating the interference term, independently from their mutual coherence.
\subsection{Young's Double Slit Experiment}
In 1802 dr. Young et al. managed to build an experiment which managed to show the interference pattern caused by the interference of two coherent electromagnetic waves.\\
The experimental setup is pretty simple, a coherent source (point-like) $S$ emits a single electromagnetic wave, which passes through two slits $S_1$ and $S_2$, placing a detector (in case of light a simple wall is enough), it's possible to observe regular peaks of intensity.\\
Theoretically it all depends on the phase difference between the two waves that come out of each slit.\\
Taken a random point $P$ on the screen, we can say that it will be at a distance $d_1$ from $S_1$ and $d_2$ from $S_2$. The phase difference will obviously depend on the difference of the two distances from the chosen point.\\
For convenience, we search for the points of max intensity, (i.e., when $\cos\theta(r)=1$ and the interference term is the maximum possible). Due to coherence between the two fields, we can say without any problem that
\begin{equation*}
	\theta(r)=k^ir_i=k_0(d_2-d_1)
\end{equation*}
Hence, our intensity minimums will be 
\begin{equation*}
	\max\left( I \right)=\left. I\right|_{\cos\theta(r)=1}\implies k_0(d_2-d_1)=\pm2n\pi,\quad n\in\N
\end{equation*}
Remembering that $k_0=2\pi\lambda^{-1}$, we get the final simple result
\begin{equation}
	\abs{d_2-d_1}=n\lambda
	\label{eq:interfdist}
\end{equation}
I.e., the distance between the intensity peaks is strictly tied to the distance between the slits and the wavelength of the electromagnetic wave, where the successive peaks will always be at integer multiples of the wavelenght.\\
In general, for two slits distant $h$ between each other, chosen a reference frame for which the origin is the middle point between the two slits and the screen is distant $x$ from this, taken a general point $(x,y)$ on it we will have that
\begin{equation*}
	\abs{d_2-d_1}=\sqrt{x^2+\left( y+\frac{h}{2} \right)^2}+\sqrt{x^2+\left( y-\frac{h}{2} \right)^2}
\end{equation*}
Therefore, solving the previous equation, searching for the intensity peaks, we have, after a second order approximation that
\begin{equation*}
	\abs{d_2-d_1}\approx x\left( 1+\frac{1}{2x^2}\left( y+\frac{h}{2} \right)^2 \right)-x\left( 1+\frac{1}{2x^2}\left( y-\frac{h}{2} \right)^2 \right)\approx n\lambda
\end{equation*}
And solving 
\begin{equation}
	y\approx\frac{nx\lambda}{h}
	\label{eq:peakdistanceapprox}
\end{equation}
I.e., the distance between peaks is approximately an integer multiple of the distance from the slits times the wavelength, over the distance between the two slits themselves.
\subsection{Michelson Interferometer}
Another slightly more complex system which analyzes the interference phenomena, is the Michelson interferometer. Michelson's interferometer was developed in 1880 by Michelson et al.\\
It's known as a \emph{division of amplitude} interferometer, for the simple reason that it divides the beam in two (ideally equal) parts after passing through a so called \emph{beam splitter}. A beam splitter is a particular optical device, which divides an electromagnetic wave in two (ideally equal) parts, sending one in the transverse direction.\\
For building up this interferometer, the setup is a source of coherent electromagnetic waves, a beam splitter, two mirrors and a detector. Placed the beam splitter at the center of the setup, with one of the two mirrors $M_2$ placed at a distance $d_2$ from the splitter, and the other one, $M_1$ at a distance $d_1$ on the other transmission axis of the splitter. Opposite to the mirror there will be our detector $D$, which will measure the interference fringes caused by the interaction of the two beams when they end up interacting again before the reflection at the beam splitter.\\
The observed fringes directly depend on the optical path of the two mirrors, which is modified by the simple motion of one of the two.\\
This setup is equivalent to having two point-like sources distant $d=\abs{d_2-d_1}$ from each other, interfering and reaching a detector.\\
Due to the two waves being ideally identical due to the inner workings of the beam splitter, (i.e., in this case $k_1=k_2$) the measured intensity at the detector will depend only on the phase shift induced by the optical distance traveled, and 
\begin{equation*}
	I=I_1+I_2+2\sqrt{I_1I_2}\cos\left( \phi_1-\phi_2 \right)
\end{equation*}
Again, the interference peaks will be when $\cos\left( \phi_1-\phi_2 \right)=1$, i.e., when
\begin{equation}
	\phi_1-\phi_2=k_0d=\frac{2\pi d}{\lambda}
	\label{eq:interferencepeaksmichmorley}
\end{equation}
With intensity
\begin{equation}
	I=I_1+I_2+2\sqrt{I_1I_2}
	\label{eq:peakintmichmorley}
\end{equation}
\section{Partial Coherence}
In reality, what we treated before is an approximation of reality. In fact, in general, two interfering electromagnetic waves \textit{aren't necessarily coherent}, or at least they aren't always coherent.\\
By definition of coherence, this means that the phases of the signals vary randomly through time, which brings forward the idea of averaging the total amplitude.
\begin{dfn}[Time Average]
	We define the \emph{time average} of a periodic function $f(t)$, as the following improper integral
	\begin{equation}
		\expval{f}=\lim_{T\to\infty}\frac{1}{T}\int_{0}^{T}f(t)\dd^{}{t}
		\label{eq:timeaveragedfn}
	\end{equation}
	Note how the final result doesn't depend on time.
\end{dfn}
Applied to what we found for the amplitude, we then have
\begin{equation}
	\expval{I}=I_1+I_2+2\real\left( \expval{E_1^i\cc{E_i^2}} \right)
	\label{eq:intterm}
\end{equation}
Where we conveniently defined $I_1=\expval{E_1^2}$ and $I_2=\expval{E_2^2}$.\\
Looking back at the previous experimental setups, it's clear how interference depends on the difference of length of the optical paths of the two interfering waves, due to our choice of time averaging we can see the difference as an added needed time to reach the detector, let's say $\tau$, for which, if $E_1$ reaches it at a time $t$, $E_2$ will reach it at a time $t+\tau$. Considering that the interference is seen when both waves have reached the detector, we have that the interference term is the following
\begin{equation}
	2\real\left( \expval{E_1^i(t)\cc{E_i^2(t+\tau)}} \right)
	\label{eq:interferencetermtau}
\end{equation}
The integral inside the real part operator, can then be redefined as follow, as a new function depending on $\tau$
\begin{dfn}[Correlation Function]
	The \emph{correlation function } $\Gamma_{12}(\tau)$ is defined as follows, forgetting the vectorial nature of electromagnetic waves for a second,
	\begin{equation}
		\Gamma_{12}(\tau)=\lim_{T\to\infty}\frac{1}{T}\int_{0}^{T}E_1(t)\cc{E_2(t+\tau)}\dd{t}
		\label{eq:correlationfunction}
	\end{equation}
	Clearly, $\Gamma_{12}(\tau)=\Gamma_{21}(\tau)$, and from this we can also define the \emph{autocorrelation function} $\Gamma_{11}(\tau)$ and $\Gamma_{22}(\tau)$. It's also evident that
	\begin{equation}
		\begin{aligned}
			\Gamma_{11}(0)&= I_1\\
			\Gamma_{22}(0)&= I_2
		\end{aligned}
		\label{eq:autocorrelationzero}
	\end{equation}
\end{dfn}
Another, more comfortable definition is the following
\begin{dfn}[Degree of Spatial Coherence]
	Defined the correlation function $\Gamma_{12}(\tau)$ as before, we define the \emph{degree of spatial coherence} $\gamma_{12}(\tau)$ as the normalized correlation function
	\begin{equation}
		\gamma_{12}(\tau)=\frac{\Gamma_{12}(\tau)}{\sqrt{\Gamma_{11}(0)\Gamma_{22}(0)}}=\frac{\Gamma_{12}(\tau)}{\sqrt{I_1I_2}}
		\label{eq:degreespatialcoherencedfn}
	\end{equation}
	Clearly, $-1\le\gamma_{12}(\tau)\le1$
\end{dfn}
Then, the irradiance or amplitude $I$, can be rewritten as follows
\begin{equation}
	I=I_1+I_2+2\sqrt{I_1I_2}\real\left( \gamma_{12}(\tau) \right)
	\label{eq:irradiancegamma12}
\end{equation}
In general, as we'll see later, $\gamma_{12}$ is a complex-valued function periodic in $\tau$. It's also obvious that the interference bands will appear only and only if $\gamma_{12}(\tau)\ne0$.\\
Two waves will then said to be
\begin{enumerate}
\item Completely coherent, if $\abs{\gamma_{12}(\tau)}=1$
\item Completely incoherent, if $\abs{\gamma_{12}(\tau)}=0$
\item Partially coherent, if $0\le\abs{\gamma_{12}(\tau)}\le1$
\end{enumerate}
The minimum and maximum values of amplitude in a regime of partial or complete coherence then, can be found as follows
\begin{equation}
	\begin{aligned}
		I_{max}&= I_1+I_2+2\sqrt{I_1I_2}\abs{\gamma_{12}(0)}\\
		I_{min}&= I_1+I_2-2\sqrt{I_1I_2}\abs{\gamma_{12}(0)}
	\end{aligned}
	\label{eq:maxmininterferenceint}
\end{equation}
\begin{dfn}[Fringe Visibility]
	Another important value in the analysis of interference is the \emph{visibility of fringes} $\mathcal{V}$, defined as follows
	\begin{equation}
		\mathcal{V}=\frac{I_{max}-I_{min}}{I_{max}+I_{min}}
		\label{eq:visibilitydef}
	\end{equation}
	Or
	\begin{equation}
		\mathcal{V}=\frac{2\sqrt{I_1I_2}\abs{\gamma_{12}(0)}}{I_1+I_2}
		\label{eq:visibilitydef2}
	\end{equation}
	Note that, by definition, then if $I_1=I_2=I$, the fringe visibility is simply the absolute value of the degree of coherence
	\begin{equation*}
		\mathcal{V}=\abs{\gamma_{12}(0)}
	\end{equation*}
	This constant, by definition, is $0\le\mathcal{V}\le1$, where a value of $0$ implies that there is no interference, while a value of $1$ implies the maximum possible interference between the waves
\end{dfn}
\subsection{Coherence Time and Coherence Length}
In order to understand better the regime of partial coherence, we take a quasimonochromatic electromagnetic wave, with a random phase $\phi(t)$, which is a periodic step function which changes value every \textit{coherence time} $\tau_0$.\\
Suppose that this electromagnetic wave gets split at some point, and the two resulting waves travel two different optical paths long $d_1,d_2$, and rejoin at some point creating an interference pattern. In general, the two waves will have the following mathematical shape
\begin{equation*}
	E^i(t)=E_0^ie^{-i\omega t+i\phi(t)}
\end{equation*}
Indicating the amplitude $I_{1,2}=\abs{E_0}^2$ we get that then, the \emph{self-degree of coherence} is 
\begin{equation*}
	\gamma(\tau)=\frac{\expval{E^i(t)E_i(t+\tau)}}{I}=\lim_{T\to\infty}\frac{1}{T}\int_{0}^{T}e^{-i\omega t+i\phi(t)}e^{i\omega(t+\tau)-i\phi(t+\tau)}\dd^{}{t}
\end{equation*}
Simplifying the exponentials inside we get
\begin{equation*}
	\gamma(\tau)=e^{i\omega\tau}\lim_{T\to\infty}\frac{1}{T}\int_{0}^{T}e^{i(\phi(t)-\phi(t+\tau))}\dd^{}{t}
\end{equation*}
For evaluating this integral we need to define the behavior of $\phi(t)-\phi(t+\tau)$. Firstly we know that it's a step function which varies randomly periodically with period $\tau_0$, hence, the difference will be zero until we find ourselves with $\phi(t+\tau)$ (or vice-versa, $\phi(t)$) in a different step. Mathematically, for the first period
\begin{equation*}
	\begin{dcases}
		\phi(t)-\phi(t+\tau)=0&0<t<\tau_0-\tau\\
		\phi(t)-\phi(t+\tau)=\Delta&\tau_0-\tau<t<\tau_0
	\end{dcases}
\end{equation*}
With $\Delta\in[0,2\pi]$ being a random value. Then, for a single period, the integral before becomes
\begin{equation*}
	\frac{e^{i\omega\tau}}{\tau_0}\left( \int_{0}^{\tau_0-\tau}\dd^{}{t}+\int_{\tau_0-\tau}^{\tau_0}e^{i\Delta}\dd^{}{t} \right)=\left( \frac{\tau_0-\tau}{\tau_0}e^{i\omega\tau}+\frac{\tau}{\tau_0}e^{i\Delta} \right)
\end{equation*}
Time-averaging the result, we get the normalized autocorrelation function for these general waves. Remembering the randomness of $e^{i\Delta}$ we have that its average will be $0$, leaving only the first part, which, simplified becomes
\begin{equation}
	\gamma(\tau)=\begin{dcases}
			\left( 1-\frac{\tau}{\tau_0} \right)e^{i\omega\tau}&\tau<\tau_0\\
			0&\tau\ge\tau_0
		\end{dcases}
	\label{eq:generalautocorrelation}
\end{equation}
In this case, $I_1=I_2$ (by definition), hence $\abs{\gamma}=\mathcal{V}$, i.e.
\begin{equation}
	\mathcal{V}=\abs{\gamma(\tau)}=\begin{dcases}
			1-\frac{\tau}{\tau_0}&\tau<\tau_0\\
			0&\tau\ge\tau_0
		\end{dcases}
	\label{eq:visibilityequalwaves}
\end{equation}
This clearly gives the name \emph{correlation time} to $\tau_0$, it's evident how the waves become completely uncorrelated after for $\tau\ge\tau_0$, and the fringes are not visible!
Analogously, we can define a related length called \emph{coherence length}, as follows
\begin{equation}
	l_c=c\tau_0
	\label{eq:coherencelength}
\end{equation}
This value corresponds to the maximum length that of an uninterrupted wave train
%%TODO Spectral resolution, Fourier analysis of wave train. Check title of subsection see if changing to section
\subsection{Spectral Resolution of Finite Wave Trains}
In nature, strictly monochromatic sources of electromagnetic waves do not exist, and the best we can hope for is a finite wave train with frequencies spread around some mean value $\expval{\omega}=\omega_0$.\\
By definition of finite wave train, there should be some way to analyze the relationship between the frequency spread (or line width) and the coherence of a source.\\
Fourier calculus comes in our help for this task. Called $f(t)$ our wavefunction we define the Fourier pair $g(\omega)$ as 
\begin{equation*}
	g(\omega)=\hat{\mathcal{F}}[f]
\end{equation*}
In our case, the Fourier transform, transforms a wavefunction from the time space to the frequency space. As we should know from mathematics, the Fourier operator is invertible, and the frequency-time pair can be defined in integral form as follows
\begin{equation*}
	\begin{aligned}
		f(t)&=\hat{\mathcal{F}}^{-1}[g]=\frac{1}{\sqrt{2\pi}}\int_{\R}^{}g(\omega)e^{-i\omega t}\dd^{}{\omega}\\
		g(\omega)&=\hat{\mathcal{F}}[f]=\frac{1}{\sqrt{2\pi}}\int_{\R}^{}f(t)e^{i\omega t}\dd^{}{t}
	\end{aligned}
\end{equation*}
Getting back to our analysis, consider $f(t)$ being the wavefunction of some finite wave train with coherence time $\tau_0$. By definition then, we have
\begin{equation}
	f(t)=\begin{dcases}
		e^{-i\omega_0 t}&-\frac{\tau_0}{2}<t<\frac{\tau_0}{2}\\
		0&\abs{t}>\frac{\tau_0}{2}
	\end{dcases}
	\label{eq:wavetrainfourier}
\end{equation}
Its Fourier transform is
\begin{equation}
	g(\omega)=\mathcal{F}[f]=\frac{1}{\sqrt{2\pi}}\int_{-\frac{\tau_0}{2}}^{\frac{\tau_0}{2}}e^{i(\omega-\omega_0)t}\dd^{}{t}=\frac{1}{\sqrt{2\pi}}\left[ \frac{e^{i(\omega-\omega_0)t}}{i(\omega-\omega_0)} \right]^{\frac{\tau_0}{2}}_{\frac{\tau_0}{2}}
	\label{eq:gomega1}
\end{equation}
Evaluating the integral and remembering the complex exponential formula for $\sin(z)$, we have
\begin{equation}
	\fopr{f}=\frac{1}{i\left( \omega-\omega_0 \right)\sqrt{2\pi}}\left( e^{\frac{i(\omega-\omega_0)\tau_0}{2}}-e^{-\frac{i(\omega-\omega_0)\tau_0}{2}} \right)=\frac{1}{\omega-\omega_0}\sqrt{\frac{2}{\pi}}\sin\left( \frac{(\omega-\omega_0)\tau_0}{2} \right)
	\label{eq:foprwavetrain}
\end{equation}
Or, using for convenience the $\sinc$ function we have
\begin{equation*}
	\fopr{f}=\frac{1}{\omega_0-\omega}\frac{2}{\sqrt{\pi}}\sinc\left( \frac{(\omega-\omega_0)\tau_0}{2} \right)
\end{equation*}
\begin{dfn}[Power Spectrum]
	The utility of defining the Fourier transform of a train waves comes in handy for defining a new entity, the so called \emph{Power spectrum} of the wave, commonly indicated as $G(\omega)$, defined as the square modulus of the Fourier transform, i.e.
	\begin{equation}
		G(\omega)=\abs{\fopr{f}}^2=\abs{g(\omega)}^2
		\label{eq:powerspectrum}
	\end{equation}
\end{dfn}
Having defined the power spectrum of a wave, we have for our generic wave train then
\begin{equation}
	G(\omega)=\frac{2}{\pi(\omega-\omega_0)^2}\sin^2\left( \frac{\omega-\omega_0}{2}\tau_0 \right)
	\label{eq:powerspectrumwwavetrain}
\end{equation}
The utility of defining the power spectrum comes from its mathematical properties, in fact we have $G(\omega)=0$ if and only if $\omega=\omega_0$, and its maximum is, when
\begin{equation}
	\max\left( G(\omega) \right)=G\left( \omega_M \right)\qquad\omega_M=\frac{\pi}{2\tau_0}\pm\omega_0
	\label{eq:maximum}
\end{equation}
Therefore, the width of the frequency distribution, $\Delta\omega$ is then
\begin{equation}
	\Delta\omega=\frac{2\pi}{\tau_0}
	\label{eq:frequencywidthwavetrain}
\end{equation}
From this, in terms of frequency, we have that 
\begin{equation*}
	\Delta\nu=\frac{1}{t_0}
\end{equation*}
I.e., a sequence of wave trains lasting $\tau_0$ will have the same exact power spectrum of the single pulse. In general, if the pulses also have random $\tau_0$, we can say without loss of generality that it's approximately equal to $\expval{\tau_0}^{-1}$.\\
Reasoning in the opposite direction, given a spectral source with line width $\Delta\nu$, the coherence time of the wave trains $\expval{\tau_0}$ can be estimated as follows
\begin{equation*}
	\expval{\tau_0}=\frac{1}{\Delta\nu}
\end{equation*}
Which, in terms of coherence length $l_c$ it gives us
\begin{equation*}
	l_c=\frac{c}{\Delta\nu}
\end{equation*}
Using also that $\Delta\nu/\nu=\abs{\Delta\lambda}/\lambda$, we get that
\begin{equation}
	l_c=\frac{\lambda^2}{\Delta\lambda}
	\label{eq:coherencelengthwavetrain}
\end{equation}
With $\Delta\lambda$ being the width of the spectral line in terms of wavelength.
\section{Spatial Coherence}
In this section we will analyze a different case. Before we dealt only with two fields that reach a single fixed point in space, from here on we will study the coherence of two or more fields reaching two or more \textit{different} points in space.\\
Suppose again a quasimonochromatic point-like source $S$ and 3 receiving points $P_i$, with $P_3$ being on the same director from the point $P_1$ and $P_2$ disposed in a different director from the source at the same distance $d$ from it.\\
The coherence for the fields $E_1,E_3$ will be known as the \emph{Longitudinal Spatial Coherence}, while the coherence between $E_1,E_2$ will be known as the \textit{Transverse Spatial Coherence}. It's already clear for what we said before that the longitudinal spatial coherence will depend only on how far is $P_1$ from $P_3$, called this distance $d_{13}$ we can also say that it depends on the value
\begin{equation*}
	t_{13}=\frac{d_{13}}{c}
\end{equation*}
Compared to the coherence time $\tau_0$. It's clear that for whatever $E_1(t)$ we have, $E_3(t)$ will have the same time dependence of $E_1$, with a retardation of $t_{13}$.\\
If $t_{13}<<\tau_0$ the two fields will be coherent, while for $t_{13}>>\tau_0$ the two fields will be little or completely not coherent between the two.\\
For the fields $E_1$ and $E_2$ instead the time dependence will be the same, hence they will always be completely mutually coherent ($t_{12}=0$), but only if the source $S$ is point-like. It's clear that in reality there are no real point sources, and for an extended source (i.e., with spatial resolution) the size of the source itself must also be accounted for.\\
In general these extended sources can be considered as a cluster of point sources, and therefore we can immediately consider the more general case of multiple sources (two for ease of calculation) with two measuring points $P_1,P_2$, distant $d_{1a},d_{1b},d_{2a},d_{2b}$ from the sources, in a regime of transverse spatial coherence.\\
Called these two sources $S_a,S_b$, we can define the two fields $E_1,E_2$ as the sum of the two fields going from the $i-$th source to the $j-$th point. I.e.
\begin{equation}
	\begin{aligned}
		E_1&= E_{1a}+E_{1b}\\
		E_2&= E_{2a}+E_{2b}
	\end{aligned}
	\label{eq:e1e2spatialcoherence}
\end{equation}
Noting that the fields emitted from the source $S_a$ \textit{will not} be coherent (actually, will be completely incoherent) with the fields emitted from the source $S_b$, the correlation function between the two fields at the two receiving points will be
\begin{equation}
	\gamma_{12}(\tau)=\frac{\expval{E_1(t)\cc{E_2}(t+\tau)}}{\sqrt{I_1I_2}}=\frac{1}{\sqrt{I_1I_2}}\left( \expval{E_{1a}(t)\cc{E_{2a}}(t+\tau)}+\expval{E_{1b}(t)\cc{E_{2b}}(t+\tau)} \right)
\end{equation}
If the two fields are generic train waves, we already know that the self coherence of such pulse is
\begin{equation*}
	\gamma(\tau)=\left( 1-\frac{\tau}{\tau_0} \right)e^{i\omega\tau}
\end{equation*}
And considered the two different optical paths with different travel times for the waves, then, in our case
\begin{equation*}
	\gamma_{12}(\tau)=\frac{1}{2}\gamma(\tau_a)+\frac{1}{2}\gamma(\tau_b)
\end{equation*}
Where, obviously, we have
\begin{equation*}
	\begin{aligned}
		\tau_a&= \frac{d_{2a}-d_{1a}}{c}+\tau\\
		\tau_b&= \frac{d_{2b}-d_{1b}}{c}+\tau
	\end{aligned}
\end{equation*}
After some algebra which I won't do so bear with me here, we have
\begin{equation}
	\abs{\gamma_{12}(\tau)}^2=\left( 1-\frac{\tau_a}{\tau_0} \right)\left( 1-\frac{\tau_b}{\tau_0} \right)\left( \frac{1+\cos\left[ \omega\left( \tau_b-\tau_a \right) \right]}{2} \right)
	\label{eq:absvalautocorrspatcoh}
\end{equation}
Where the approximation $\tau_a-\tau_b<<\tau_a,\tau_b$ has been used.\\
It's clear then that the visibility and coherence of waves in such system also depends on the difference of travel times for the waves, from the sources to the points, also in a periodic manner.\\
This clearly means that at the two points the waves will show periodically interference (i.e., coherence, $\gamma_{12}\ne0$) also if the two sources emit electromagnetic radiation which in which is completely incoherent. This is called a \textit{periodic spatial dependence}.\\
Consider now the special case of having the two points disposed in a symmetrical manner with respect to the sources, hence with $d_{1a}=d_{1b}$. Then
\begin{equation}
	\tau_{b}-\tau_a=\frac{d_{2a}-d_{1b}}{c}\approx\frac{sl}{2cd}
	\label{eq:taudiffapproxspch}
\end{equation}
Where, $s$ is the distance between the sources, $l$ is the distances between the points, $d$ is the mean distance between the points and the sources. Here we used the approximation $r>>s,l$.\\
Or, extending the calculations, we have
\begin{equation*}
	r_{2a,b}=\sqrt{\left( l\pm\frac{s}{2} \right)^2+d^2}=d\sqrt{1+\frac{1}{d^2}\left( l\pm\frac{s}{2} \right)^2}
\end{equation*}
Approximating the root using the same approximation idea we had before, we have
\begin{equation*}
	d_{2a,b}\approx d+\frac{1}{2d}\left( l^2+\frac{s^2}{4}\pm ls \right)\implies\tau_b-\tau_a\approx\frac{r}{c}+\frac{l^2}{2dc}+\frac{s^2}{8dc}\pm\frac{ls}{2dc}
\end{equation*}
The experimental setup in this case is really similar to the Young interferometer's case, and the variation of lateral coherence between the two will be described by a periodic bell shaped curve.\\
The dips of this curve, will be then for $\abs{\gamma_{12}}^2=0$. Considering the main peak at the center, these peaks will be reached at some distance $l_t$, which is approximately the length of the area of maximum coherence. Then
\begin{equation*}
	\frac{\omega l_ts}{2dc}=\pi
\end{equation*}
Using $\omega/c=k_0=2\pi/\lambda$, which implies $\omega=2\pi c/\lambda$ we have
\begin{equation}
	\frac{2\pi l_ts}{2\lambda d}=\pi\implies l_t=\frac{\lambda d}{s}
	\label{eq:transversecoherencewidth}
\end{equation}
Or, in terms of angular separation $\theta_s\approx s/d$
\begin{equation}
	l_t=\frac{\lambda}{\theta_s}
	\label{eq:tcw2}
\end{equation}
The value $l_t$ is known as the \emph{transverse coherence width}.
%\subsection{Van Cittert-Zernike Theorem}
\section{Multiple Beam Interference}
The most general treatment of coherence comes from the study of multi-beam interference. For the treatment of coherence here, the division of amplitude is needed, and one way of obtaining this is by using multiple reflection between two semi-reflecting parallel plates.\\
Set up the experiment, with the two parallel plates fixed at some distance $d$ between each other, we send the initial beam with amplitude $E_0$ towards them. Due to their being semi-reflecting, at contact with the first one the beam will be reflected at some angle $\theta$ and amplitude $E_0r$, while the second will travel through it with amplitude $E_0t$.\\
The process continues when the beam hits the second semi-reflecting surface, which will again repeat the same process, sending towards the first surface a beam with amplitude $E_0tr$ and transmitting a beam with amplitude $E_0t^2$.\\
Studying the process for $n$ iterations, we get the following succession of amplitudes for the beams found inside the two surfaces:
\begin{equation*}
	E_0t,E_0tr,E_0tr^2,\cdots,E_0tr^n
\end{equation*}
While, the transmitted ones \textit{outside} the second surface, will follow the following succession
\begin{equation*}
	E_0t^2,E_0t^2r^2,E_0t^2r^4,\cdots,E_0t^2r^{2n}
\end{equation*}
Each of the transmission will have an added phase change to the wave depending on the added optical path of the wave. In our case it comes from the $2n$-th reflection ($n\in\N,\ n>1$), since we want to see the actual phase displacement of the outgoing final beam, which we will then measure.\\
Hence, supposing that between the two plates $n=n_2/n_1=1$ we have
\begin{equation}
	\delta=2kd\cos\theta=\frac{4\pi}{\lambda}d\cos\theta
	\label{eq:phasediff}
\end{equation}
Or, in general, if $n\ne1$
\begin{equation}
	\delta=2k_0dn\cos\theta=\frac{4\pi n}{\lambda_0}d\cos\theta
	\label{eq:phasediffnnot1}
\end{equation}
Then, accounting the phase shift, the final total amplitude of the beam will be the superposition (i.e. the sum) of the amplitudes of every beam that gets transmitted post multiple reflection, with a multiplied phase shift, i.e.
\begin{equation*}
	E_T=E_0t^2+E_0t^2r^2e^{i\delta}+E_0t^2r^4e^{2i\delta}+\cdots
\end{equation*}
At the limit of infinite reflection, or, in practical terms approximating for many reflections, we have a geometric series for the final result of the amplitude, which is easily calculable with the usual mathematical result
\begin{equation}
	E_T=E_0t^2\sum_{j=0}^{\infty}r^{2j}e^{ij\delta}=\frac{E_0t^2}{1-r^2e^{i\delta}}
	\label{eq:totalamplitudembi}
\end{equation}
In terms of intensity $I_T=\abs{E_T}^2$, and remembering that in general $r,t\in\Cf$, we have
\begin{equation}
	I_T=\abs{E_T}^2=\frac{I_0\abs{t}^4}{\abs{1-r^2e^{i\delta}}^2}
	\label{eq:intmbibefore}
\end{equation}
Called $\delta_r/2$ the phase shift for a single reflection (the half unity factor is just for ease of calculation later), we have that 
\begin{equation*}
	r=\abs{r}e^{i\frac{\delta_r}{2}}
\end{equation*}
And, writing $\Delta=\delta+\delta_r$ (the half factor here gets elided since we're considering only pairs of reflections), we have in terms of reflectance $R$ and transmittance $T$
\begin{equation*}
	I_T=I_0\frac{T^2}{\abs{1-Re^{i\Delta}}^2}
\end{equation*}
Calculating the absolute value in the denominator as $z\cc{z}=\abs{z}^2$ we have, after some simple algebra
\begin{equation*}
	\abs{1-Re^{i\Delta}}^2=1-R\left( e^{i\Delta}+e^{-i\Delta} \right)+R^2=1-2R\cos\left( \Delta \right)+R^2
\end{equation*}
Rewriting the cosine as one minus sine squared of half the argument, we have
\begin{equation*}
	\abs{1-Re^{i\Delta}}=1-2R+4R\sin^2\left( \frac{\Delta}{2} \right)+R^2=\left( 1-R \right)^2\left( 1+\frac{4R}{(1-R)^2}\sin^2\left( \frac{\Delta}{2} \right) \right)
\end{equation*}
Defined the \emph{coefficient of Finesse} $F$ as 
\begin{equation}
	F=\frac{4R}{(1-R)^2}
	\label{eq:finessecoefficient}
\end{equation}
We have that finally the total intensity of the different beams is 
\begin{equation}
	I_T=\frac{I_0T^2}{(1-R)^2}\frac{1}{1+F\sin^2\left( \frac{\Delta}{2} \right)}
	\label{eq:intensityairymbi}
\end{equation}
The second factor on the product is known as Airy's function, which will define (up to an intensity coefficient) the interference lines observed from the convergence of all these beams. The physical meaning of the finesse coefficient then can be seen as a value defining the sharpness of the fringes. Note that, also, if $\Delta/2$ is a whole multiple of $\pi$, Airy's function will always be equal to one for all values of $F$. Keep this in mind since it will be an useful concept for when we'll treat the Fabry-Pérot interferometer and the concept of \emph{free spectral range} of a Fabry-Pérot interferometer (or etalon, depending on the experimental configuration).\\
As it's clear from the shape of the function and the previous observation, we have that the condition for having the maximum of Airy's function is then
\begin{equation*}
	\frac{\Delta_{max}}{2}=N\pi
\end{equation*}
Where $N\in\N$ is the \emph{order of interference}, i.e. it's an indicator of the path difference between two successive beams. In fact, from the definition of $\Delta$ we have
\begin{equation}
	\Delta_{max}=2N\pi=\frac{4\pi}{\lambda_0}nd\cos\theta+\delta_r
	\label{eq:maxdeltafabryperot}
\end{equation}
In the most general case, for which the reflecting properties are not equal, we they'll contribute with two different reflection coefficients $r_1,r_2\in\Cf$, which will contribute each with a phase shift $\delta_1,\delta_2$, i.e.
\begin{equation}
	\begin{aligned}
		r_1&= \abs{r_1}e^{i\delta_1}\\
		r_2&= \abs{r_2}e^{i\delta_2}
	\end{aligned}
\end{equation}
All the previous derivations hold if and only if we define the transmittance and reflectance as follows
\begin{equation}
	\begin{aligned}
		T&= \abs{t_1}\abs{t_2}=\sqrt{T_1T_2}\\
		R&= \abs{r_1}\abs{r_2}=\sqrt{R_1R_2}
	\end{aligned}
\end{equation}
Going back to our study of the intensity function for multi-beam interference, defined $\mathcal{I}=I_T/I_0$ then we have that the maximum and minimum values for intensity will be the following
\begin{equation}
	\begin{aligned}
		\mathcal{I}_{max}&= \frac{T^2}{(1-R)^2}\\
		\mathcal{I}_{min}&= \frac{T^2}{(1+R)^2}
	\end{aligned}
	\label{eq:maxminweighpow}
\end{equation}
Which, in the realistic case of absorption of energy, defined an absorption function $A$ for which
\begin{equation*}
	A+T+R=1
\end{equation*}
We can rewrite the maximum of $\mathcal{I}$ in terms of reflectance and absorption as
\begin{equation*}
	\mathcal{I}_{max}=\frac{(1-A-R)^2}{(1-R)^2}
\end{equation*}
Where we used the simple substitution $T=1-A-R$
\section{Fabry-Pérot Instruments}
An instruments which uses the exact concept that we defined in the previous section is the \emph{Fabry-Pérot Interferometer} and the \emph{Fabry-Pérot Etalon}, invented by C.Fabry and A.Pérot in 1889.\\
The instrument is composed, excluding the source, by one collimating lens which redirects the electromagnetic waves towards two semi-reflecting plates, and a focusing lens which focuses all the transmitted waves towards a single point, in which we would usually put a photomultiplier, an amplifier and then a detector or recorder.\\
The main difference between the Etalon configuration and the Interferometer configuration of a Fabry-Pérot instrument is whether the semi reflecting plates are movable or fixed. The Etalon configuration has fixed plates, while an interferometer or \textit{scanning Fabry-Pérot} has moving plates.\\
These plates are usually made of glass or quartz, and the reflecting surfaces are parallel and \textit{as smooth as possible}. The flatness required for having a working Fabry-Pérot is from at least $\lambda/20$, up to $\lambda/100$, with $\lambda$ being the wavelength of the studied beam.\\
Commonly, scanning Fabry-Pérot are used with point sources, and the transmitted electromagnetic waves are focused to a pinhole, while the etalon is used with broad sources, the final beams here get focused to a single point on the focal plane.\\
The result of the interference created in a Fabry-Pérot instrument are circular interference fringes. Each ring corresponds to values of constant $\theta$ (remember that $\delta=\delta(\theta)$), and they're also known as \textit{fringes of equal inclination}.\\
We continue by defining a fundamental concept in Fabry-Pérot instruments, the \textit{free spectral range}.
\begin{dfn}[Free Spectral Range]
	The \emph{free spectral range} of a Fabry-Pérot is defined as the phase separation between two adjacent orders of interference $N$, as for $N,N+1$. I.e., the phase values $\Delta$ inside this range satisfy
	\begin{equation}
		\Delta_{N+1}-\Delta_N=2\pi
		\label{eq:adjfreespectralrange}
	\end{equation}
	Substituting $\Delta_N=2N\pi$ and $\Delta_{N+1}=2(N+1)\pi=\Delta_N+2\pi$, and their definitions in terms of $\delta(\theta)$, we get 
	\begin{equation*}
		\frac{4\pi}{\lambda_{N+1}}nd\cos\theta-\frac{4\pi}{\lambda_N}nd\cos\theta=2\pi
	\end{equation*}
	Solving in terms of $\lambda_{N+1}^{-1}-\lambda_{N}^{-1}$ we have
	\begin{equation}
		\frac{1}{\lambda_{N+1}}-\frac{1}{\lambda_{N}}=\frac{1}{2nd\cos\theta}
		\label{eq:fsrlambda}
	\end{equation}
	Or, substituting $\omega=2\pi c/\lambda$, in terms of angular frequency
	\begin{equation}
		\omega_{N+1}-\omega_N=\frac{\pi c}{nd\cos\theta}
		\label{eq:fsromega}
	\end{equation}
	Or, also using $\omega=2\pi\nu$, in terms of frequency
	\begin{equation}
		\nu_{N+1}-\nu_N=\frac{c}{2nd\cos\theta}
		\label{eq:fsrnu}
	\end{equation}
	Or, again, using $k_0=\omega/c=2\pi/\lambda=2\pi\nu/c$, in terms of wavenumber
	\begin{equation}
		k_{N+1}-k_{N}=\frac{\pi}{nd\cos\theta}
		\label{eq:fsrk}
	\end{equation}
\end{dfn}
\subsection{Resolution of Fabry-Pérot Instruments}
We will now treat the resolution of interference fringes with Fabry-Pérot instruments.\\
For simplicity in calculation we will consider a non-monochromatic wave for which the spectrum is composed by two exact frequencies $\omega,\omega_1$, and this spectrum will be studied with a Fabry-Pérot.\\
Each frequency will contribute to the total intensity, therefore the fringe intensity pattern will be given by the sum of the two. Assuming that the initial intensity $I_0$ is the same for both, we get the total intensity as the sum of the two Airy functions
\begin{equation}
	I_T=\frac{I_0}{1+F\sin^2\left( \frac{\Delta}{2} \right)}+\frac{I_0}{1+F\sin^2\left( \frac{\Delta_1}{2} \right)}
	\label{eq:intensityrfpi}
\end{equation}
As usual, the phase $\Delta$ is
\begin{equation*}
	\Delta=\delta(\theta)+\delta_r=\frac{4\pi n}{\lambda_0}d\cos\theta+\delta_r
\end{equation*}
Which, if we assume an almost-normal incidence of the electromagnetic wave to the plates ($\theta<<1$), we have at first order and in terms of frequencies
\begin{equation*}
	\begin{aligned}
		\Delta&= \frac{2\omega d}{c}+\delta_r+\order{\theta^3}\\
		\Delta_1&= \frac{2\omega_1d}{c}+\delta_r+\order{\theta^3}
	\end{aligned}
\end{equation*}
And here comes in our help what's known as Taylor's criterion, which states that two \textit{equal} lines are resolved if and only if the individual curves cross at the point of half intensity of each, for which then the intensity at the saddle is equal to exactly twice the initial intensity.\\
Since we're exactly smack in the middle between the two single peaks, considered the range between the two $\Delta-\Delta_1$, the total intensity at this saddle will be the usual Airy function times $2I_0$ this time, considered at half range, i.e. at $(\Delta-\Delta_1)/2$, i.e.
\begin{equation}
	\left. I_T \right|_{\frac{\Delta-\Delta_1}{2}}=\frac{2I_0}{1+F\sin^2\left( \frac{\Delta-\Delta_1}{4} \right)}=I_0
	\label{eq:saddleintfpir}
\end{equation}
Solving for $\Delta-\Delta_1$, we're left with
\begin{equation*}
	F\sin^2\left( \frac{\Delta-\Delta_1}{4} \right)=1
\end{equation*}
Assumed (we hope so for a good resolution!) $\Delta-\Delta_1<<1$, we can approximate to the first order the sine function, getting
\begin{equation}
	F\left( \left(\frac{\Delta-\Delta_1}{4}\right)^2+\order{ (\Delta-\Delta_1)^6} \right)=1\implies\abs{\Delta-\Delta_1}\approx\frac{4}{\sqrt{F}}=2\frac{1-R}{\sqrt{R}}
	\label{eq:rayleighcritfpi}
\end{equation}
I.e., the minimum resolvable range depends directly to the inverse of the square root of the finesse coefficient, which is only determined by the physical characteristics of the instrument!\\
In terms of frequencies, remembering that $\Delta\approx\frac{2d}{c}\omega$, we have that the smallest resolvable interval of frequencies is then
\begin{equation}
	\abs{\omega-\omega_1}\approx\frac{2c}{d\sqrt{F}}=\frac{c}{d}\frac{1-R}{\sqrt{R}}
	\label{eq:minresolvableintfreq}
\end{equation}
Therefore, the smallest resolvable interval of frequencies for a Fabry-Pérot instrument depends only on the reflectance of the plates $R$ and their distance $d$.\\
We also give a new definition for two elements often used when treating Fabry-Pérot instruments
\begin{dfn}[Reflecting Finesse]
	Given a Fabry-Pérot instrument, the \emph{reflecting finesse} $\mathcal{F}$, is defined as the free spectral range divided by the smallest resolvable interval, i.e.
	\begin{equation}
		\mathcal{F}=\frac{\Delta_{N+1}-\Delta_N}{\abs{\Delta-\Delta_1}}=\frac{\pi}{2}\sqrt{F}=\frac{\pi}{2}\frac{\sqrt{R}}{1-R}
		\label{eq:reflectingfinesse}
	\end{equation}
\end{dfn}
\begin{dfn}[Resolving Power]
	The \emph{resolving power} $RP$ of a Fabry-Pérot instrument is defined as the inverse of the minimum resolvable interval times the scanned frequency (or angular frequency, or also wavelength), i.e.
	\begin{equation}
		RP=\frac{\omega}{\abs{\omega-\omega_1}}=\frac{\nu}{\abs{\nu-\nu_1}}=\frac{\lambda}{\abs{\lambda-\lambda_1}}
		\label{eq:resolvingpowerfpi}
	\end{equation}
	It's clear, by direct substitution and some algebra, that the resolving power is directly tied to the reflecting finesse as
	\begin{equation}
		RP=N\mathcal{F}=N\pi\frac{\sqrt{R}}{1-R}
		\label{eq:rpfcrossover}
	\end{equation}
	I.e., it's directly proportional to the reflecting finesse and the order of interference. Clearly, the resolving power can be made arbitrarily big simply by choosing higher interference orders. Note that by definition of N, this can be accomplished by increasing the mirror separation, which unfortunately reduces the free spectral range of the instrument.\\
	Another idea for increasing the $RP$ is by increasing $\mathcal{F}$, bringing the reflectance closer and closer to $1$. There's a clear limit to this, in fact it's physically limited by the absorption of the used material, which reduces the intensity of the transmitted fringes.
\end{dfn}
\end{document}
