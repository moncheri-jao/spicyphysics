\documentclass[../electromagnetism.tex]{subfiles}
\begin{document}
\section{Propagation of Electromagnetic Waves}
Consider a random point in spacetime, where only an electromagnetic field is present.\\
Here we have
\begin{equation*}
	\begin{aligned}
		\rho&=0\\
		J^i&=0
	\end{aligned}
\end{equation*}
Using $H^i$ instead of $B^i$, since, in free space
\begin{equation*}
	B^i=\mu_0H^i
\end{equation*}
We get Maxwell's equation for what's known as a free electromagnetic field
\begin{equation}
	\left\{ \begin{aligned}
		\del_iE^i&=0\\
		\cpr{i}{j}{k}\del^jE^k&=-\mu_0\pdv{H^i}{t}\\
		\del_iH^i&=0\\
		\cpr{i}{j}{k}\del^jH^k&=\epsilon_0\pdv{E^i}{t}
	\end{aligned}\right.
	\label{eq:freespacemaxwell}
\end{equation}
The absence of sources here is given by the two divergence relations. Note also that these coupled PDEs are valid both in the dynamic and static case.\\
These equations, although it might not be that clear from the system, are completely separable. Taken the two curl equations we have, using the relations found before, that
\begin{equation*}
	\left\{ \begin{aligned}
			\cpr{i}{j}{k}\cpr{k}{l}{m}\del^j\del^lE^m&= -\mu_0\epsilon_0\pdv[2]{E^i}{t}\\
			\cpr{i}{j}{k}\cpr{k}{l}{m}\del^j\del^lH^m&= -\mu_0\epsilon_0\pdv[2]{H^i}{t}
	\end{aligned}\right.
\end{equation*}
Using the Levi-Civita identity we have that the double curl becomes
\begin{equation*}
	\del^i\left( \del_jE^j \right)-\del^j\del_jE^i
\end{equation*}
Which, if substituted inside the previous system, using that $\del_i(H^i,E^i)=0$ gives back the already well known wave equation
\begin{equation}
	\begin{aligned}
		\frac{1}{c^2}\pdv[2]{E^i}{t}-\del^j\del_jE^i&=\square E^i= 0\\
		\frac{1}{c^2}\pdv[2]{H^i}{t}-\del^j\del_jH^i&=\square H^i= 0
	\end{aligned}
	\label{eq:emwaveeq}
\end{equation}
The solution will be what's known as an \textit{electromagnetic wave}, a wave moving in space composed by both an electric and magnetic field. From the wave equation we can already say that it's moving at speed $c$, as it should. To be precise it's what we know as \emph{light}.
\subsection{Electromagnetic Waves in Dielectric Media}
In case that we're dealing with the movement of this wave in dielectric or magnetic media, we must remember that the fields \emph{will be different}. The equations will be the same in shape, but will need some tweaking.\\
The tweaking is not actually in the fields, but in the constants. Given the speed of an electromagnetic wave in the vacuum is $c$, where
\begin{equation*}
	c=\frac{1}{\sqrt{\mu_0\epsilon_0}}
\end{equation*}
In a media, we will have the \textit{speed of propagation} of the waves, $u$, given by the substitution of $\mu_0\epsilon_0$ with $\mu\epsilon$ (remember that $\mu=\mu_r\mu_0$ and $\epsilon_r\epsilon_0$)
\begin{equation}
	u=\frac{1}{\sqrt{\mu\epsilon}}=\frac{1}{\sqrt{\mu_r\mu\epsilon_r\mu_0}}=\frac{c}{\sqrt{\mu_r\epsilon_r}}
	\label{eq:speedofprop}
\end{equation}
From this we define the \textit{index of refraction} of a medium, $n$ as
\begin{equation}
	n=\frac{c}{u}=\sqrt{\mu_r\epsilon_r}
	\label{eq:refractionindex}
\end{equation}
So that, the propagation speed can be written also
\begin{equation*}
	u=\frac{c}{n}
\end{equation*}
And the wave equations become
\begin{equation}
	\left\{ \begin{aligned}
			\frac{n^2}{c^2}\pdv[2]{E^i}{t}-\del^j\del_jE^i&= 0\\
			\frac{n^2}{c^2}\pdv[2]{H^i}{t}-\del^j\del_jH^i&= 0
	\end{aligned}\right.
	\label{eq:waveequationinmedia}
\end{equation}
\section{Polarization of Electromagnetic Waves}
<++>
%%TODO Polarization from Fowles (CH 2 VECTORIAL NATURE OF LIGHT)
\subsection{Jones Calculus}
\section{Reflection and Refraction}
\subsection{External and Internal Reflection}
\subsection{Total Reflection}
\subsection{Phase Changes}
\end{document}
