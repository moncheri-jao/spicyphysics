\documentclass[../electromagnetism.tex]{subfiles}
\begin{document}
\section{Propagation of Electromagnetic Waves}
Consider a random point in spacetime, where only an electromagnetic field is present.\\
Here we have
\begin{equation*}
	\begin{aligned}
		\rho&=0\\
		J^i&=0
	\end{aligned}
\end{equation*}
Using $H^i$ instead of $B^i$, since, in free space
\begin{equation*}
	B^i=\mu_0H^i
\end{equation*}
We get Maxwell's equation for what's known as a free electromagnetic field
\begin{equation}
	\left\{ \begin{aligned}
		\del_iE^i&=0\\
		\cpr{i}{j}{k}\del^jE^k&=-\mu_0\pdv{H^i}{t}\\
		\del_iH^i&=0\\
		\cpr{i}{j}{k}\del^jH^k&=\epsilon_0\pdv{E^i}{t}
	\end{aligned}\right.
	\label{eq:freespacemaxwell}
\end{equation}
The absence of sources here is given by the two divergence relations. Note also that these coupled PDEs are valid both in the dynamic and static case.\\
These equations, although it might not be that clear from the system, are completely separable. Taken the two curl equations we have, using the relations found before, that
\begin{equation*}
	\left\{ \begin{aligned}
			\cpr{i}{j}{k}\cpr{k}{l}{m}\del^j\del^lE^m&= -\mu_0\epsilon_0\pdv[2]{E^i}{t}\\
			\cpr{i}{j}{k}\cpr{k}{l}{m}\del^j\del^lH^m&= -\mu_0\epsilon_0\pdv[2]{H^i}{t}
	\end{aligned}\right.
\end{equation*}
Using the Levi-Civita identity we have that the double curl becomes
\begin{equation*}
	\del^i\left( \del_jE^j \right)-\del^j\del_jE^i
\end{equation*}
Which, if substituted inside the previous system, using that $\del_i(H^i,E^i)=0$ gives back the already well known wave equation
\begin{equation}
	\begin{aligned}
		\frac{1}{c^2}\pdv[2]{E^i}{t}-\del^j\del_jE^i&=\square E^i= 0\\
		\frac{1}{c^2}\pdv[2]{H^i}{t}-\del^j\del_jH^i&=\square H^i= 0
	\end{aligned}
	\label{eq:emwaveeq}
\end{equation}
The solution will be what's known as an \textit{electromagnetic wave}, a wave moving in space composed by both an electric and magnetic field. From the wave equation we can already say that it's moving at speed $c$, as it should. To be precise it's what we know as \emph{light}.
\subsection{Electromagnetic Waves in Dielectric Media}
In case that we're dealing with the movement of this wave in dielectric or magnetic media, we must remember that the fields \emph{will be different}. The equations will be the same in shape, but will need some tweaking.\\
The tweaking is not actually in the fields, but in the constants. Given the speed of an electromagnetic wave in the vacuum is $c$, where
\begin{equation*}
	c=\frac{1}{\sqrt{\mu_0\epsilon_0}}
\end{equation*}
In a media, we will have the \textit{speed of propagation} of the waves, $u$, given by the substitution of $\mu_0\epsilon_0$ with $\mu\epsilon$ (remember that $\mu=\mu_r\mu_0$ and $\epsilon_r\epsilon_0$)
\begin{equation}
	u=\frac{1}{\sqrt{\mu\epsilon}}=\frac{1}{\sqrt{\mu_r\mu\epsilon_r\mu_0}}=\frac{c}{\sqrt{\mu_r\epsilon_r}}
	\label{eq:speedofprop}
\end{equation}
From this we define the \textit{index of refraction} of a medium, $n$ as
\begin{equation}
	n=\frac{c}{u}=\sqrt{\mu_r\epsilon_r}
	\label{eq:refractionindex}
\end{equation}
So that, the propagation speed can be written also
\begin{equation*}
	u=\frac{c}{n}
\end{equation*}
And the wave equations become
\begin{equation}
	\left\{ \begin{aligned}
			\frac{n^2}{c^2}\pdv[2]{E^i}{t}-\del^j\del_jE^i&= 0\\
			\frac{n^2}{c^2}\pdv[2]{H^i}{t}-\del^j\del_jH^i&= 0
	\end{aligned}\right.
	\label{eq:waveequationinmedia}
\end{equation}
In general, it's clear that both fields solve the same basic wave equation
\begin{equation*}
	\del^i\del_if=\frac{1}{u^2}\pdv[2]{f}{t}
\end{equation*}
The solution of this equation (in terms of complex exponentials) is
\begin{equation}
	f(x^i,t)=e^{i\left( k^ix_i-\omega t \right)}
	\label{eq:wavesolutionemwaves}
\end{equation}
What happens when we apply the same differential operators that appear in Maxwell's equation to this solution?\\
Well we simply take the derivatives and see that we're actually dealing with an eigenfunction of the differential operator, with eigenvalues
\begin{equation}
	\left\{ \begin{aligned}
		\del_if&= ik_if\\
		\del_tf&= -i\omega f
\end{aligned}\right.
	\label{eq:eigenwave}
\end{equation}
Thus, in terms of electromagnetic fields we can write Maxwell's equations as follows
\begin{equation}
	\left\{ \begin{aligned}
			k^iE_i&= 0\\
			\cpr{i}{j}{k}k^jE^k&= \mu\omega H^i\\
			k^iB_i&= 0\\
			\cpr{i}{j}{k}k^jB^k&= -\epsilon\omega E^i
	\end{aligned}\right.
	\label{eq:eigenmaxwell}
\end{equation}
It's clear that these three vectors compose a mutually orthogonal triad. Therefore, if we consider the magnitudes of these vectors from the third equation, we get
\begin{equation}
	H=\frac{\epsilon\omega}{k}E=\epsilon uE
	\label{eq:HErelwaves}
\end{equation}
Where we used $\omega/k=u$. If we rewrite them in terms of the refraction index $n=c/u$, defined $Z_0=\sqrt{\mu_0/\epsilon_0}$ the ``free space impedance'' we also have
\begin{equation*}
	H=\frac{n}{Z_0}E
\end{equation*}
\subsection{Energy Flow}
As we saw before, the Poynting vector is defined as the cross product of $E$ with $H$, and is a representation of the flux of electromagnetic energy per unit area. If we take an electromagnetic field which solves the wave equation, as 
\begin{equation}
	\begin{aligned}
		E^i&= E_0^i\cos\left( k^ix_i-\omega t \right)\\
		H^i&= H_0^i\cos\left( k^ix_i-\omega t \right)
	\end{aligned}
	\label{eq:planewaveem}
\end{equation}
We have that the Poynting vector for such fields is
\begin{equation*}
	S^i=\cpr{i}{j}{k}E^jH^k=S^i_0\cos^2\left( k^ix_i-\omega t \right)
\end{equation*}
Where $S_0=E_0\times H_0$. Since, as we saw before $k^i$ is perpendicular to both $E$ and $H$, it must be parallel to the Poynting vector. Taken the average (remember that $\expval{\cos^2(\theta)}=1/2$) we get
\begin{equation}
	\expval{S^i}=\frac{1}{2}S^i_0=I\frac{k^i}{k}=I\hat{n}
	\label{eq:poyntingvec}
\end{equation}
Where $I$ is known as the \textit{irradiance} and has, clearly, value
\begin{equation*}
	I=\frac{1}{2}E_0H_0=\frac{n}{2Z_0}E_0^2
\end{equation*}
The irradiance, defined as before, is nothing more than the rate of flow of energy, and it's proportional to the square of the amplitude of the electric field.\\
For isotropic media, then the direction of the energy flux is defined by both $S^i$ and $k^i$.
\section{Polarization of Electromagnetic Waves}
\subsection{Linear Polarization}
\begin{dfn}[Linear Polarization]
	Consider a general plane harmonic wave with the following solution to Maxwell's equations:
	\begin{equation*}
		\begin{aligned}
			E^i&=E_0^ie^{i\left( k^ir_i-\omega t \right)}\\
			H^i&=H_0^ie^{i\left( k^ir_i-\omega t \right)}\\
		\end{aligned}
	\end{equation*}
	If both $E_0^i,\ H_0^i\in\R^3$ are constant (real constant vectors), then the wave is said to be \textit{linearly polarized}.
\end{dfn}
\begin{dfn}[Polarizer]
	A \textit{polarizer} is an optical element that generates linearly polarized light. One of such instrument is the Polaroid filter.\\
	A polarizer is said to have two main axes, one transmission axis and one blocking axis. The transmission axis is the one that will let the component of the $E^i$ field pass, therefore by definition polarizing the light wave. If the polarizer is \emph{completely} transparent to the incoming light parallel to the transmission axis it's known as an \textit{ideal polarizer}
\end{dfn}
Consider now some randomly polarized light that passes through an ideal linear polarizer, and suppose that it arrives such that $E^i$ arrives at an angle $\theta$ with respect to the transmission axis. The transmitted magnitude then will simply be the projection of $E^i$ onto the axis.\\
Called $\tau$ our axis, then
\begin{equation*}
	E_\tau=E\cos\theta
\end{equation*}
And since $I\propto E^2$, the transmitted intensity is
\begin{equation*}
	I_\tau=I\cos^2\theta
\end{equation*}
By the same reasoning as before, for unpolarized light then, since $\expval{\cos^2\theta}=1/2$ we have that, the transmitted intensity through this polarizer is exactly half the incoming intensity, i.e.
\begin{equation*}
	I_\tau=\frac{1}{2}I_u
\end{equation*}
\begin{dfn}[Partial Polarization]
	A light wave is said to be \textit{partially polarized} if it's made by a mixture of polarized and unpolarized light. The \textit{degree of polarization} $P$ is defined as the ratio between the intensity of polarized light $I_p$ and the total intensity $I_T$ (the sum of the intensity of polarized and unpolarized light)
	\begin{equation*}
		P=\frac{I_p}{I_T}
	\end{equation*}
	For the special case of partial \emph{linear} polarization, it can be calculate as the difference between the maximum intensity minus the minimum intensity of light, normalized with respect to the total intensity
	\begin{equation*}
		P_{pl}=\frac{I_{max}-I_{min}}{I_{max}+I_{min}}
	\end{equation*}
	It's clear that $P\in[0,1]$, where $P=0$ indicates completely unpolarized light and $P=1$ indicates completely polarized light
\end{dfn}
\subsection{Circular and Elliptical Polarization}
Consider two orthogonal waves $\psi_1,\ \psi_2$ with equal amplitudes $A_1=A_2$. Due to their orthogonality there's a $\pi/2$ phase shift between the two, and can be written as:
\begin{equation}
	\begin{aligned}
		\psi_1&=E_0\cos\left( kz-\omega t \right)\hat{x}^i
		\psi_2&=E_0\sin\left( kz-\omega t \right)\hat{y}^i
	\end{aligned}
	\label{eq:circpol2waves}
\end{equation}
Using the principle of superposition, then, the total field (i.e., total wave) is then
\begin{equation}
	E^i=E_0^i\left( \cos\left( kz-\omega t \right)\hat{x}^i+\sin\left( kz-\omega t \right)\hat{y}^i \right)
	\label{eq:totfieldcircular}
\end{equation}
This is clearly a solution to the Maxwell equation $\square E^i=0$ and therefore can be seen as a valid electromagnetic wave, especially one which rotates around the $z$ axis with angular frequency $\omega$. The direction of rotation defines if it's a \textit{right circularly polarized wave} ($-y$) or a \textit{left circularly polarized wave} ($+y$).\\
Using complex notation and noting that the $\pi/2$ shift can be defined with a multiplication by $i$ of the $y$ component, hence
\begin{equation}
	E^i=\left( \hat{x}^i\pm i\hat{y}^i \right)e^{i\left( kz-\omega t \right)}
	\label{eq:circpol}
\end{equation}
The circular polarization is just a special case of an elliptical polarization, where $A_1\ne A_2$, where
\begin{equation}
	E_0^i=E_0\hat{x}^i+iE_1\hat{y}^i\quad\implies\quad E^i=E_0^ie^{i\left( kz-\omega t \right)}
	\label{eq:complexwaveelpol}
\end{equation}
It's also clear how this notation generalizes polarization altogether, in fact, if $E_0^i$ is a real vector we get back our usual linear polarization that we treated before.\\
A question arises now, what optical element can create elliptical polarization from a general light wave? The answer is \textit{quarter-wave plates}.
\begin{dfn}[Quarter Wave Plate]
	A \textit{quarter wave plate}, also known as a $\lambda/4$-\textit{plate} is an optical element made of two different refracting transparent crystals, which combined give a polarizer with two different transmission axes.\\
	The axis with the highest refraction index ($n_1$) is known as the \textit{fast axis}, while the axist with the smallest refraction index ($n_2$), is known as the \textit{slow axis}.\\
	The instrument is built such that the optical thickness ($nd$, where $d$ is the thickness), obeys the following relation
	\begin{equation*}
		n_1d-n_2d=\frac{\lambda_0}{4}
	\end{equation*}
	With $\lambda_0$ being the vacuum wavelength of the considered wave. Solving for $d$, we then have the quarter-wave plate relation
	\begin{equation}
		d=\frac{\lambda_0}{4(n_1-n_2)}
		\label{eq:thicknessquarterwave}
	\end{equation}

\end{dfn}
\subsection{Jones Calculus}
Consider a general plane harmonic electromagnetic wave, i.e.
\begin{equation}
	E_0^i=E_{0x}\hat{x}^i+E_{0y}\hat{y}^i
	\label{eq:generalplanewavejones}
\end{equation}
Since in the most general case $E_{0x},E_{0y}\in\Cf$ we might choose to use the $\hat{x}^i,\hat{y}^i$ basis and work directly in $\Cf^2$. We can define then the \textit{Jones Vector} as follows
\begin{dfn}[Jones Vector]
	We define a \textit{Jones vector} as a vector $v^i\in\Cf^2$ defined as follows:
	\begin{equation}
		\begin{pmatrix}
			E_{0x}\\
			E_{0y}
		\end{pmatrix}=\begin{pmatrix}
			\abs{E_{0x}}e^{i\phi_x}\\
			\abs{E_{0y}}e^{i\phi_y}
		\end{pmatrix}
		\label{eq:jonesvector}
	\end{equation}
	Then, this vector is the most general way of defining a plane harmonic electromagnetic wave. As we have seen before, then we have
	\begin{enumerate}
	\item For linearly polarized waves in one direction (x or y):
		\begin{equation}
			A\begin{pmatrix}
				1\\0
			\end{pmatrix}\qquad A\begin{pmatrix}
				0\\1
			\end{pmatrix}
			\label{eq:linearpoljones}
		\end{equation}
	\item For linearly polarized waves (45 degrees):
		\begin{equation}
			A\begin{pmatrix}
				1\\1				
			\end{pmatrix}
			\label{eq:45poljones}
		\end{equation}
	\item For left-circularly polarized waves:
		\begin{equation}
			A\begin{pmatrix}
				1\\i
			\end{pmatrix}
			\label{eq:lcpjones}
		\end{equation}
	\item For right-circularly polarized waves:
		\begin{equation}
			A\begin{pmatrix}
				1\\-i
			\end{pmatrix}
			\label{eq:rcpjones}
		\end{equation}
	\end{enumerate}
\end{dfn}
This representation in terms of complex vectors is really useful for calculating the polarization (and amplitude) of the wave resulting from the superposition of two differently polarized waves.
I.e.: suppose you have a RCP (Right Circularly Polarized) wave and a LCP (Left Circularly Polarized) wave with unitary amplitude in some units. After the superposition we get:
\begin{equation*}
	\begin{pmatrix}
		1\\-i
	\end{pmatrix}+\begin{pmatrix}
		1\\i
	\end{pmatrix}=2\begin{pmatrix}
		1\\0
	\end{pmatrix}
\end{equation*}
Then, RCP+LCP=x-Linear polarization.\\
The action of optical elements can then be seen as the action of a matrix upon these vectors. Then, as an example, we can represent a linear polarizer on the x direction as:
\begin{equation}
	\begin{pmatrix}
		1&0\\0&0
	\end{pmatrix}
	\label{eq:xpolarizerjones}
\end{equation}
And so on for other optical elements.\\
In general, we have:
\begin{enumerate}
\item Linear Polarizers
	\begin{itemize}
	\item Horizontal polarizers
		\begin{equation}
			\begin{pmatrix}
				1&0\\0&0
			\end{pmatrix}
			\label{eq:horizontalpolarjones}
		\end{equation}
	\item Vertical polarizers
		\begin{equation}
			\begin{pmatrix}
				0&0\\0&1
			\end{pmatrix}
			\label{eq:verticalpolarjones}
		\end{equation}
	\item $\pm45^{\circ}$ polarizers
		\begin{equation}
			\frac{1}{2}\begin{pmatrix}
				1&\pm1\\\pm1&1
			\end{pmatrix}
			\label{eq:45polarjones}
		\end{equation}
	\end{itemize}
\item $\lambda/4$ Plates
	\begin{itemize}
	\item Vertical fast axis
		\begin{equation}
			\begin{pmatrix}
				1&0\\0&i
			\end{pmatrix}
			\label{eq:fastyjones}
		\end{equation}
	\item Horizontal fast axis
		\begin{equation}
			\begin{pmatrix}
				1&0\\0&-i
			\end{pmatrix}
			\label{eq:fastxjones}
		\end{equation}
	\item $\pm45^\circ$ fast axis
		\begin{equation}
			\frac{1}{\sqrt{2}}\begin{pmatrix}
				1&\pm i\\\pm i&1
			\end{pmatrix}
			\label{eq:fast45jones}
		\end{equation}
	\end{itemize}
\item $\lambda/2$ Plates
	\begin{itemize}
	\item Horizontal or vertical fast axis
		\begin{equation}
			\begin{pmatrix}
				1&0\\0&-1
			\end{pmatrix}
			\label{eq:lambda2jones}
		\end{equation}
	\end{itemize}
\item Retarders
	\begin{equation}
		\begin{pmatrix}
			e^{i\phi}&0\\0&e^{i\phi}
		\end{pmatrix}
		\label{eq:retarderjones}
	\end{equation}
\item Phase Changers
	\begin{equation}
		\begin{pmatrix}
			e^{i\phi_x}&0\\
			0&e^{i\phi_y}
		\end{pmatrix}
		\label{eq:phasechangerjones}
	\end{equation}
\item Circular Polarizers
	\begin{itemize}
	\item Left Circular Polarizer
		\begin{equation}
			\frac{1}{2}\begin{pmatrix}
				1&-i\\i&1
			\end{pmatrix}
			\label{eq:lcircularpolarjones}
		\end{equation}
	\item Right Circular Polarizer
		\begin{equation}
			\frac{1}{2}\begin{pmatrix}
				1&i\\-i&1
			\end{pmatrix}
			\label{eq:rcircularpoljones}
		\end{equation}
	\end{itemize}
\end{enumerate}
In general, for a train of optical devices $(A^i_j)_1,\cdots,(A^i_j)_n$, where $A_i$ are complex matrices, we have that the resulting wave $R^i$ will be simply the following product, given $I^i$ as our incident wave:
\begin{equation}
	\prod_{\alpha=1}^n(A^i_j)_\alpha I^j=R^i
	\label{eq:multipleoptdevjones}
\end{equation}
\subsection{Orthogonal Polarization}
Given two waves $E_1^i,E_2^i\in\Cf^2$, they're said to be \textit{orthogonally polarized} if, the complex scalar product between the two is null.\\
I.e.
\begin{equation}
	\langle E_1^i, E_2^j \rangle = E_1^i\cc{E}^2_i=0
	\label{eq:orthpol}
\end{equation}
%%TODO diomerda che cazzo ci scrivo qui? Bastardo il signore 
%
\section{Reflection and Refraction}
\subsection{Snell's Law}
<++>
\subsection{Fresnel Equations}
\section{Total Internal Reflection}
\subsection{External and Internal Reflection}
\subsection{Brewster Angle}
\subsection{Phase Changes}
\subsubsection{Fresnel Rhomb}
\section{Reflection Matrix}
\end{document}
