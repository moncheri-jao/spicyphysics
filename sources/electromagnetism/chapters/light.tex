\documentclass[../electromagnetism.tex]{subfiles}
\begin{document}
\section{The Wave Equation}
Maxwell's equations are of utmost importance in studying the behavior of electromagnetic field, due to their versatility and generality. It's due to Maxwell himself that we now treat light as electromagnetic radiation, particularly, electromagnetic waves.
The reasoning behind Maxwell's work comes \textit{directly} from his equation. Consider a location of space in which there are no charges nor currents, i.e. no \textit{sources}. For such system, Maxwell's equations are\footnote{From now on, here we will use the standard boldface vector notation for ease of reading. Index notation will be used in the chapter on crystals and in some other section}
\begin{equation}
	\left\{ \begin{aligned}
			\nabla\times\vec{E}&=-\pdv{\vec{B}}{t}\\
			\nabla\times\vec{B}&=\frac{1}{c^2}\pdv{\vec{E}}{t}
	\end{aligned}\right.
	\label{eq:mweq.waves}
\end{equation}
Where, the divergence equations are both equally zero
\begin{equation}
	\nabla\cdot\vec{E}=\nabla\cdot\vec{B}=0
	\label{eq:diveq.waves}
\end{equation}
It's important to remember that 
\begin{equation*}
	c^2=\frac{1}{\mu_0\epsilon_0}
\end{equation*}
Is the ``\textit{speed of light}'' for reasons that will be clear after a couple of manipulations. For reasons of symmetry of the two curl equations we use the following substitution
\begin{equation}
	\vec{B}=\mu_0\vec{H}
	\label{eq:btoh.waves}
\end{equation}
for which, the only nonzero equations are
\begin{equation}
	\left\{\begin{aligned}
		\nabla\times\vec{E}&=-\mu_0\pdv{\vec{H}}{t}\\
		\nabla\times\vec{H}&=\epsilon_0\pdv{\vec{E}}{t}
	\end{aligned}\right.
	\label{eq:nonzeromwe.waves}
\end{equation}
The main problem of these equations is that they're still coupled between eachother, and there are still the previous divergence equations, but this can be solved quickly. Remembering the following operator identity
\begin{equation*}
	\nabla\times\nabla\times[\quad]=\nabla\left(\nabla\cdot[\quad]\right)-\nabla^2[\quad]
\end{equation*}
We get, by taking the curl of both equations, and reinserting them to the right hand side
\begin{equation}
\left\{	\begin{aligned}
		\nabla\times\nabla\times\vec{E}&=-\mu_0\epsilon_0\pdv[2]{\vec{E}}{t}\\
		\nabla\times\nabla\times\vec{H}&=-\mu_0\epsilon_0\pdv[2]{\vec{H}}{t}
	\end{aligned}\right.
	\label{eq:nablanabla.waves}
\end{equation}
Or, inserting the identity and \eqref{eq:diveq.waves}
\begin{equation}
	\left\{\begin{aligned}
			\nabla^2\vec{E}&=\frac{1}{c^2}\pdv[2]{\vec{E}}{t}\\
			\nabla^2\vec{H}&=\frac{1}{c^2}\pdv[2]{\vec{H}}{t}
	\end{aligned}\right.
	\label{eq:waveeq.waves}
\end{equation}
Or, more compactly
\begin{equation}
	\left\{ \begin{aligned}
			\square\vec{E}&=0\\
			\square\vec{H}&=0
	\end{aligned}\right.
	\label{eq:square.waves}
\end{equation}
Where
\begin{equation}
	\square=\frac{1}{c^2}\pdv[2]{t}-\nabla^2
	\label{eq:dalambertian.waves}
\end{equation}
Is known as the \textit{D'Alembertian} operator. The equation we wrote is a wave equation for waves traveling at $v=c$.
\subsection{Dielectric Wave Equation}
If we consider now ourselves inside some media, we have to take account of both \textit{polarization} and \textit{magnetization}, given by the presence of atoms inside the medium. These atoms will absorb some special frequencies, which will be known as \textit{resonance frequencies}.\\
Far from the resonant frequencies, the medium is known as a transparent and non-absorbent medium. Maxwell's equations are the usual complete ones:
\begin{equation}
	\left\{ \begin{aligned}
			\nabla\cdot\vec{E}&= \frac{\rho}{\epsilon_0}\\
			\nabla\times\vec{E}&= -\pdv{\vec{B}}{t}\\
			\nabla\cdot\vec{B}&= 0\\
			\nabla\times\vec{B}&= \mu_0\vec{J}+\mu_0\epsilon_0\pdv{\vec{E}}{t}
	\end{aligned}\right.
	\label{eq:mwgeneralmedia}
\end{equation}
Consider now a real electromagnetic wave, it will be composed of multiple frequencies, denoted with $\omega$. An electromagnetic wave will be denoted as \textit{monochromatic} if and only if it's composed by a single frequency, (note that in nature there are no monochromatic waves). In this ideal case, the electric field is:
\begin{equation}
	\vec{E}(\vec{r},t)=\vec{E}_\omega(\vec{r})e^{-i\omega t}
	\label{eq:frequencydec.waves}
\end{equation}
With the exponential coming from the wave equation itself.\\
Since we are dealing with dielectrics, we gotta consider charge polarization $\vec{P}$, which will also be decomposed in frequencies, therefore, inserting the $D$ field in our calculations
\begin{equation}
	\vec{D}_\omega=\epsilon_0\vec{E}_\omega+\vec{P}_\omega=\epsilon_\omega\vec{E}_\omega
	\label{eq:dvector.waves}
\end{equation}
Where we used the following two relations
\begin{equation}
	\left\{\begin{aligned}
			\vec{P}_\omega&= \epsilon_0\chi_\omega\vec{E}_\omega\\
			\epsilon_\omega&= \epsilon_0(1+\chi_\omega)
	\end{aligned}\right.
	\label{eq:dtoe.waves}
\end{equation}
Noting that here, in general, $\epsilon_\omega$ depends on position, the first Maxwell equation for dielectrics becomes slightly more complicated
\begin{equation*}
	\nabla\cdot\vec{D}_\omega=\vec{E}_\omega\cdot\nabla\epsilon_\omega+\epsilon_\omega\nabla\cdot\vec{E}_\omega=0
\end{equation*}
Solving for $E_\omega$ we get
\begin{equation}
	\nabla\cdot\vec{E}_\omega=-\frac{\vec{E}_\omega\cdot\nabla\epsilon_\omega}{\epsilon_\omega}
	\label{eq:gaussdie.waves}
\end{equation}
Evaluating the time derivatives for the monochromatic fields we have
\begin{equation*}
	\begin{aligned}
		\pdv{\vec{E}}{t}&= -i\omega\vec{E}\\
		\pdv{\vec{H}}{t}&= -i\omega\vec{H}
	\end{aligned}
\end{equation*}
Inserting everything into Maxwell's equations we get the set of equations for monochromatic waves in general non-magnetic media
\begin{equation}
	\left\{\begin{aligned}
			\nabla\cdot\vec{E}_\omega&= -\frac{\vec{E}_\omega\cdot\nabla\epsilon_\omega}{\epsilon_\omega}\\
			\nabla\times\vec{E}_\omega&= i\omega\mu_0\vec{H}_\omega\\
			\nabla\cdot\vec{H}_\omega&= 0\\
			\nabla\times\vec{H}_\omega&= -i\omega\epsilon_\omega\vec{E}_\omega
	\end{aligned}\right.
	\label{eq:mweqdie.waves}
\end{equation}
Now, using the same technique we used before for finding the wave equation, we get
\begin{equation*}
	\begin{aligned}
		\nabla\times\nabla\times\vec{E}_\omega&= -\omega^2\mu_0\epsilon_\omega\vec{E}_\omega\\
		\nabla\times\nabla\times\vec{H}_\omega&= -\omega^2\mu_0\epsilon_\omega\vec{H}_\omega
	\end{aligned}
\end{equation*}
Therefore, inserting the divergence equations and taking care of the minus signs
\begin{equation*}
	\begin{aligned}
		\nabla^2\vec{E}_\omega+\nabla\left( \frac{\vec{E}_\omega\cdot\nabla\epsilon_\omega}{\epsilon_\omega} \right)&= \omega^2\mu_0\epsilon_\omega\vec{E}_\omega\\
		\nabla^2\vec{H}_\omega&= \omega^2\mu_0\epsilon_\omega\vec{H}_\omega
	\end{aligned}
\end{equation*}
\begin{dfn}[Refraction Index]
	We define the \textit{refraction index} $n_\omega$ as follows
	\begin{equation}
		n_\omega(\vec{r})=\sqrt{\frac{\epsilon_\omega(\vec{r})}{\epsilon_0}}
		\label{eq:refractionind.waves}
	\end{equation}
	Hence
	\begin{equation*}
		\epsilon_\omega=\epsilon_0n_\omega^2
	\end{equation*}
\end{dfn}
Inserting the previous definition into the divergence of $E_\omega$ we see that
\begin{equation*}
	\frac{\vec{E}_\omega\cdot\nabla\epsilon_\omega}{\epsilon_\omega}=2\vec{E}_\omega\cdot\nabla\log\left( n_\omega \right)
\end{equation*}
And, the right hand side becomes
\begin{equation*}
	\omega^2\mu_0\epsilon_\omega\vec{E}_\omega=\frac{\omega^2}{c^2}n_\omega^2\vec{E}_\omega
\end{equation*}
\begin{dfn}[Wavenumber]
	We define the \textit{vacuum wavenumber} $k_0$ as follows
	\begin{equation}
		k_0=\frac{\omega}{c}=\frac{2\pi}{\lambda}
		\label{eq:wavenumber.waves}
	\end{equation}
\end{dfn}
Reuniting both definitions and the simplification we then get, for the equation on $E$
\begin{equation*}
	\nabla^2\vec{E}_\omega+2\nabla\left( \vec{E}_\omega\cdot\nabla\log(n_\omega) \right)=k_0^2n_\omega^2\vec{E}_\omega
\end{equation*}
Now, in order to ease calculations in our range of frequencies (or wavelenghts, or wavenumbers also) we check if defined a characteristic length $l_n$ which indicates the spatial scale of variation of $n_\omega$ (remember that it depends on space position), we have that:
\begin{equation*}
	\begin{aligned}
		\nabla^2E_\omega&\propto\frac{E_\omega}{l_n\lambda}\frac{\Delta n_\omega}{n_\omega}\propto k_0^2n_\omega^2E_\omega\\
		\norm{\nabla\nabla\cdot\vec{E}_\omega}&\propto\frac{E_\omega}{l_n^2}\frac{\Delta n_\omega}{n_\omega}
	\end{aligned}
\end{equation*}
And since in optical ranges of light $l_n>>\lambda$ and $\Delta n_\omega<<n_\omega$ we can discard immediately the divergence term, and get two symmetric wave equations for the wave in a generic dielectric (nonmagnetic or transparent) medium
\begin{equation}
	\left\{ \begin{aligned}
			\nabla^2\vec{E}_\omega&= k_0^2n_\omega^2\vec{E}_\omega\\
			\nabla^2\vec{H}_\omega&= k_0^2n_\omega^2\vec{H}_\omega
	\end{aligned}\right.
	\label{eq:helmholtz.waves}
\end{equation}
These equations are known as \textit{Helmholtz equations} for the single chromatic part of the $E, H$ fields. Note that these equations are exactly the equations solved by a single Fourier component of the transformed wave equation ($n$ appears since we are in some media and $\epsilon_r\ne0$).\\
Note that via Fourier transforms it's possible to go back to the already known wave equation, where the speed of propagation is not $c$ but it's $c/n_\omega=u$. It's then also obvious that in vacuum $n_\omega=1$, which is also clear from the definition of the refraction index.\\
%%TODO waves appendix, phase-group velocity
\section{Vectorial Behavior of Electromagnetic Waves}
The general solution to the wave equation \eqref{eq:waveeq.waves} can be written both as a real function or a complex exponential. The latter one, although non ``real'', eases a lot calculations and therefore will be the favored approach. Using all previous definitions the general result is
\begin{equation}
	\begin{aligned}
		\vec{E}&= \vec{E}_0e^{i\vec{k}\cdot\vec{r}-i\omega t}\\
		\vec{H}&= \vec{H}_0e^{i\vec{k}\cdot\vec{r}-i\omega t}
	\end{aligned}
	\label{eq:wavesolution.waves}
\end{equation}
Evaluating the curl, the divergence and the time derivative for these solutions it's possible to write the following operator equation
\begin{equation}
	\begin{aligned}
		\hat{\del}_t&\to-i\omega\\
		\hat{\nabla}&\to i\vec{k}
	\end{aligned}
	\label{eq:operatorrel.waves}
\end{equation}
Where $\vec{k}$ is known as the \textit{wavevector}, which is the vector with magnitude $\norm{\vec{k}}=nk_0$
And Maxwell's equations can be rewritten as
\begin{equation}
	\left\{ \begin{aligned}
			\vec{k}\cdot\vec{E}&= 0\\
			\vec{k}\cdot\vec{H}&= 0\\
			\vec{k}\times\vec{E}&= \omega\mu_0\vec{H}\\
			\vec{k}\times\vec{H}&= -\omega\epsilon\vec{E}\\
	\end{aligned}\right.
	\label{eq:maxwellktimes.waves}
\end{equation}
Therefore, $\vec{k},\vec{E},\vec{H}$ are three mutually orthogonal vectors. By definition, then, $\vec{k}$ must be oriented parallel to the direction of motion.\\
Also from the last three equations, taken the norm of the two and solved the equations for $E/H$ we have
\begin{equation}
	\frac{E}{H}=n\sqrt{\frac{\epsilon_0}{\mu_0}}=\frac{n}{Z_0}
	\label{eq:impedancefs.waves}
\end{equation}
With $Z_0^2=\mu_0/\epsilon_0$ being the \textit{free space impedance} which has value of $Z_0\approx377\Omega$.
Note also that by definition of the Poynting vector $\vec{S}$, we also have that $\vec{k}\parallel\vec{S}$, which implies that, if we define the irradiance as $I=\norm{\vec{S}}$, that:
\begin{equation}
	\vec{S}=I\frac{\vec{k}}{k}=I\hat{\vec{k}}
	\label{eq:irradiance.waves}
\end{equation}
Or, using what we found before
\begin{equation}
	I=\frac{n}{2Z_0}E_0^2
	\label{eq:irradiancevse.waves}
\end{equation}
For plane waves it's actually better if we take the time average of the fields using phasor notation (complex exponential notation) for the fields, we have
\begin{equation*}
	\expval{\vec{S}(t)}=\vec{S}=\real{\expval{\vec{E}_0e^{i\vec{k}\cdot\vec{r}-i\omega t}}}\times\real{\expval{\vec{H}_0e^{i\vec{k}\cdot\vec{r}-i\omega t}}}
\end{equation*}
Since $\real(z)=\frac{1}{2}(z+\cc{z})$ and the complex conjugate is distributive, the calculation boils down to simplifying the following expression
\begin{equation*}
	\frac{1}{4}\expval{\left( \vec{E}_0\times\vec{H}_0e^{2i\vec{k}\cdot\vec{r}-2i\omega t}+\cc{\vec{E}}_0\times\vec{H}_0=\vec{E}_0\times\cc{\vec{H}}_0+\cc{\vec{E}}_0\times\cc{\vec{H}}_0e^{-2i\vec{k}\cdot\vec{r}+2i\omega t}\right)}
\end{equation*}
Noting that the average value for the real part of the exponential is $0$, we have 
\begin{equation}
	\vec{S}=\frac{1}{2}\vec{E}_0\times\cc{\vec{H}}_0 
	\label{eq:averagepoynting.waves}
\end{equation}
Where we omitted both the real part operator and the time average.
\subsection{Polarization}
The vectorial nature of waves comes up in most part with the phenomenon of \textit{polarization}, which is simply the ``favored'' direction of oscillation of the wave. Waves can also be non-polarized, as is the case for natural light, when there is no well defined oscillation direction.\\
The simplest polarization state obtainable is \textit{linear polarization}. Setting $\vec{k}\parallel\hat{\vec{z}}$ we have $\vec{E},\vec{H}$ orthogonal and coplanar on the $xy$ plane.\\
Linear polarization is then achieved when $\vec{E}$ oscillates with a constant angle from the chosen x axis. This can be expressed mathematically as:
\begin{equation}
	\vec{E}=E_x\hat{\vec{x}}+E_y\hat{\vec{y}}
	\label{eq:linear.pol}
\end{equation}
Note how here, in general, $\vec{E}\in\R^2$.\\
The instrument used to generate linearly polarized light is the \textit{linear polarizer}. This object is built in a way such that it transmits light only in one orientation. The associated axis is known as the \textit{transmission axis}.\\
Chosen a 2D orthonormal reference system $\hat{\vec{t}},\hat{\vec{s}}$ in which the $t$ axis is parallel to the transmission axis, then, we must have
\begin{equation*}
	\begin{aligned}
		\vec{E}_{in}&= E_t\hat{\vec{t}}+E_s\hat{\vec{s}}\\
		\vec{E}_{out}&= E_t\hat{\vec{t}}
	\end{aligned}
\end{equation*}
Since $E_t=E_{in}\cos\theta$ then the irradiance of the outgoing field is
\begin{equation}
	I_{out}=I_{in}\cos^2\theta
	\label{eq:malus.pol}
\end{equation}
This behavior is known as \textit{Malus' law}.\\
In case that the incoming light isn't polarized and it can't be described with the previous decomposition, taken the time average, we get
\begin{equation*}
	I_{out}=\frac{1}{2}I_{in}
\end{equation*}
Due to the superposition principle, it's not hard to imagine a mixture of polarized and unpolarized light. The \textit{degree of polarization} $P$ of this light can be evaluated using Malus' law, and it will be equal to
\begin{equation}
	P=\frac{I_{max}-I_{min}}{I_{max}+I_{min}}
	\label{eq:degpol.pol}
\end{equation}
Or, also, as the fraction of polarized irradiance
\begin{equation}
	P=\frac{I_{pol}}{I_{pol}+I_{unp}}
	\label{eq:degpol2.pol}
\end{equation}
There is one more possible (general) state of polarization, in which the field is totally described as a complex vector. In this state the two components of the field are dephased by exactly $\phi=\pi/2$, adding the phase to the exponential, we get a new factor of $i$ on one of the two components.\\
The field can then be described by a complex vector in the following way
\begin{equation}
	\vec{E}=E_0\hat{\vec{x}}+iE_1\hat{\vec{y}}
	\label{eq:elliptical.pol}
\end{equation}
This configuration is known as \textit{elliptical polarization} or \textit{circular polarization} when $E_1=E_0$, i.e.
\begin{equation}
	\vec{E}=E_0(\hat{\vec{x}}+i\hat{\vec{y}})
	\label{eq:circular.pol}
\end{equation}
This special polarization state has two version, called \textit{right hand} and \textit{left hand} polarization, depending on whether the dephasing between the components is $+i$ or $-i$. Left hand elliptical (circular) polarization is defined by a positive phase difference of $i$.
\section{Reflection and Refraction}
<++>
\subsection{Snell's Law}
\subsection{Fermat's Principle}
\subsection{Fresnel Equations}
\subsubsection{Fresnel Equations for Irradiance}
\subsubsection{Brewster's Angle}
\section{Total Internal Reflection}
\subsection{Evanescent Waves}
\subsection{Fiber Optics}
\subsection{Fresnel's Rhomb}
\end{document}
