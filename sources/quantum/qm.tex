\documentclass[a4paper, 11pt]{book}
\usepackage{fancypages}
\usepackage{sapienza}
\usepackage{tikzorbital}
\usepackage{chemfig}
\usepackage{tikz}
\usetikzlibrary{patterns}
\usetikzlibrary{decorations}
\usetikzlibrary{calc}
\usepackage[compat=1.1.0]{tikz-feynman}
\usepackage{dsfont}
\usepackage[]{tipa}
%
\renewcommand{\vec}[1]{\underline{#1}}
\renewcommand{\trace}{\mathrm{Tr}}
\newcommand{\1}{\opr{\mathds{1}}}
\newcommand{\lbar}{\mbox{\textipa\textcrlambda}}
\renewcommand{\unit}[1]{\ \mathrm{#1}}
\newcommand{\diff}[2][]{\ \mathrm{d}^{#1}#2}
\newcommand{\ddiff}[3][]{\ \mathrm{d}^{#1}#2\mathrm{d}^{#1}#3}
\newcommand{\dddiff}[4][]{\ \mathrm{d}^{#1}#2\mathrm{d}^{#1}#3\mathrm{d}^{#1}#4}
\newcommand{\ham}{\mathcal{H}}
\newcommand{\opr}[1]{\hat{#1}}
\newcommand{\tposed}[1]{#1^{\text{T}}}
\newcommand{\pcomm}[2]{\comm{#1}{#2}_{PB}}
\newcommand{\Ut}{\opr{\mathcal{U}}}
\newcommand{\pot}{\mathcal{U}}
\newcommand{\lc}[1]{\epsilon_{#1}}
\newcommand{\kd}[1]{\delta_{#1}}
\newcommand{\kdopr}[1]{\opr{\delta}_{#1}}
\newcommand{\mc}[1]{\mathcal{#1}}
\newcommand{\ladopru}[1]{\opr{#1}_{+}}
\newcommand{\ladoprd}[1]{\opr{#1}_{-}}
\newcommand{\ladoprpm}[1]{\opr{#1}_{\pm}}
\newcommand{\vecopr}[1]{\opr{\vec{#1}}}
\newcommand{\up}{\uparrow}
\newcommand{\down}{\downarrow}
\newcommand{\hilbert}{\mathbb{H}}
\newcommand{\sph}{Y^{m_l}_l(\theta,\phi)}
\newcommand{\tsph}{\mc{Y}^{ls}_{jm}(\theta,\phi)}
\newcommand{\oham}{\opr{\mathcal{H}}}
\newcommand{\sla}[1]{\slashed{#1}}
\newcommand{\term}[3][]{^{#3}#2_{#1}}
\renewcommand{\F}{\hat{\mathcal{F}}}
\newcommand{\act}{\mathcal{S}}
\newcommand{\lag}{\mathcal{L}}
\newcommand{\dopr}{\hat{\rho}}
\newcommand{\qsum}{\sideset{}{'}\sum}
\newcommand{\nuc}[4][]{ {}_{#4}^{#3}{\mathrm{#2}}_{#1} } %\nuc[N]{M}{A}{Z}
\DeclarePairedDelimiter\rounding{\lceil}{\rfloor}
%
\newtheorem{pos}{Postulate}
\newtheorem{defn}{Definition}[section]
\newtheorem{law}{Law}
%
%
%\newtheorem{thm}{Theorem}
%\theoremstyle{plain}
%\newtheorem{cor}{Corollary}
%\newtheorem{hyp}{Hypothesis}
%%
%%
\begin{document}
\subfile{./title/title}
\subfile{./title/frontispice}
\chapter*{Preface}
	\subfile{./chapters/0-preface}
\tableofcontents
\pagenumbering{arabic}
\part{Quantum Mechanics}
	\chapter{The Failure of Classical Physics}
		\subfile{./chapters/1-failure}
	\chapter{The Fundamentals}
		\subfile{./chapters/2-fundamentals}
	\chapter{Quantum Dynamics in 1D}
		\subfile{./chapters/3-quantum1d}
		\chapter{Angular Momentum}
		\subfile{./chapters/4-angularmomentum}
	\chapter{Quantum Dynamics in 3D}
		\subfile{./chapters/5-quantum3d}
	\chapter{Approximation Methods}
		\subfile{./chapters/6-approximation}
	\chapter{Identical Particles}
		\subfile{./chapters/7-identical}
\part{Thermodynamics and Statistical Mechanics}
	%\chapter{Thermodynamical systems}
		\subfile{./chapters/thermosystems.tex}
	%\chapter{Heat}
		\subfile{./chapters/thermoheat.tex}
%	\chapter{2nd principle}
		\subfile{./chapters/thermo2nd.tex}
%	\chapter{Phase Transitions}
		\subfile{./chapters/thermopot.tex}
	\chapter{Introduction to Thermostatistics}
		\subfile{./chapters/15-msintro}
	\chapter{Microcanonical Ensemble}
		\subfile{./chapters/16-microcanonical}
	\chapter{Canonical Ensemble}
		\subfile{./chapters/17-canonical}
	\chapter{Grand Canonical Ensemble}
		\subfile{./chapters/18-grandcanonical}
	\chapter{Ideal Quantum Gases}
		\subfile{./chapters/idealquantumgas}
\part{Atomic Physics}
	\chapter{One Electron Atoms}
		\subfile{./chapters/8-oneelectron}
	\chapter{Two Electron Atoms}
		\subfile{./chapters/9-twoelectrons}
	\chapter{Many Electron Atoms}
		\subfile{./chapters/10-manyelectrons}
	\chapter{Electromagnetic Interactions}
		\subfile{./chapters/11-eminteractions}
\part{Quantum Chemistry}
	\chapter{Basic Chemistry}
		\subfile{./chapters/chem}
	\chapter{Molecular Structure}
		\subfile{./chapters/12-molecules}
	\chapter{Molecular Spectra}
		\subfile{./chapters/14-spectra}
	\chapter{Molecular Electronic Structure}
		\subfile{./chapters/13-electronsmolecules}
	\chapter{Introduction to Solid State Physics}
		\subfile{./chapters/solids}
%\part{Statistical Mechanics}
%	\chapter{Brief Introduction}
%		\subfile{./chapters/15-msintro}
%	\chapter{Microcanonical Ensemble}
%		\subfile{./chapters/16-microcanonical}
%	\chapter{Canonical Ensemble}
%		\subfile{./chapters/17-canonical}
%	\chapter{Grand Canonical Ensemble}
%		\subfile{./chapters/18-grandcanonical}
%	\chapter{Ideal Quantum Gases}
%		\subfile{./chapters/idealquantumgas}
%%
\part{Nuclear and Particle Physics}
	\chapter{Scattering and Cross Sections}
	\subfile{./chapters/pphys}
	\chapter{Nuclear Physics}
	\subfile{./chapters/nuclear}
	\chapter{Electromagnetic Interactions}
	\subfile{./chapters/interactions}
	\chapter{Particle Detectors}
	\subfile{./chapters/detectors}
	\chapter{Basics of Quantum Field Theory}
	\subfile{./chapters/qft.tex}
%	\chapter{Simmetries and Invariance}
%	\subfile{./chapters/simmetries}
%	\chapter{Nuclear Physics}
%	\subfile{./chapters/nuclear}
%\part{Quantum Field Theory}
%	\chapter{Historical Discoveries}
%		\subfile{./chapters/qftintro}
\part{Appendices}
\appendix
	\chapter{Mathematical Methods}
		\subfile{./appendices/maths}
	\chapter{Clebsch-Gordan Table}\label{app:cgt}
		\subfile{./appendices/cbtable}
	\chapter{Tensor Spherical Harmonics}\label{app:tsh}
		\subfile{./appendices/tsh}
	\chapter{Calculus of $\expval{r^k}_{lnm}$ Integrals}
		\subfile{./appendices/integral}
	\chapter{Periodic Table}\label{app:E}
		\subfile{./appendices/periodic}
%	\chapter{Dipole Transitions}
%		\subfile{./appendices/transitions}
	\chapter{Symmetry and Point Groups}\label{app:groups}
		\subfile{./appendices/groups}
	\chapter{Special Relativity}
		\subfile{./appendices/lorentz}
		\subfile{./appendices/lep}
		\subfile{./appendices/relc}
% E N D %
\nocite{quantistica,landau3,statistica,struttura,struttura1,griffmq,sakuraimqm,patritesta,molekulphysik,complessa,griffpart,nucleare,relativitat,relativita,pdg,griffpart,thermoditt}
\printbibliography
\end{document}
%a
%\usepackage[style=alphabetic]{biblatex}
%\addbibresource{~/Dropbox/appunti/bibliografia/bib.bib}
%\usepackage{subfiles}
%\usepackage{microtype}
%\usepackage{mathtools}
%\usepackage{amssymb}
%\usepackage{amsfonts}
%\usepackage{amsthm}
%\usepackage{fontspec}
%\usepackage{lmodern}
%\usepackage[]{graphicx}
\graphicspath{{./images/}}
%\usepackage{physics}
%\usepackage{pgfplots}
%\pgfplotsset{compat=1.16}
%\usepackage{pdfpages}
%\usepackage{tikzorbital}
%\usepackage[]{geometry}
%\usepackage{xcolor}
%\usepackage{fancyref}
%\usepackage{fancyhdr}
%\usepackage{float}
%\usepackage{slashed}
%\usepackage{titlesec}
%\usepackage{subfiles}
%\usepackage{fancyhdr}
%\usepackage{etoolbox}
%\definecolor{sapienza}{cmyk}{.10,1.00,.61,.50}
%\usepackage[colorlinks=true,linkcolor=black,urlcolor=blue,citecolor=red]{hyperref}
%%
%\definecolor{title}{cmyk}{1,0.60,0,0.40}
%\newcommand{\plogo}{\fbox{$\mathcal{MC}$}}
%%
%\setmainfont[
%	Ligatures=TeX,
%	UprightFont=*_45_light,
%	ItalicFont=SALFrutiger-Italic,
%	BoldFont=*_bold,
%	BoldItalicFont=SALFrutiger-BoldItal,
%]{frutiger}
%\setlength{\headheight}{13.6pt}
%\fancyhf{}
%\fancyhead[R]{\textbf{\textsf{\thepage}}}
%\fancyhead[LE]{\textbf{\textsf{\leftmark}}}
%\fancyhead[LO]{\textbf{\textsf{\rightmark}}}
%\pagestyle{fancy}
%\titleformat{\chapter}[block]{\bfseries}{\fontsize{40}{30}\selectfont\color{gray}\textsf\thechapter}{1.5em}{\fontsize{25}{25}\selectfont\scshape}[\vspace{-1ex}\hfill\rule{\textwidth}{0.5pt}]
%\titleformat{\section}[block]{}{\fontsize{14}{12}\textsf{\S\ \bfseries\thesection}}{0.5em}{\fontsize{14}{12}\bfseries\textsf}[\vspace{-1ex}\hfill\rule{\textwidth}{0.1pt}]
%\titleformat{\subsection}[block]{}{\textsf{\S\S\ \bfseries\thesubsection}}{0.5em}{\bfseries\textsf}
%%
%%%
%%
%\AtBeginEnvironment{equation}{\color{sapienza}}
%\AtBeginEnvironment{equation*}{\color{sapienza}}
%\renewcommand{\vec}[1]{\mathbf{#1}}
