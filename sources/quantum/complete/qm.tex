\documentclass[a4paper, 11pt]{book}
\usepackage[style=alphabetic]{biblatex}
\addbibresource{bib.bib}
\usepackage{subfiles}
\usepackage{microtype}
\usepackage{mathtools}
\usepackage{amssymb}
\usepackage{amsfonts}
\usepackage{amsthm}
\usepackage{fontspec}
\usepackage{lmodern}
\usepackage[]{graphicx}
\graphicspath{{images/}{../images/}}
\usepackage{physics}
\usepackage{chemfig}
\usepackage{tikz}
\usepackage{pgfplots}
\usepackage{tikz-feynman}
\usepackage{tikzorbital}
\usepackage{dsfont}
\usepackage[]{geometry}
\usepackage{xcolor}
\usepackage{fancyref}
\usepackage{fancyhdr}
\usepackage{float}
\usepackage[]{tipa}
\usepackage{slashed}
\usepackage{titlesec}
\usepackage{subfiles}
\usepackage[colorlinks=true,linkcolor=black,urlcolor=blue,citecolor=red]{hyperref}
\usepackage{fancyhdr}
\definecolor{titlepagecolor}{cmyk}{1,.60,0,.40}
\newcommand{\plogo}{\fbox{$\mathcal{MC}$}}
\setlength{\headheight}{13.6pt}
\fancyhf{}
\fancyhead[R]{\textbf{\textsf{\thepage}}}
\fancyhead[LE]{\textbf{\textsf{\leftmark}}}
\fancyhead[LO]{\textbf{\textsf{\rightmark}}}
\pagestyle{fancy}
\titleformat{\chapter}[block]%
{\bfseries\large}%
{\fontsize{40}{30}\selectfont\color{gray}\textsf\thechapter}%
{1.5em}
{\fontsize{25}{20}\selectfont\scshape}%
[\vspace{-1ex}%
	\hfill%
\rule{\textwidth}{0.5pt}]
\renewcommand{\maketitle}[5][7.5cm]{
	\begin{titlepage}
		\newgeometry{left=#1}
		\scshape
		\pagecolor{titlepagecolor}
		%		\noindent
		%		\includegraphics[width=2cm]{sfondo}\\[-1em]
		%		\vskip\baselineskip
		\noindent
		\color{white}
		{\Huge #2}
		\vskip0.1\baselineskip\noindent
		\makebox[0pt][l]{\rule{1.2\textwidth}{0.5pt}}
		\par
		\noindent
		\textit{Università degli Studi di Roma "La Sapienza"}\\\textit{Physics BSc}
		\vskip\baselineskip
		\vskip\baselineskip
		\noindent
		\textsf{#5}
		\vfill
		\noindent
		\textsf{Notes on #2}
		\vskip0.1\baselineskip\noindent
		\makebox[0pt][l]{\rule{1.2\textwidth}{0.5pt}}
		\vskip\baselineskip
		\noindent
		\textsf{#3}
		\vskip\baselineskip
		\noindent
		\textsf{\Large Version #4}
		\restoregeometry
		\pagecolor{white}
	\end{titlepage}
	\begin{titlepage}
		\centering
		\scshape
		\rule{\textwidth}{1.6pt}\vspace*{-\baselineskip}\vspace*{2pt}
		\rule{\textwidth}{0.4pt}

		\vspace{0.75\baselineskip}
		{\Huge #2\\}
		\vspace{0.75\baselineskip}
		\rule{\textwidth}{0.4pt}\vspace*{-\baselineskip}\vspace{3.2pt}
		\rule{\textwidth}{1.6pt}
		\vspace{2\baselineskip}

		Notes on #2

		\vspace*{6.5\baselineskip}

		Written by

		\vspace{0.5\baselineskip}

		{\scshape\Large  #5}
		\vspace{0.5\baselineskip}

		\textit{Università degli Studi di Roma "La Sapienza"\\ Physics BSc}

		\vfill

		\plogo

		\vspace{0.7\baselineskip}

		#3

		\vspace{0.5\baselineskip}

		{\Large Version #4}
		\newpage
	\end{titlepage}
}
\renewcommand{\vec}[1]{\mathbf{#1}}
\renewcommand{\trace}{\mathrm{Tr}}
\newcommand{\ver}[1]{\vec{e}_{#1}}
\newcommand{\1}{\opr{\mathds{1}}}
\newcommand{\lbar}{\mbox{\textipa\textcrlambda}}
\newcommand{\unit}[1]{\ \mathrm{#1}}
\newcommand{\diff}[2][]{\ \mathrm{d}^{#1}#2}
\newcommand{\ddiff}[3][]{\ \mathrm{d}^{#1}#2\mathrm{d}^{#1}#3}
\newcommand{\dddiff}[4][]{\ \mathrm{d}^{#1}#2\mathrm{d}^{#1}#3\mathrm{d}^{#1}#4}
\newcommand{\ham}{\mathcal{H}}
\newcommand{\opr}[1]{\hat{#1}}
\newcommand{\adj}[2][]{#2^{\dagger#1}}
\newcommand{\tposed}[1]{#1^{\text{T}}}
\newcommand{\pcomm}[2]{\comm{#1}{#2}_{PB}}
\newcommand{\U}{\opr{\mathcal{U}}}
\newcommand{\N}{\mathbb{N}}
\newcommand{\cc}[1]{\overline{#1}}
\newcommand{\lc}[1]{\epsilon_{#1}}
\newcommand{\kd}[1]{\delta_{#1}}
\newcommand{\kdopr}[1]{\opr{\delta}_{#1}}
\newcommand{\mc}[1]{\mathcal{#1}}
\newcommand{\ladopru}[1]{\opr{#1}_{+}}
\newcommand{\ladoprd}[1]{\opr{#1}_{-}}
\newcommand{\ladoprpm}[1]{\opr{#1}_{\pm}}
\newcommand{\vecopr}[1]{\opr{\vec{#1}}}
\newcommand{\up}{\uparrow}
\newcommand{\down}{\downarrow}
\newcommand{\hilbert}{\mathbb{H}}
\newcommand{\sph}{Y^{m_l}_l(\theta,\phi)}
\newcommand{\tsph}{\mc{Y}^{ls}_{jm}(\theta,\phi)}
\newcommand{\oham}{\opr{\mathcal{H}}}
\newcommand{\sla}[1]{\slashed{#1}}
\newcommand{\term}[3][]{^{#3}#2_{#1}}
\newcommand{\F}{\hat{\mathcal{F}}}
\newcommand{\R}{\mathbb{R}}
\newcommand{\C}{\mathbb{C}}
\newcommand{\dopr}{\hat{\rho}}
\newcommand{\qsum}{\sideset{}{'}\sum}
\newtheorem{pos}{Postulate}
\newtheorem{thm}{Theorem}
\theoremstyle{plain}
\newtheorem{defn}{Definition}
\newtheorem{cor}{Corollary}
\newtheorem{hyp}{Hypothesis}
\begin{document}
\maketitle{Quantum Mechanics}{\today}{$0.7\beta$}{Matteo Cheri}
\tableofcontents
\part{Quantum Mechanics}
		\chapter{The Failure of Classical Physics}
	\subfile{./chapters/1-failure}
		\chapter{The Fundamentals}
	\subfile{./chapters/2-fundamentals}
		\chapter{Quantum Dynamics in 1D}
	\subfile{./chapters/3-quantum1d}
		\chapter{The Theory of Angular Momentum}
	\subfile{./chapters/4-angularmomentum}
		\chapter{Quantum Dynamics in 3D}
	\subfile{./chapters/5-quantum3d}
		\chapter{Approximation Methods}
	\subfile{./chapters/6-approximation}
		\chapter{Identical Particles}
	\subfile{./chapters/7-identical}
\part{Atomic Physics}
		\chapter{One Electron Atoms}
	\subfile{./chapters/8-oneelectron}
		\chapter{Two Electron Atoms}
	\subfile{./chapters/9-twoelectrons}
		\chapter{Many Electron Atoms}
	\subfile{./chapters/10-manyelectrons}
		\chapter{Electromagnetic Interactions}
	\subfile{./chapters/11-eminteractions}
\part{Quantum Chemistry}
		\chapter{Molecular Structure}
	\subfile{./chapters/12-molecules}
		\chapter{Electronic Structure of Molecules}
	\subfile{./chapters/13-electronsmolecules}
		\chapter{Molecular Spectra}
	\subfile{./chapters/14-spectra}
\part{Statistical Mechanics}
		\chapter{Brief Introduction}
	\subfile{./chapters/15-msintro}
		\chapter{Microcanonical Ensemble}
	\subfile{./chapters/16-microcanonical}
		\chapter{Canonical Ensemble}
	\subfile{./chapters/17-canonical}
		\chapter{Grand Canonical Ensemble}
	\subfile{./chapters/18-grandcanonical}
		\chapter{Ideal Quantum Gases}
	\subfile{./chapters/idealquantumgas}

\appendix
		\chapter{Mathematical Methods}
	\subfile{./appendices/maths}
		\chapter{Clebsch-Gordan Table}
	\subfile{./appendices/cbtable}
		\chapter{Tensor Spherical Harmonics}\label{app:tsh}
	\subfile{./appendices/tsh}
		\chapter{Calculus of $\expval{r^k}_{lnm}$ Integrals}
	\subfile{./appendices/integral}
		\chapter{Periodic Table}
	\subfile{./appendices/periodic}
		\chapter{Symmetry and Point Groups}\label{app:groups}
	\subfile{./appendices/groups}
\nocite{quantistica,landau3,statistica,struttura,struttura1,griffmq,sakuraimqm,patritesta,molekulphysik,complessa}
\printbibliography
\end{document}
