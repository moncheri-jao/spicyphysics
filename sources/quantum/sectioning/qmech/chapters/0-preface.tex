\documentclass[../qm.tex]{subfiles}
\begin{document}
The idea of writing my notes this particular way came from attending the classes my third year of my BSc course in Physics, especially \textit{Meccanica Quantistica e Statistica} with Prof.Presilla and \textit{Struttura della Materia} with Prof.Postorino. Both these beautiful subjects were explained flawlessly during the lectures, but with a huge problem: there are way too many books from where to get your really needed study, and most aren't really books I'd say I like, they're either too long, don't dwelve with enough precision on important topics and most, unfortunately, get lost in a sea of words that just messes with the mind of the reader and helps to make such a beautiful subject even harder than it already is. The reader I'm talking about is obviously me, and my lazy mind thought that it would be better to write this set of notes using the power of typesetting with \LaTeXe\ finally creating a ``huge'' collection of notes, enlarged using various books I found really helpful in my time passed by studying.\\
These notes are, obviously, only a student's notes, and might have errors, phrases or whole paragraphs that make no sense and stains of coffee on some equations\footnote{\scalebox{0.1}{I absolutely don't, and won't, take responsibility for any explicitly and voluntarily malicious error present in these notes}}. I hope that these ``little'' problems won't hurt the reader but rather encourage him to write a detailed mail to me at this address \href{mailto:cheri.1686219@studenti.uniroma1.it}{cheri.1686219@studenti.uniroma1.it}, so that I can actually fix the error and send out a new, updated, version.\\
I'd like to thank \textit{AISF Local Committee Roma Sapienza} and everyone that contributed even with a simple ``thanks'' or a ``keep going''. One day when I'll be rich I'll pay you back with a coffee and a hug, thank you.
\vskip 0.5cm
\begin{raggedleft}
\ \hfill\textit{Rome, \today}
\end{raggedleft}
\vskip 1cm
\begin{center}
\textit{Sincerely yours, Matteo Cheri}
\end{center}
\vfill
\end{document}
