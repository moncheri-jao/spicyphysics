\documentclass[../qm.tex]{subfiles}
\begin{document}
\section{Basic Notions}
\subsection{Moles and Molar Mass}
Given some chemical compound $\mathrm{X}=\mathrm{M_aN_b}$ it's possible to find the molar mass $M_X$ by summing the single masses of the atomic components. The masses are found in the periodic table (\ref{app:E}) precisely on the top right of the box of the element considered, and is expressed in $\unit{g/mol}$. Therefore, for the compound $X$ the total molar mass is
\begin{equation}
	M_X=am_M+bm_N
	\label{eq:molarmass}
\end{equation}
Where $m_M,m_N$ are the masses of the single atoms. Note that for compounds with more than two composing elements, this calculation easily generalizes to a simple sum.\\
Given an amount $m_X$ of the compound, the total amount of moles $n_X$ contained is
\begin{equation}
	n_X=\frac{m_X}{M_X}
	\label{eq:molescompound}
\end{equation}
And the total number of molecules is
\begin{equation}
	p=n_XN_A
	\label{eq:totalnummol}
\end{equation}
Where $N_A$ is the Avogadro number, $N_A=6.022\cdot10^{23}\unit{mol^-1}$
\begin{eg}
	Given $m=12.49\unit{g}$ of $\mathrm{C_3H_8O_3}$, find the amount of moles and how many particles of the compound are contained in this mass.
	\begin{equation*}
		\begin{aligned}
			m_C&=12.011\unit{g/mol}\\
			m_H&=1.008\unit{g/mol}\\
			m_O&=15.999\unit{g/mol}
		\end{aligned}
	\end{equation*}
	Therefore
	\begin{equation*}
		M=3m_C+8m_H+3m_O=92.094\unit{g/mol}
	\end{equation*}
	Which means
	\begin{equation*}
		n=\frac{m}{M}=\frac{12.49}{92.094}\unit{mol}=0.136\unit{mol}
	\end{equation*}
	The number of particles is then
	\begin{equation*}
		p=nN_A=8.19\cdot10^{22}
	\end{equation*}
\end{eg}
\subsection{Elemental Analysis}
If the compound itself is unknown, with some data it's possible to find the chemical formula for the compound.\\
Suppose that the unknown compound $\mathrm{X}$ is composed by $n-$ secondary compounds in unknown quantities, say two without loss of generality $\mathrm{X}=\mathrm{A_xB_y}$. If the percentage concentration of the two elements is known, we have that for $100\unit{g}$ of compound there will be exactly $\%_A,\%_B$ grams of the single elements.\\
The two unknown coefficients $x,y$ then can be determined through a \textit{minimal combination} of the moles of the single atoms. We have then, that in $100\unit{g}$ of $\mathrm{X}$ there will be
\begin{equation}
	\begin{aligned}
		n_A&=\frac{\%_A}{m_A}\\
		n_B&=\frac{\%_B}{m_B}
	\end{aligned}
	\label{eq:molesin100g}
\end{equation}
The minimal combination will then made by taking $n_m=\min\left\{ n_A,n_B \right\}$, and the coefficients will be
\begin{equation}
	\begin{aligned}
		x&=\rounding*{\frac{n_A}{n_m}}\\
		y&=\rounding*{\frac{n_B}{n_m}}
	\end{aligned}
	\label{eq:coefelan}
\end{equation}
Note that if the value inside the rounding operator isn't close to a natural number $r$ ($\abs{n_i/n_m-r}>0.25$) the coefficient can't be properly determined. The solution is found by taking a number $k\in\N$ and multiplying all the values inside the rounding operator. This shall be repeated until there aren't any dubious values.
\begin{eg}[An Organic Compound]
	An unknown composite $\mathrm{X}$ is experimentally known to contain $\mathrm{C,H,O}$ with the following percentages:
	\begin{equation*}
		\begin{aligned}
			\%_C&=55.81\\
			\%_H&=7.02\\
			\%_O&=37.17
		\end{aligned}
	\end{equation*}
	Find the chemical formula for the compound $C_aH_bO_c$
	We have that in $100\unit{g}$ of $\mathrm{X}$
	\begin{equation*}
		\begin{aligned}
			n_C&=\frac{\%_C}{m_C}=\frac{55.81\unit{g}}{12.01\unit{g/mol}}=4.65\unit{mol}\\
			n_H&=\frac{\%_H}{m_H}=\frac{7.02\unit{g}}{1.01\unit{g/mol}}=6.96\unit{mol}\\
			n_O&=\frac{\%_O}{m_O}=\frac{37.17\unit{g}}{16.00\unit{g/mol}}=2.32\unit{mol}
		\end{aligned}
	\end{equation*}
	And therefore $\min\left\{n_C,n_H,n_O\right\}=n_O$ and the minimal combination gives the following coefficients
	\begin{equation*}
		\begin{aligned}
			a&=\rounding*{\frac{n_C}{n_O}}=2\\
			b&=\rounding*{\frac{n_H}{n_O}}=3\\
			c&=1
		\end{aligned}
	\end{equation*}
	The complete formula is therefore $\mathrm{C_2H_3O}$
\end{eg}
%check if wanna add the last two
\end{document}
