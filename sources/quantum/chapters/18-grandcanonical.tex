\documentclass[../qm.tex]{subfiles}
\begin{document}
	After having considered the canonical ensemble, i.e. where one system embedded in another can only exchange energy with it (like in a heat bath), we're going to study the grand canonical ensemble, where an overall system is isolated as in the canonical ensemble, but both system can exchange energy and particles.\\
	The probability distribution of the state variables $E_1,N_1,V_1$ is the following, in analogy with what we wrote for the canonical ensemble
	\begin{equation}
		\omega(E_1,N_1,V_1)=\frac{\Omega_1(E_1,N_1,V_1)\Omega_2(E-E_1,N-N_1,V-V_1)}{\Omega(E,N,V)}
		\label{eq:pdfgc}
	\end{equation}
	From the equilibrium conditions ($T_1=T_2,\ V_1=V_2, N_1=N_2$) obtained from the equivalence of the logarithmic derivatives obtained from the normalization conditions of $\Omega$, and from the differential of entropy in the variables $N,V,T$, we get, after defining the \textit{chemical potential} $\mu$
	\begin{equation*}
		\begin{aligned}
			\diff{S}&=\frac{1}{T}\diff{E}+\frac{P}{T}\diff{V}-\frac{\mu}{T}\diff{N}\\
			\frac{\mu}{T}&=k_B\pdv{N}\log\Omega
		\end{aligned}
	\end{equation*}
	We have finally
	\begin{equation}
		\mu=-k_BT\pdv{N}\log\Omega=-T\pdv{S}{N}
		\label{eq:chemicalpotential}
	\end{equation}
	We now need to define the density operator $\dopr_{G}$ for the subsystem $1$. The probability that in the embedded system there are $N_1$ particles in the state $\ket{n}$ is
	\begin{equation}
		p(N_1,E_{1n}(N_1),V_1)=\frac{\Omega_2(E-E_{1n},N-N_1,V_2)}{\Omega(E,N,V)}=\frac{1}{Z_G}e^{-\frac{E_{1n}-\mu N_1}{k_BT}}
		\label{eq:grandcanon}
	\end{equation}
	Where $Z_G$ is the grand canonical partition function, also known as \textit{Gibbs distribution}. We thus have
	\begin{equation}
		\dopr_G=\frac{1}{Z_G}e^{-\frac{\opr{\ham}_1-\mu N_1\1}{k_BT}}
		\label{eq:grandcanonrho}
	\end{equation}
	And
	\begin{equation}
		Z_G=\trace\left( e^{-\frac{\opr{\ham}_1 -\mu N_1\1}{k_BT}} \right)=\sum_{N_1}Z(N_1)e^{\frac{\mu N_1}{k_BT}}
		\label{eq:grandcanonpartition}
	\end{equation}
	The classical limit can be achieved simply applying this abuse of notation
	\begin{equation*}
		\trace\to\sum_{N_1}\int\diff{\Gamma_{N_1}}
	\end{equation*}
	From the definition of the density matrix, the entropy of the grand canonical ensemble will be
	\begin{equation}
		S_G=-k_B\expval{\log\dopr_g}=\frac{1}{T}\left( E-\mu N \right)+k_B\log\dopr_G
		\label{eq:gcentropy}
	\end{equation}
	\section{Thermodynamic Quantities}
	Analogously to the free energy of the canonical ensemble, we can define the \textit{grand potential} as follows
	\begin{equation}
		\Phi=-k_BT\log(Z_G)
		\label{eq:grandpotential}
	\end{equation}
	Where, using \eqref{eq:gcentropy}, we can write alternatively
	\begin{equation}
		\Phi(V,T,\mu)=E-TS_G-\mu N
		\label{eq:gpotcomp}
	\end{equation}
	And its differential relations follow
	\begin{equation}
		\begin{aligned}
			\pdv{\Phi}{T}&=\frac{1}{T}(\Phi-E+\mu N)=-S_G\\
			\pdv{\Phi}{V}&=\expval{\pdv{\opr{\ham}}{V}}=-P\\
			\pdv{\Phi}{\mu}&=-N
		\end{aligned}
		\label{eq:derivatives}
	\end{equation}
	Now, in order to avoid confusion between the ensembles and their properties, we can compose a table.
	\begin{table}[H]
		\centering
		\begin{tabular}{|c|c|c|c|}
			\hline
			Ensemble&Microcanonical&Canonical&Grand Canonical\\
			\hline
			Situation&Isolated&Energy Exchange&Energy\\
			&&&and\\
			&&&Particle Exchange\\
			\hline
			Independent Variables&$E,V,N$&$T,V,N$&$T,V,\mu$\\
			\hline
			Density Operator&$\Omega^{-1}\opr{\delta}\left( \opr{\ham}-E \right)$&$Z^{-1}e^{-\opr{\ham}/k_BT}$&$Z_G^{-1}e^{-(\opr{\ham}-\mu N\1)/k_BT}$\\
			\hline
			Normalization&$\Omega=\trace\left( \opr{\delta}(\opr{\ham}-E) \right)$&$Z=\trace(e^{-\opr{\ham}/k_BT})$&$Z_G=\trace\left( e^{-(\opr{\ham}-\mu N\1)/k_BT} \right)$\\
			\hline
			Thermodynamic Potential&$S$&$F$&$\Phi$\\
			\hline
		\end{tabular}
		\caption{Table of the various ensembles, with the definition of their density operators, their normalization and the main thermodynamic potential for each ensemble}
		\label{tab:ensembles}
	\end{table}
	\section{Examples}
	As our main example we choose again the classical ideal gas.\\
	We begin by calculating the partition function for $N$ particles (keep in mind that $\beta=(k_BT)^{-1}$
	\begin{equation}
		Z_N=\frac{1}{N!h^{3N}}\int_{V}\int_{}^{}e^{-\beta\sum_{i=1}^N\frac{p_i^2}{2m}}\ddiff[3N]{p}{q}
		\label{eq:partitionfunctioncggc}
	\end{equation}
	Noting that the integral on the coordinates is simply $V^N$, we can write, also noting the isotropy of momenta
	\begin{equation}
		Z_N=\frac{V^N}{N!h^{3N}}\left( \int_{0}^{\infty}e^{-\beta\frac{p^2}{2m}}\diff{p} \right)^{3N}=\frac{V^N}{N!}\left( \frac{2m\pi}{\beta h^2} \right)^{\frac{3N}{2}}
		\label{eq:cgnpart}
	\end{equation}
	Using the definition of thermal wavelength of $\lambda=h(2\pi mk_BT)^{-1/2}$, we get finally
	\begin{equation}
		Z_N=\frac{1}{N!}\left( \frac{V}{\lambda^3} \right)^{\frac{3N}{2}}
		\label{eq:partitionfunctionlambda}
	\end{equation}
	Writing the definition of the grand partition function, we get
	\begin{equation}
		Z_G=\sum_{N=0}^{\infty}Z_Ne^{\beta\mu N}=\sum_{N=0}^{\infty}\frac{e^{\beta\mu N}}{N!}\left( \frac{V}{\lambda^3} \right)^{N}=e^{z\frac{V}{\lambda^3}}
		\label{eq:grandcanonicalpartcg}
	\end{equation}
	Where we introduced $z$ as what is usually called as \textit{fugacity} as
	\begin{equation}
		z=e^{\beta\mu}
		\label{eq:fugacitydef}
	\end{equation}
	From this, we can write the grand potential and get all the thermodynamic relations of the classical gas
	\begin{equation}
		\Phi(T,V,\mu)=-k_BT\log(Z_G)=-k_BT\frac{zV}{\lambda^3}
		\label{eq:grandpotential}
	\end{equation}
	Therefore, omitting all the algebraical passages and calculating $\mu$ through the derivatives of the grand potential and the definition of fugacity, we have
	\begin{equation}
		\begin{aligned}
			-\pdv{\Phi}{\mu}&=N=\frac{zV}{\lambda^3}\\
			-V\pdv{\Phi}{V}&=PV=Nk_BT\\
			-\pdv{\Phi}{T}&=S=k_BN\left( \frac{5}{2}+\log\left( \frac{V}{N\lambda^3} \right) \right)\\
			\mu&=-k_BT\log\left( \frac{V}{N\lambda^3} \right)
		\end{aligned}
		\label{eq:thermodynamicvargcg}
	\end{equation}
\end{document}
