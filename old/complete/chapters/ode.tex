\documentclass[../complete.tex]{subfiles}
\begin{document}
\section{Existence of Solutions}
\textit{\scalebox{0.7}{\bfseries You might test that assumption at your convenience}\footnote{\textit{Captain Picard, Star Trek: The Next Generation}}}
\vskip2\baselineskip
\begin{dfn}[Differential Equation]
	Given $y\in C^m(I),\quad I\subseteq\R$ we define a \textit{differential equation of order} $m$ the following
	\begin{equation*}
		F(x,y,y',\cdots,y^{(m)})=0
	\end{equation*}
	If the equation can be rewritten as follows
	\begin{equation*}
		\derivative[m]{y}{x}=f(x,y,\cdots,y^{(m-1)})
	\end{equation*}
	It's called in \textit{normal form}.\\
	If only total derivatives appear, the differential equation is called \textit{ordinary}, whereas if also partial derivatives appear, the differential equation is called \textit{partial}.\\
\end{dfn}
\begin{thm}[Reduction of Order]
	Given an ODE (ordinary differential equation) of order $m$, one can reduce the order of the equation through a mapping to $\R^m$, where
	\begin{equation*}
		(y,y',\cdots,y^{(m)})\mapsto y^\mu(x)=\begin{pmatrix}y(x)\\y'(x)\\\vdots\\y^{(m)}(x)\end{pmatrix}
	\end{equation*}
	And $f(x,y,\cdots,y^{(m)})\mapsto f^\mu(x,y^\mu)$.\\
	The equation becomes a first order differential equation
	\begin{equation*}
		\derivative{y^\mu}{x}=f^\mu(x,y^\mu)
	\end{equation*}
	If $f$ is linear, this system can be expressed in matrix form, where
	\begin{equation*}
		\derivative{y^\mu}{x}=A^\mu_\nu y^\mu
	\end{equation*}
\end{thm}
\begin{dfn}[Cauchy Problem]
	A \textit{Cauchy} or \textit{initial value problem} is defined as the following system
	\begin{equation}
		\left\{ \begin{aligned}
				y'(x)&=f(x,y(x))\\
				y(x_0)&=y_0
		\end{aligned}\right.
		\label{eq:cauchyprob}
	\end{equation}
	Where $y_0,t_0\in\R$ are known values
\end{dfn}
\begin{thm}
	Given the Cauchy problem \eqref{eq:cauchyprob} we say that $y(x)\in C^1$ is a solution of the system if and only if is a continuous solution of the following integral equation
	\begin{equation}
		y(x)=y_0+\int_{x_0}^{x}f(s,y(s))\diff s
		\label{eq:cauchyprobsol}
	\end{equation}
\end{thm}
\begin{proof}
	Supposing $y\in C^1$ we have that for the fundamental theorem of integral calculus
	\begin{equation*}
		y(x)=y(x_0)+\int_{x_0}^{x}y'(s)\diff s=y_0+\int_{x_0}^{x}f(s,y(s))\diff s
	\end{equation*}
	Instead, considering $f\circ y$ we have that the new composed function must be continuous, therefore we have for the fundamental theorem of integral calculus that $y(x)\in C^1$ and that
	\begin{equation*}
		y(x_0)=y_0+\int_{x_0}^{x_0}f(s,y(s))\diff s=y_0
	\end{equation*}
	And therefore
	\begin{equation*}
		\derivative{y}{x}=\derivative{x}\int_{x_0}^{x}f(s,y(s))\diff s=f(x,y(x))
	\end{equation*}
	Which gives back the thesis
\end{proof}
\begin{thm}[Picard-Lindelöf-Cauchy Existence]
	Let $A\subset\R^2$ with $f:A\fto\R$ a continuous function which defines an ODE $\eqref{eq:cauchyprob}$ where
	\begin{equation*}
		y(x)=y_0+\int_{x_0}^{x}f(s,y(s))\diff s
	\end{equation*}
	Let $a,b>0$ and take $R=[x_0-a,x_0+a]\times[y_0-b,y_0+b]=I_a\times J_b\subset A$. Supposing $f$ Lipschitz continuous on the second variable, i.e.
	\begin{equation*}
		\exists L>0\st\abs{f(x,u)-f(x,w)}\le L\abs{u-w}\quad \forall (x,u),(x,w)\in R
	\end{equation*}
	Then, if $M=\max_R\abs{f(x,y)}$
	\begin{equation*}
		\exists\epsilon>0\st\exists!y(x)\in C^1(B_\epsilon(y_0))
	\end{equation*}
	Where $\epsilon=\min\left\{ a,b/M,1/L \right\}$\\
	Therefore, there is an unique local solution $\cc{y}(x)$ to the ODE
\end{thm}
\begin{proof}
	We firstly define a Banach space $(X,\unorm{\cdot})$ where $X:=\left\{ \derin{y\in C\left( I_\epsilon \right)}\unorm{y(x)-y_0}\le b \right\}$.\\
	Define the \textit{Picard operator} as follows
	\begin{equation*}
		\opr{T}:X\fto X
	\end{equation*}
	Where
	\begin{equation*}
		\opr{T}(w)=y_0+\int_{x_0}^{x}f(s,w(s))\diff s=z(x)
	\end{equation*}
	We have
	\begin{equation*}
		\unorm{\opr{T}(y)-\opr{T}(z)}=\unorm{\opr{T}(y-z)}=\unorm{\int_{x_0}^{x}f(s,u(s))-f(s,v(s))\diff s}\le L\epsilon\unorm{u-v}\quad\forall u,v,y,z\in X
	\end{equation*}
	Since $x\in I_\epsilon$ is completely arbitrary we have therefore, putting $d(x,y)=\unorm{x-y}$
	\begin{equation*}
		d(y,z)=d(\opr{T}(y),\opr{T}(z))\le L\epsilon d(u,v)
	\end{equation*}
	Since $L\epsilon\in(0,1),\quad\opr{T}$ is a contractor and it has a single fixed point in $X$.\\
	Taking this fixed point as $\cc{y}(x)$ we have
	\begin{equation*}
		\opr{T}(\cc{y})=y_0+\int_{x_0}^{x}f(s,\cc{y}(s))\diff s=\cc{y}(x)
	\end{equation*}
	Which is our searched solution, and it's unique
\end{proof}
\begin{dfn}[Maximal and Global Solutions]
	Given $y(x)$ a solution to the Cauchy problem \eqref{eq:cauchyprob} we define a \textit{maximal solution} $y_m(x)$ a solution for which $y_m(x)=y(x)\quad\forall x\in I_\epsilon$ and which still solves the problem in a set $(a,b)\supset I_\epsilon$, and $(a,b)$ is the biggest set for which $y_m(x)$ solves the ODE.\\
	If $y_m$ is defined $\forall x\in\R$, the solution is called \textit{global}
\end{dfn}
\begin{thm}[Prolungability of Solutions]
	Given the ODE \eqref{eq:cauchyprob} and let $f:(a,b)\times\R\fto\R$. Let $c_1,c_2\in K\subset(a,b)$ with $K$ a compact set, such that
	\begin{equation*}
		\abs{f(x,y)}\le c_1+c_2\abs{y(x)}\quad x\in K,\ \forall y\in\R
	\end{equation*}
	Then the solution can be extended in the whole set $(a,b)$. The previous statement means that $f$ is \textit{sublinear} in the second variable
\end{thm}
\begin{thm}
	Let $y_m(x)$ be a maximal solution to \eqref{eq:cauchyprob} defined on $(a,b)$. Then $\forall K\subset A\quad\exists\delta>0\st\forall x\not\in(a+\delta,b-\delta),\ (x,y_m(x))\not\in K$
\end{thm}
\begin{thm}
	Taken $y(x)$ a solution to the ODE \eqref{eq:cauchyprob}. If $\exists c>0$ such that
	\begin{equation*}
		\abs{y(x)}\le c\quad\forall t\in A
	\end{equation*}
	Then $\exists y_m(x):(a,b)\fto\R$
\end{thm}
\begin{lem}[Peano-Gronwall Inequality]
	Let $\varphi:I\subset\R\fto\R$ be a function $\varphi\in C(I)$, if
	\begin{equation*}
		\abs{\varphi(t)}\le c+L\abs{\int_{t_0}^{t}\abs{\varphi(s)}\diff s}
	\end{equation*}
	Then the following inequality holds
	\begin{equation*}
		\abs{\varphi(t)}\le ce^{L\abs{t-t_0}}
	\end{equation*}
	Therefore, if $y'(t)=f(t,y)$ is an ODE, taken $f(t,y)$ Lipschitz continuous on the second variable and chosen $\varphi(t)=u(t)-v(t)$ where $u,v$ are two solutions, we have that
	\begin{equation*}
		\abs{f(t,u)-f(t,v)}\le ce^{L\abs{t-t_0}}
	\end{equation*}
\end{lem}
\section{Common Solving Methods}
%\begin{mtd}[Matrix Exponential]
%	Given the following two ODEs
%	\begin{equation}
%		\left\{ \begin{aligned}
%				y''(x)-y(x)&=0\\
%				y'(0)&=0\\
%				y(0)&=1
%		\end{aligned}\right.
%		\label{eq:mode1}
%	\end{equation}
%	And
%	\begin{equation}
%		\left\{ \begin{aligned}
%				w''(x)+2w'(x)+2w(x)&=0\\
%				w'(0)&=0\\
%				w(0)&=1
%		\end{aligned}\right.
%		\label{eq:mode2}
%	\end{equation}
%	We can write the following two in normal form
%	\begin{equation*}
%		\left\{ \begin{aligned}
%				y''(x)&=y(x)\\
%				y'(0)&=0\\
%				y(0)&=1
%	\end{aligned}\right.
%	\end{equation*}
%	And
%	\begin{equation*}
%		\left\{ \begin{aligned}
%				w''(x)&=-2\left( w'(x)+w(x) \right)\\
%				w'(0)&=0\\
%				w(0)&=1
%		\end{aligned}\right.
%	\end{equation*}
%	Since $f(x,y(x))$ is linear, transforming everything in matrix form we have
%	\begin{equation}
%		\derivative{y^\mu}{x}=\begin{pmatrix}0&1\\1&0\end{pmatrix}^\mu_\nu y^\nu(x)\quad\derivative{w^\mu}{x}=\begin{pmatrix}0&1\\-2&-2\end{pmatrix}^\mu_\nu w^\nu(x)
%		\label{eq:matrixformmode12}
%	\end{equation}
%\end{mtd}
\begin{dfn}[Separable ODE]
	Given $y'(t)=f(t,y(t))$ an ODE we define a \textit{separable ODE} a differential equation such that
	\begin{equation*}
		y'(t)=f(t,y(t))=g(t)h(y)
	\end{equation*}
\end{dfn}
\begin{mtd}[Separation of Variables]
	Suppose that we have the previous ODE, then, thanks to the Picard-Lindelöf-Cauchy theorem we can immediately say that
	\begin{enumerate}
	\item There is at least one local solution in $B_\epsilon(t_0)$ if $f\in C$, i.e. if $g,h\in C$
	\item There is a unique local solution in $B_\epsilon(t_0)$ if $\del_y f\in C$, i.e. if $h\in C^1$
	\end{enumerate}
	The first thing to do is finding all constant solution to the ODE, called the \textit{equilibrium solutions}.\\
	Then, a general solution can be found by direct integration, where we're integrating the following differential form
	\begin{equation*}
		\frac{1}{h(y)}\diff y=g(t)\diff t
	\end{equation*}
	Integrating, and putting $H(t), G(t)$ as the two primitives, we can say that
	\begin{equation*}
		H(y)=G(t)+H(y_0)-G(t_0)
	\end{equation*}
	Where $c=H(y_0)-G(t_0)$ is the integration constant in terms of the initial values, if given
\end{mtd}
\begin{eg}
	Take the following initial value problem
	\begin{equation*}
		\left\{\begin{aligned}
				w'(x)&=w^2(x)\\
				w(0)&=1
		\end{aligned}\right.
	\end{equation*}
	We have $f(x,w)=w^2(x)$ i.e. the equation is separable, therefore
	\begin{equation*}
		\int\frac{w'(x)}{w^2(x)}\diff x=\int_{}^{}\diff x-1
	\end{equation*}
	Therefore, integrating the LHS
	\begin{equation*}
		-\frac{1}{w(x)}=x-1
	\end{equation*}
	And
	\begin{equation*}
		w(x)=\frac{1}{1-x}
	\end{equation*}
\end{eg}
\begin{dfn}[Autonomous Differential Equation]
	Given a differential equation in normal form $y'(x)=f(x,y(x))$, it's said to be an \textit{autonomous differential equation} if
	\begin{equation*}
		f(x,y(x))=g(y(x))
	\end{equation*}
	Therefore
	\begin{equation}
		y'(x)=g(y(x))
		\label{eq:autonomousode}
	\end{equation}
	Note that it's a separable equation, and therefore, writing the primitive of $1/g(y)=G(y)$ we have that the general solution of this problem would be
	\begin{equation*}
		G(y)=x+c
	\end{equation*}
	Where $c\in\R$ is a constant
\end{dfn}
%\begin{eg}[Logistical Equation]
%	Taken the logistical differential equation
%	\begin{equation*}
%		\left\{ \begin{aligned}
%				y'(x)&=y(1-y)\\
%				y(t_0)=y_0
%		\end{aligned}\right.
%	\end{equation*}
%	We see that's it's an autonomous differential equation, since $\del_1f(x,y(x))=0$.\\
%	We find immediately that if $y_0=0$ there are already two stable solutions $y_1=0$ and $y_2=1$. In the other cases we have that if $y_0\in[0,1)\implies y(t)\in(0,1)$
%	The solution then will be
%	\begin{equation*}
%		\int_{}^{}\frac{1}{y(1-y)}\diff y=t+c
%	\end{equation*}
%	We use the partial fraction decomposition for the denominator, and we get
%	\begin{equation*}
%		\frac{1}{y(1-y)}=\frac{a}{y}+\frac{b}{(1-y)}\implies a-y(b-a)=1\quad\therefore\begin{dcases}a=-b\\a=1\end{dcases}
%	\end{equation*}
%\end{eg}
\begin{eg}
	Take the following initial value problem
	\begin{equation*}
		\left\{ \begin{aligned}
				y'(x)&=t\abs{y}\\
				y(0)&=y_0
		\end{aligned}\right.
	\end{equation*}
	This is obviously a separable ODE, and we have $f(t,y(t))=t\abs{y}$.\\
	We have that $f\in C(\R^2)$ therefore there exists at least one solution for this equation.\\
	Now, applying Picard-Lindelöf-Cauchy we can see that, $\forall [\alpha,\beta]\subset\R$ compact set
	\begin{equation*}
		\abs{f(t,y)-f(t,z)}=\abs{t\abs{y}-t\abs{z}}=\abs{t}\abs{\abs{y}-\abs{z}}\le\max\left\{ \abs{\alpha},\abs{\beta} \right\}\abs{y-z}
	\end{equation*}
	Taken $L=\max\left\{ \abs{\alpha},\abs{\beta} \right\}$ we have that $f$ is Lipschitz continuous in the second variable in every compact set $K\subset\R$ and therefore there exists a unique solution $y:K\fto\R$.\\
	Successively we move on to integrate directly the differential equation. We have
	\begin{equation*}
		\int_{y_0}^{y}\frac{\diff y}{\abs{y}}=\int_{t_0}^{t}t\diff t
	\end{equation*}
	I.e.
	\begin{equation*}
		\begin{aligned}
			\log\abs{\frac{y}{y_0}}&=\frac{1}{2}\left( t^2-t_0^2 \right)\\
			\abs{\frac{y}{y_0}}&=e^{\frac{1}{2}t_0^2}e^{\frac{1}{2}t^2}
		\end{aligned}
	\end{equation*}
	Since $t_0=0$ we get
	\begin{equation*}
		\abs{y}(t)=\abs{y_0}e^{\frac{1}{2}t^2}
	\end{equation*}
	The solutions depend on the sign of $y_0$ and we finally get
	\begin{equation*}
		y(t)=\begin{dcases}y_0e^{\frac{1}{2}t^2}&y_0>0\\y_0e^{-\frac{1}{2}t^2}&y_0<0\end{dcases}
	\end{equation*}
\end{eg}
\subsection{First Order Linear ODEs}
\begin{dfn}[Integrating Factor]
	An example of linear ODE is the following kind of equation
	\begin{equation}
		y'(t)=p(t)y(t)+q(t)\qquad p,q\in C(I),\ I\subseteq\R
		\label{eq:integratingfactor}
	\end{equation}
	This equation has a unique solution in all of $\R$, since $f(t,y)$ is Lipschitz continuous and sublinear in $\R$.\\
	We immediately know that the solution $y\in C^1$ exists since $f\in C(\R^2)$ and by Cauchy-Picard-Lindelöf we have
	\begin{equation*}
		\abs{f(t,y)-f(t,z)}=\abs{p(t)y(t)-p(t)z(t)}=\abs{p(t)}\abs{y-z}\le\unorm{p}\abs{y-z}
	\end{equation*}
	Taken $L=\unorm{p}$ we see immediately that $f$ satisfies the theorem and the solution is uniquely defined in every compact set $K\subset\R$.\\
	We also see that $f$ is sublinear, since
	\begin{equation*}
		\abs{f(t,y)}=\abs{p(t)y(t)+q(t)}\le\unorm{p}\abs{y(t)}+\unorm{q}
	\end{equation*}
	Which means that the solution can be extended in $\R$, and $\exists!y:\R\fto\R$ solution to the ODE.\\
	We now proceed to integrate the ODE, for which we will proceed with a different method than usual.\\
	Let $\mu(t)$ be what we will call the \textit{integrating factor}, defined as follows
	\begin{equation*}
		\mu(t)=e^{-\int p(t)\diff t}
	\end{equation*}
	Which has the property that $p(t)\mu(t)=\mu'(t)$
\end{dfn}
\begin{mtd}[Integrating Factor]
	Let $y'(t)=p(t)y(t)+q(t)$ be a linear ODE, we solve by finding the integrating factor $\mu(t)$ and then multiplying both sides of the equayion by the integrating factor and moving the $p(t)y(t)$ term from the RHS to the LHS
	\begin{equation*}
		\mu(t)y'(t)-p(t)\mu(t)y(t)=q(t)\mu(t)
	\end{equation*}
	By definition of the integrating factor we have that $p(t)\mu(t)=-\mu'(t)$ and therefore
	\begin{equation*}
		\mu(t)y'(t)+\mu'(t)y(t)=q(t)\mu(t)
	\end{equation*}
	Using the chain rule we get
	\begin{equation*}
		\derivative{t}\left( \mu(t)y(t) \right)=\mu(t)q(t)
	\end{equation*}
	Integrating
	\begin{equation*}
		\mu(t)y(t)=\int_{}^{}\mu(t)q(t)\diff t+c
	\end{equation*}
	And therefore, the final general solution will be the following
	\begin{equation*}
		y(t)=\frac{1}{\mu(t)}\int_{}^{}\mu(t)q(t)\diff t+\frac{c}{\mu(t)}
	\end{equation*}
\end{mtd}
\begin{eg}
	Take the following initial value problem
	\begin{equation*}
		\left\{\begin{aligned}
			y'(t)&=\frac{1}{t}y(t)+3t^3\\
			y(-1)&=2
		\end{aligned}\right.
	\end{equation*}
	We immediately see the previous definition, therefore we can immediately say that since $f$ is Lipschitz continuous and sublinear the solution to the problem will be uniquely defined on all of $\R$.\\
	For integrating this ODE we use the integrating factor method, taking $p(t)=t^{-1}$ and $q(t)=3t^3$.\\
	By definition, we have $\mu(t)=\abs{t}$, and therefore, since $t<0$, $\mu(t)=-t^{-1}$, which gives
	\begin{equation*}
		-\frac{1}{t}y(t)=-3\int_{}^{}t^2\diff t+c
	\end{equation*}
	Therefore
	\begin{equation*}
		y(t)=t^4-ct
	\end{equation*}
	Applying the initial condition
	\begin{equation*}
		y(-1)=1+c=2\implies c=1
	\end{equation*}
	And therefore the final, unique solution of the initial value problem is the following
	\begin{equation*}
		y(t)=t^4-t
	\end{equation*}
\end{eg}
\begin{thm}[General Solutions]
	Taken $y'(x)+p(x)y(x)=q(x)$ a linear non homogeneous ODE, we define the \textit{associated homogeneous ODE} as the equation
	\begin{equation*}
		z'(x)+p(x)z(x)=0
	\end{equation*}
	The solution of the complete ODE will be given by the sum of the particular solution $\cc{y}(x)$ and the homogeneous solution $z(x)$
\end{thm}
\begin{proof}
	Take $y(x)$ a solution of the equation and $\cc{y}(x)$ the particular solution of the ODE, we then have, since both are solutions
	\begin{equation*}
		\left\{ \begin{aligned}
				y'(x)+p(x)y(x)&=q(x)\\
				\cc{y}'(x)+p(x)\cc{y}(x)&=q(x)
		\end{aligned}\right.
	\end{equation*}
	Subtracting term a term we have
	\begin{equation*}
		(y'(x)-\cc{y}(x))+p(x)(y(x)-\cc{y}(x))=0
	\end{equation*}
	Chosen $y'(x)-\cc{y}(x)=z(x)$ we have that
	\begin{equation*}
		z'(x)+p(x)z(x)=0
	\end{equation*}
	Therefore $z(x)$ is a solution, but it's also the solution to the associated homogeneous ODE, and therefore, inverting for the general solution $y(x)$, we have
	\begin{equation*}
		y(x)=\cc{y}(x)+z(x)
	\end{equation*}
\end{proof}
\begin{eg}
	Taken the following initial value problem
	\begin{equation*}
		\left\{ \begin{aligned}
				y'(x)+\frac{2}{x}y(x)&=\frac{1}{x^2}\\
				y(-1)&=2
		\end{aligned}\right.
	\end{equation*}
	We want to find the general integral of the equation, $y(x)$. We begin by solving the associated homogeneous equation
	\begin{equation*}
		z'(x)+\frac{2}{x}z(x)=0
	\end{equation*}
	In order to integrate this we use the integrating factor $\mu(x)$, where
	\begin{equation*}
		\mu(x)=\exp\left( 2\int_{-1}^{x}\frac{1}{x}\diff x \right)=\exp\left( 2\log(\abs{x}) \right)=\exp\left( \log\left( x^2 \right) \right)=x^2
	\end{equation*}
	We multiply by the integrating factor, and get
	\begin{equation*}
		\derivative{x}(x^2z(x))=0
	\end{equation*}
	Therefore
	\begin{equation*}
		x^2z(x)=c\implies z(x)=\frac{c}{x^2}
	\end{equation*}
	The particular solution $\cc{y}(x)$ will be
	\begin{equation*}
		\cc{y}(x)=\frac{1}{\mu(x)}\int_{x_0}^{x}\frac{1}{x^2}\mu(x)\diff x
	\end{equation*}
	Therefore
	\begin{equation*}
		\cc{y}(x)=\frac{1}{x^2}\int_{-1}^{x}\diff x=\frac{1}{x}
	\end{equation*}
	The general solution will be $y(x)=\cc{y}(x)+z(x)$, and we finally get
	\begin{equation*}
		y(x)=\frac{c}{x^2}+\frac{1}{x}
	\end{equation*}
	Imposing the initial condition we have
	\begin{equation*}
		y(-1)=c-1=2\implies c=3
	\end{equation*}
	Therefore
	\begin{equation*}
		y(x)=\frac{3}{x^2}+\frac{1}{x}
	\end{equation*}
	Note how we could have found the general integral directly using the formula
	\begin{equation*}
		y(x)=\frac{c}{x^2}+\frac{1}{x^2}\int_{-1}^{x}\diff x=\frac{3}{x^2}+\frac{1}{x^2}\int_{-1}^{x}\diff x
	\end{equation*}
\end{eg}
\begin{mtd}[Variation of Constants]
	Given a linear ODE $y'(t)+a(t)y=f(t)$, with $a,f\in C(I),\ I\subseteq\R$ suppose that $z(t)$ is the solution of the homogeneous equation, we have therefore
	\begin{equation*}
		z(t)=\frac{c}{\mu(t)}=c\exp\left( \int a(t)\diff t \right)
	\end{equation*}
	We can find a particular solution through \textit{variation of constants}, i.e. we suppose $c=c(t)$, and we have
	\begin{equation*}
		\cc{y}(t)=c(t)\exp\left( \int_{}^{}a(t)\diff t \right)
	\end{equation*}
	Imposing this on the differential equation, we have
	\begin{equation*}
		\cc{y}'(t)=c'(t)\exp\left( -\int_{}^{}a(t)\diff t \right)-a(t)c(t)\exp\left( -\int_{}^{}a(t) \right)\diff t
	\end{equation*}
	And therefore
	\begin{equation*}
		\frac{1}{\mu(t)}\left( c'(t)-a(t)c(t) \right)+\frac{a(t)c(t)}{\mu(t)}=f(t)
	\end{equation*}
	Therefore
	\begin{equation*}
		\frac{c'(t)}{\mu(t)}=f(t)\implies c(t)=\int_{}^{}f(t)\mu(t)\diff t
	\end{equation*}
	Which implies
	\begin{equation*}
		\cc{y}(t)=\frac{1}{\mu(t)}\int_{}^{}f(t)\mu(t)\diff t
	\end{equation*}
	We then recover the previous formula by adding the particular and homogeneous solutions
	\begin{equation*}
		y(t)=\frac{c}{\mu(t)}+\frac{1}{\mu(t)}\int_{}^{}f(t)\mu(t)\diff t
	\end{equation*}
\end{mtd}
\subsection{Second Order Linear ODEs}
\begin{dfn}[Initial Value Problem for 2nd Order Linear ODEs]
	Given a second order linear differential equation $y''(x)+a(x)y'(x)+b(x)y(x)=f(x)$ we define the \textit{initial value problem} for such differential equation as follows
	\begin{equation*}
		\left\{ \begin{aligned}
				y''(x)+a(x)y'(x)+b(x)y(x)&=f(x)\\
				y'(x_0)&=y'_0\\
				y(x_0)&=y_0
		\end{aligned}\right.
	\end{equation*}
	This kind of problem, if $a,b,f\in C(I)\ I\subseteq\R$ is said to be \textit{well defined}
\end{dfn}
\begin{thm}[Space of Solutions]
	Given a second order linear ODE, the solutions of the homogeneous equation span a vector space of dimension 2 $\mathcal{S}$, and the general solution of the homogeneous equation has the following form
	\begin{equation*}
		z_g(x)=c_1z_1(x)+c_2z_2(x)\in\mathcal{S}
	\end{equation*}
	I.e., if we define a matrix $W_{\mu\nu}(t)$ as follows
	\begin{equation*}
		W_{\mu\nu}(t)=\begin{pmatrix}z_1(t)&z_2(t)\\z_1'(t)&z_2'(t)\end{pmatrix}
	\end{equation*}
	If $z_1,z_2$ solve the differential equation, the matrix $W_{\mu\nu}(t)$ is non-singular
\end{thm}
\begin{proof}
	Suppose $z_1,z_2\in C^2(I)$ are two solutions. If they're linearly inddependent we have that they solve the following system with $c_1,c_2\in\R\setminus\{0\}$
	\begin{equation*}
		\left\{ \begin{aligned}
				c_1z_1(x)+c_2z_2(x)&=0\\
				c_1z'_1(x)+c_2z'_2(x)&=0
		\end{aligned}\right.
	\end{equation*}
	This system is solvable only if the determinant of the matrix $W_{\mu\nu}$ is non zero, and therefore the two functions must be linearly independent and span the vector space of solutions $\mathcal{S}$
\end{proof}
\begin{mtd}[Characteristic Polynomial]
	Take the following linear homogeneous ODE of order 2
	\begin{equation*}
		z''(x)+az'(x)+bz(x)=0\qquad a,b\in\R
	\end{equation*}
	We begin by ``guessing'' a solution in exponential form $z(x)=e^{\lambda x}$, where $\lambda\in\Cf$. Therefore, reinserting into the ODE, we have
	\begin{equation*}
		e^{\lambda x}\left( \lambda^2+a\lambda+b \right)=0
	\end{equation*}
	Since the exponential is never zero $\forall x\in\R,\ \forall\lambda\in\Cf$ we have that the polynomial must be zero, and therefore we get two roots
	\begin{equation*}
		\lambda_{1,2}=-\frac{a}{2}\pm\frac{1}{2}\sqrt{a^2-4b}
	\end{equation*}
	We can now discern 3 cases.\\
	Case 1), $a^2-4b>0$\\
	There are two real roots $\lambda_1,\ \lambda_2$, and therefore we have
	\begin{equation*}
		z(x)=c_1e^{\lambda_1x}+c_2e^{\lambda_2x}
	\end{equation*}
	Case 2), $a^2-4b<0$\\
	There are two complex conjugate roots $\lambda,\ \cc{\lambda}$, and the solution will be
	\begin{equation*}
		z(x)=\real\left( c_1e^{\lambda x}+c_2e^{\cc{\lambda}x} \right)=c_1e^{\real(\lambda)x}\cos(\imaginary(\lambda)x)+c_2e^{\real(\lambda)x}\sin(\imaginary(\lambda)x)
	\end{equation*}
	Or
	\begin{equation*}
		z(x)=Ae^{\real(\lambda)x}\cos(\imaginary(\lambda)+\varphi)
	\end{equation*}
	With $A,\varphi\in\R$
	Case 3) $a^2-4b=0$\\
	There is only one root $\lambda=a/2$ with multiplicity $2$. In order to find the general integral $z(x)$ we impose the variation of constants, and we have
	\begin{equation*}
		z(x)=c(x)e^{\lambda x}
	\end{equation*}
	And
	\begin{equation*}
		\begin{aligned}
			z'(x)&=c'(x)e^{\lambda x}+\lambda c(x)e^{\lambda x}=e^{\lambda x}(c'(x)+\lambda c(x))\\
			z''(x)&=c''(x)e^{\lambda x}+\lambda c'(x)e^{\lambda x}+\lambda^2 c(x)e^{\lambda x}=e^{\lambda x}(c''(x)+\lambda c'(x)+\lambda^2c(x))
		\end{aligned}
	\end{equation*}
	Since $\lambda$ is a root of the characteristic polynomial, we end up with
	\begin{equation*}
		c''(x)=0\implies c(x)=c_2+c_1t
	\end{equation*}
	Therefore
	\begin{equation*}
		z(x)=e^{\lambda t}(c_1+c_2t)=e^{-\frac{a}{2}t}(c_1+c_2t)
	\end{equation*}
\end{mtd}
\begin{mtd}[Similarity Method]
	Another method for finding a particular solution is the similarity method. Given the following linear ODE
	\begin{equation*}
		y''(x)+ay'(x)+by(x)=f(x)
	\end{equation*}
	The particular solution of this problem can be found using the shape of $f(x)$.\\
	Case 1) $f(x)\in\R_n[x]$\\
	In this case we will suppose that the solution will be a polynomial, and we can discern 3 cases
	\begin{equation*}
		\cc{y}(x)=\begin{dcases}
			q(x)&b\ne0\\
			xq(x)&b=0,\ a\ne0\\
			x^2q(x)&b=0,\ a=0
		\end{dcases}
	\end{equation*}
	Where $q(x)\in\R_n[x]$\\
	Case 2) $f(x)=Ae^{\lambda x}$, $\lambda\in\Cf$\\
	In this case we will suppose the particular solution as the real part of a complex exponential, where
	\begin{equation*}
		\cc{y}(x)=\real\left( \tilde{y}(x) \right)=\gamma(x)\real\left( e^{\lambda x} \right)
	\end{equation*}
	Where $\gamma(x)\in C^2(I),\ I\subset\R$ is some unknown function. From this we will have, deriving twice
	\begin{equation*}
		\begin{aligned}
			\tilde{y}'(x)&=e^{\lambda x}\left( \gamma'(x)+\lambda\gamma(x) \right)\\
			\tilde{y}''(x)&=e^{\lambda x}\left( \lambda^2\gamma(x)+2\lambda\gamma'(x)+\gamma''(x) \right)
		\end{aligned}
	\end{equation*}
	Substituting into the differential equation we obtain
	\begin{equation*}
		\gamma''(x)+(2\lambda+a)\gamma'(x)+(\lambda^2+b\lambda+a)\gamma(x)=A
	\end{equation*}
	We get again 3 different cases\\
	a) $\lambda^2+a\lambda+b=0$\\
	In this case we have
	\begin{equation*}
		\gamma(x)=\frac{A}{\lambda^2+a\lambda+b}\implies \tilde{y}(x)=\frac{A}{\lambda^2+a\lambda+b}e^{\lambda x}
	\end{equation*}
	b) $\lambda^2+a\lambda+b=0,\ 2\lambda+a\ne0$\\
	In this gase instead we have
	\begin{equation*}
		\gamma'(x)=\frac{A}{2\lambda+a}\implies \tilde{y}(x)=\frac{A}{2\lambda+a}xe^{\lambda x}
	\end{equation*}
	c) $\lambda^2+a\lambda+b=0,\ 2\lambda+a=0$\\
	Lastly we have
	\begin{equation*}
		\gamma''(x)=A\implies \tilde{y}(x)=\frac{A}{2}x^2e^{\lambda x}
	\end{equation*}
	The solution for the differential equation will lastly be either the real (or imaginary, if there is a sine in the RHS) part of the function we found.
\end{mtd}
\begin{eg}
	We will use the similarity method to solve the following differential equation
	\begin{equation*}
		y''(x)+2y'(x)+3y(x)=2e^{3x}
	\end{equation*}
	We see immediately that $f(x)=2e^{3x}$ so we choose a particular solution $\cc{y}(x)$ with the following form
	\begin{equation*}
		\cc{y}(x)=ke^{3x}
	\end{equation*}
	We therefore have
	\begin{equation*}
		\begin{aligned}
			\cc{y}'(x)&=3ke^{3x}\\
			\cc{y}''(x)&=9ke^{3x}
		\end{aligned}
	\end{equation*}
	Substituting
	\begin{equation*}
		18k=2\implies k=\frac{1}{9}
	\end{equation*}
	Therefore our searched solution will be the following
	\begin{equation*}
		\cc{y}(x)=\frac{1}{9}e^{3x}
	\end{equation*}
\end{eg}
\begin{eg}
	Let's now solve the following differential equation
	\begin{equation*}
		y''(x)+2y'(x)-3y(x)=2e^{-3x}
	\end{equation*}
	Following the same approach, we see that the characteristic polynomial is null, and we end up with a contradiction, therefore we find a solution of the following kind
	\begin{equation*}
		\cc{y}(x)=kxe^{-3x}
	\end{equation*}
	Deriving the particular solution, we have
	\begin{equation*}
		\begin{aligned}
			\cc{y}'(x)&=ke^{-3x}\left( 1-3x \right)\\
			\cc{y}''(x)&=3ke^{-3x}\left( 3x-2 \right)
		\end{aligned}
	\end{equation*}
	Substituting
	\begin{equation*}
		\begin{aligned}
			3k(3x-2)&+2k(1-3x)-3kx=2\\
			-4k&=2\\
			k&=-\frac{1}{2}
		\end{aligned}
	\end{equation*}
	Therefore the particular solution we're searching for is, finally
	\begin{equation*}
		\cc{y}(x)=-\frac{x}{2}e^{-3x}
	\end{equation*}
\end{eg}
\begin{mtd}[Variation of Constants]
	Another method that might be used to solve a second order linear differential equation is the method of variation of constants.\\
	Given the following differential equation and the associated homogeneous equation
	\begin{equation*}
		\begin{aligned}
			y''(x)+ay'(x)+by(x)&=f(x)\\
			z''(x)+az'(x)+bz(x)&=0
		\end{aligned}
	\end{equation*}
	Suppose that $z_1(x),z_2(x)$ are two linearly independent homogeneous solutions, and
	\begin{equation*}
		z(x)=c_1z_1(x)+c_2z_2(x)
	\end{equation*}
	We search for a particular solution supposing that the two constants are actually functions, and we have
	\begin{equation*}
		\cc{y}(x)=c_1(x)z_1(x)+c_2(x)z_2(x)
	\end{equation*}
	Due to the necessity of a non-singular Wronskian determinant, we impose the following condition
	\begin{equation*}
		c_1'(x)z_1(x)+c_2'(x)z_2(x)=0
	\end{equation*}
	Deriving two times the particular solution and then substituting in the original equation we end up with the following system
	\begin{equation*}
		\left\{\begin{aligned}
			c_1'(x)z_1(x)+c_2'(x)z_2(x)&=0\\
			c_1'(x)z_1'(x)+c_2'(x)z_2'(x)&=f(x)
		\end{aligned}\right.
	\end{equation*}
	The necessity for the linear independence of the solutions is clear here, since the system is solvable if and only if the Wronskian determinant is zero.\\
	We solve the system by substitution, and we have
	\begin{equation*}
		\left\{ \begin{aligned}
				c_1'(x)&=-c_2'(x)\frac{z_2(x)}{z_1(x)}\\
				c_2'(x)&=\frac{f(x)-c_1'(x)z_1'(x)}{z_2'(x)}
		\end{aligned}\right.
	\end{equation*}
	Therefore
	\begin{equation*}
		\left\{ \begin{aligned}
				c_1'(x)&=-\frac{z_1(x)}{z_1(x)z_2'(x)-z_2(x)z_1'(x)}f(x)\\
				c_2'(x)&=\frac{z_2(x)}{z_1(x)z_2'(x)-z_2(x)z_1'(x)}f(x)
		\end{aligned}\right.
	\end{equation*}
	Written in terms of the Wronskian determinant $\det_{\mu\nu}W_{\mu\nu}(x)$ we finally have as particular solution
	\begin{equation*}
		\cc{y}(x)=z_2(x)\int_{}^{}\frac{z_2(x)}{\det_{\mu\nu}W_{\mu\nu}(x)}f(x)\diff x-z_1(x)\int_{}^{}\frac{z_1(x)}{\det_{\mu\nu}W_{\mu\nu}(x)}f(x)\diff x+c
	\end{equation*}
	And the general solution of the differential equation will be the following
	\begin{equation*}
		y(x)=z_2(x)\int_{x_0}^{x}\frac{z_2(x)}{\det_{\mu\nu}W_{\mu\nu}(x)}f(x)\diff x-z_1(x)\int_{x_0}^{x}\frac{z_1(x)}{\det_{\mu\nu}W_{\mu\nu}(x)}f(x)\diff x+c_1z_1(x)+c_2z_2(x)
	\end{equation*}
\end{mtd}
%\section{Green Identities and Kelvin Transform}
\end{document}
