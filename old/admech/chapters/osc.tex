\documentclass[../admech.tex]{subfiles}
\begin{document}
\section{Stability and Free Oscillations}
One of the most important topics of mechanics is the idea of stability and small oscillations around these ``stable points''. The idea of stable and unstable points stems from the main idea of equilibrium.
\begin{dfn}[Mechanical Equilibrium, Stability and Instability]
	Given a system interacting with a certain potential $\pot$, equilibrium points of such are defined as the \emph{critical points} of the potential $\pot$. The stability of such points is then defined by their nature, where minimums of the potential function are said to be \emph{stable equilibrium points} and maximums as \emph{unstable equilibrium points} of the system.
\end{dfn}
In order to evaluate the stability of equilibrium points, given the definition, immediately uses optimization theory applied on the potential function.\\
With this, for a system with $n$ degrees of freedom, given a critical point $q^\mu_0$ of $\pot(q^\mu)$, and having defined the Hessian matrix of $\pot$ as $\del^2_{\mu\nu}\pot=\pot_{\mu\nu}$ one has two main results.
\begin{enumerate}
\item If $\pot_{\mu\nu}(q^\mu_0)$ is positive definite, the critical point is a minimal and therefore a stable equilibrium point
\item If $\pot_{\mu\nu}(q^\mu_0)$ is negative definite, then it's a maximal and therefore an unstable equilibrium point
\end{enumerate}
In case that the Hessian matrix is undefined, nothing can be derived through this criterion.\\
Note that using Sylvester's criterion for determining the signature of the eigenvalues of a matrix, in two dimensions this reduces to evaluating sign of the determinant of $\pot_{\mu\nu}$ and the sign of the first entry of the matrix, $\pot_{11}$, where
\begin{enumerate}
\item $\abs{\pot_{\mu\nu}}>0$ and $\pot_{11}>0$ imply a stable equilibrium point
\item $\abs{\pot_{\mu\nu}}>0$ and $\pot_{11}<0$ imply an unstable equilibrium point
\item $\abs{\pot_{\mu\nu}}<0$ imply a saddle (unstable) point of equilibrium
\item $\abs{\pot_{\mu\nu}}=0$ implies that $\pot_{\mu\nu}$ is indefinite and therefore no clear conclusion can be given on the type of equilibrium at the considered point
\end{enumerate}
\subsection{Free Oscillations in 1 Dimension}
Suppose having some system with one degree of freedom interacting in a field with potential $\pot(q)$, which has $q_0$ as a local minimum point. Therefore we have
\begin{equation}
	\left( \dv{\pot}{q} \right)_{q_0}=0
	\label{eq:critpoint}
\end{equation}
For evaluating small oscillations we expand with a power series to the second order this potential around $q_0$
\begin{equation}
	\pot(q)\simeq\pot(q_0)+\frac{1}{2}\left( \dv[2]{\pot}{q} \right)_{q_0}(q-q_0)^2+\order{(q-q_0)^3}
	\label{eq:potapprox}
\end{equation}
Imposing $\pot(q_0)=0$ and writing $\pot''(q_0)=k$ we have that our potential can be approximated to a harmonic oscillator potential up to second order, yielding
\begin{equation}
	\pot(q)\approx\frac{1}{2}k(q-q_0)^2
	\label{eq:approxharmpot}
\end{equation}
Substituting $q-q_0=x$ we can then write the approximate harmonic Lagrangian as the usual harmonic oscillator Lagrangian
\begin{equation}
	\lag(x,\dot{x})=\frac{1}{2}m\dot{x}^2-\frac{1}{2}kx^2
	\label{eq:osclag1}
\end{equation}
Deriving the Lagrangian we get the usual Euler-Lagrange equations for the harmonic oscillator, and substituting $\omega^2=k/m$ we get
\begin{equation}
	\dv{t}\pdv{\lag}{\dot{q}}-\pdv{\lag}{q}=\ddot{x}+\omega^2x=0
	\label{eq:osceleq1}
\end{equation}
This differential equation has two possible solutions
\begin{equation}
	\begin{aligned}
		x(t)&=c_1\cos(\omega t)+c_2\sin(\omega t)\\
		x(t)&=A\cos(\omega t+\phi)
	\end{aligned}
	\label{eq:oscsol1}
\end{equation}
Where $c_1,c_2,A,\phi\in\R$. The second solution can be obtained using the formula
\begin{equation*}
	\cos(\omega t+\phi)=\cos(\omega t)\cos\phi-\sin(\omega t)\sin\phi
\end{equation*}
Which gives us the expression of $A$ and $\phi$ in terms of the two constants $c_1,c_2$ as follows
\begin{equation}
	A=\sqrt{c_1^2+c_2^2}\qquad\tan\phi=-\frac{c_2}{c_1}
	\label{eq:Aphic1c2}
\end{equation}
$A$ is known as teh \emph{amplitude} of the motion, while $\phi$ is the initial value of the phase with frequency $\omega$.\\
Writing the mechanical energy of an oscillatory system and substituting the solution $x(t)$ we have
\begin{equation}
	E=\frac{1}{2}m\omega^2A^2
	\label{eq:energypropamp}
\end{equation}
Which gives us $E\propto A^2$. Noting also how $x(t)$ is shaped, we can also write it using Euler's identity as the real part of a complex function, as
\begin{equation}
	x(t)=\real\left\{ ae^{i\omega t} \right\}
	\label{eq:complexoscsol}
\end{equation}
Where we set $a=Ae^{i\phi}$ as the complex amplitude, which has the amplitude as modulus and the phase as argument.
\section{Free Oscillations in $n$ Degrees of Freedom}
Working analogously in $n$ dimensions, for studying these oscillations we choose a certain critical point $q_0^\mu$ of the potential, using $x^\mu=q^\mu-q_0^\mu$ we approximate the potential as
\begin{equation}
	\pot(x^\mu)\approx\frac{1}{2}\del^2_{\mu\nu}\pot(0)x^\mu x^\nu=\frac{1}{2}k_{\mu\nu}x^\mu x^\nu
	\label{eq:approxpotndeg}
\end{equation}
The matrix $k_{\mu\nu}$ is the Hessian matrix of $\pot$, and therefore, since $\pot\in C^2(\R^n)$, $k_{\mu\nu}=k_{\nu\mu}$.\\
Analogously, for the kinetic energy we have
\begin{equation}
	T=\frac{1}{2}a_{\mu\nu}(q_0^\gamma)\dot{q}^\mu\dot{q}^\nu
	\label{eq:osckinndeg}
\end{equation}
Writing $a_{\mu\nu}(q_0^\gamma)$ as $m_{\mu\nu}$ and using the fact that $T\in C^2(\R^n)$ we have the linearized Lagrangian for a system with $n$ degrees of freedom
\begin{equation}
	\lag(\dot{x}^\mu,x^\mu)=\frac{1}{2}\left( m_{\mu\nu}\dot{x}^\mu\dot{x}^\nu-k_{\mu\nu}x^\mu x^\nu \right)
	\label{eq:linlagnosc}
\end{equation}
For finding the equations of motion of this system we write the total differential of the Lagrangian, getting
\begin{equation}
	\dd\lag=m_{\mu\nu}\dot{x}^\mu\dd\dot{x}^\nu-k_{\mu\nu}x^\mu\dd x^\nu=\pdv{\lag}{\dot{x}^\nu}\dd\dot{x}^\nu+\pdv{\lag}{x^\nu}\dd x^\nu
	\label{eq:linlagdif}
\end{equation}
Therefore, the searched equations of motion will be a system of ODEs, which will be
\begin{equation}
	m_{\mu\nu}\ddot{x}^\mu+k_{\mu\nu}x^\mu=0
	\label{eq:odeoscndeg}
\end{equation}
Keeping in mind what we found before for oscillating system we suppose the solution as a complex vector where
\begin{equation*}
	x^\mu(t)=A^\mu e^{i\omega t}\quad A^\mu\in\Cf^n
\end{equation*}
Deriving and substituting back this function we get a linear system with $A^\mu$ as our unknown vector
\begin{equation}
	\left( -\omega^2m_{\mu\nu}+k_{\mu\nu} \right)A^\mu=0
	\label{eq:oscsystem}
\end{equation}
The solution of this system boils down to searching simultaneous eigenvectors and eigenvalues for the two matrices, and therefore this is possible if and only if
\begin{equation}
	\det\abs{-\omega^2m_{\mu\nu}+k_{\mu\nu}}=0
	\label{eq:oscsistsolcond}
\end{equation}
Since both $m_{\mu\nu},k_{\mu\nu}$ are symmetric matrices we must have that $\omega^2\in\R$, i.e. that all the eigenvalues of the combined system are real. This also implies that, if $\Delta_a^\mu$ is the $a-$th minor of the composite matrix $d_{\mu\nu}=-\omega^2m_{\mu\nu}+k_{\mu\nu}$, then $A^\mu\propto\Delta^\mu_a$, which implies that the general solution therefore is
\begin{equation}
	x_a^\mu(t)=\Delta^\mu_aC_ae^{i\omega_at}
	\label{eq:athsol}
\end{equation}
Where $x_a^\mu$ is the solution associated with the $a-$th eigenvalue $\omega_a$. The complete solution will therefore be a real superposition of the previous one, which boils down to the following function
\begin{equation}
	x^\mu(t)=\sum_{a=1}^n\Delta^\mu_a\real\left\{ C_ae^{i\omega_at} \right\}=\sum_{a=1}^n\Delta^\mu_a\Theta_a(t)
	\label{eq:completesol}
\end{equation}
Where with $\Theta_a(t)$ we have indicated a simple periodic oscillation with frequency $\omega_a$. These simple oscillations are called the \emph{normal oscillations} of the system and they are linearly independent from one another, forming an orthogonal basis in which the matrix $d_{\mu\nu}$ is diagonal. Obviously they also solve the ODE
\begin{equation}
	\ddot{\Theta}_a+\omega^2_a\Theta_a=0
	\label{eq:normODE}
\end{equation}
With a basis transformation, the linearized Lagrangian \eqref{eq:linlagnosc} becomes as follows
\begin{equation}
	\lag\left( \dot{\Theta},\Theta \right)=\sum_a\frac{1}{2}m_a\left( \dot{\Theta}_a^2-\omega^2\Theta_a^2 \right)
	\label{eq:diaglagosc}
\end{equation}
These basis vectors can be immediately normalized to a new basis, where $Q_a=\sqrt{m_a\Theta_a}$, which changes slightly our Lagrangian to the new version
\begin{equation}
	\lag(\dot{Q},Q)=\frac{1}{2}\sum_{a=1}^n\left( \dot{Q}_a^2-\omega^2_aQ_a^2 \right)
	\label{eq:normosclag}
\end{equation}
\end{document}
